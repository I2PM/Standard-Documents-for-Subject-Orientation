% !TeX spellcheck = en_US
% Review Christian Stary January 2020
% Merge mit Thomas Schaller Pfad Chapter 4

% testcomment

\RequirePackage{snapshot}

\documentclass[11pt, showtrims, final, oldfontcommands]{memoir}
% Resolve conflict between functions in class memoir and package changes
\let\classadded\added 	 % save the class’ definition
\let\added\relax		 % ‘undefine’ \foo
\let\deleted\relax    % neutralize \deleted command
\let\comment\relax    % neutralize \comment command



\usepackage[T1]{fontenc}
\usepackage[textsize=footnotesize]{todonotes}

\usepackage{morewrites}

% prepare package changes for managing the review process
\usepackage[draft, markup=underlined, highlightmarkup=background]{changes}
\definechangesauthor[name= Albert Fleischmann, color=green]{AF}
\definechangesauthor[name= Werner Schmidt, color=blue]{WS}
\definechangesauthor[name= Robert Singer, color=cyan]{RS}
\definechangesauthor[name= Christian Stary, color=purple]{CS}
\definechangesauthor[name= Andre Wolski, color=magenta]{AW}
\definechangesauthor[name= Stephan Borgert, color=orange]{SB}
\definechangesauthor[name= Matthes Elstermann, color=pink]{ME}
\definechangesauthor[name= Thomas Schaller, color=violet]{TS}
\definechangesauthor[name= alle anderen, color=lightgray]{STAR}

\usepackage{soulutf8}
\usepackage{soul}

\usepackage{memsty}
\usepackage{memlays}
\usepackage{titlepages}
\usepackage{dpfloat}
\usepackage{fonttable}[2009/04/01]
\usepackage{sidecap}
\usepackage{wrapfig}
\usepackage[htt]{hyphenat}
\usepackage{lscape}
\usepackage{listings}
\usepackage{enumitem}
\usepackage{longtable}
\usepackage{pdfpages}
\usepackage{caption}

\usepackage{amssymb}
\usepackage{program}
\usepackage{mathtools}
\usepackage{graphicx}
\usepackage{minted}

\usepackage{xargs}                      % Use more than one optional parameter in a new commands
\usepackage{xcolor}  % Coloured text etc.









\newmintinline[asminline]{lexer.py:CoreASMLexer -x}{escapeinside=~~}

\setlistdepth{9}

\setlist[itemize,1]{label=$\bullet$}
\setlist[itemize,2]{label=$\bullet$}
\setlist[itemize,3]{label=$\bullet$}
\setlist[itemize,4]{label=$\bullet$}
\setlist[itemize,5]{label=$\bullet$}
\setlist[itemize,6]{label=$\bullet$}
\setlist[itemize,7]{label=$\bullet$}
\setlist[itemize,8]{label=$\bullet$}
\setlist[itemize,9]{label=$\bullet$}

\renewlist{itemize}{itemize}{9}

% Use the built-in division styling
\headstyles{memman}

% BoD bietet zwei Grossformate an: A4 und etwas kleiner
% Springer empfiehlt 24 x 17
\setstocksize{29.7cm}{21cm}

% size of the page after trimming the typeblock
% Springer Textfeld inkl. Kopfzeile 20.5 x 13.5
\settrimmedsize{29.7cm}{21.0cm}{*}

% lines of text
\settypeblocksize{50\onelineskip}{32pc}{*}

% Stocksize ist deckungsgleich mit Druckbereich
\setlength{\trimtop}{0pt}
\setlength{\trimedge}{\stockwidth}
\addtolength{\trimedge}{-\paperwidth}

% Margins
%\setulmargins{2.0cm}{*}{*}
%\setlrmargins{2.0cm}{*}{*}

%\setheadfoot{\onelineskip}{\onelineskip}
%\setheaderspaces{*}{\onelineskip}{*}

%\setmarginnotes{17pt}{3cm}{\onelineskip}

%\sideparmargin{outer}
%\sidecapmargin{outer}

\checkandfixthelayout[lines]

\title{Standard for the Subject-oriented\\ Specification of Systems}
\author{Stephan Borgert, Matthes Elstermann, Albert Fleischmann, \\ Reinhard Gniza, Herbert Kindermann, Florian Krenn,\\ Christian Stary, Werner Schmidt, Robert Singer, Florian Strecker, Andr\'e Wolski}

% ToC down to subsections
\settocdepth{subsection}
% Numbering down to subsections as well
\setsecnumdepth{subsection}

% extra index for first lines
%\makeindex[lines]

% keine Links anzeigen
\hypersetup{
	colorlinks=false,
	pdfborder={0 0 0},
}

\renewcommand*{\sideparfont}{\normalfont\footnotesize}

\DeclareMathSymbol{\dres}{\mathbin}{AMSa}{"43}
\DeclareMathSymbol{\rres}{\mathbin}{AMSa}{"42}
\def\boole{{ \sf I\!B}}
%\def\natnull{{ \sf I\!N}_{o}}           % |No
%\def\natohnenull{{ \sf I\!N}}           % |N
\def\llhd{\dsub}
%\def\llhd{{\lhd\!\!\!\!\!-}}

%---- Funktioniert im Standard-Dokument nicht 
%\def\rrhd{\rsub}

%---- Neuer Versuch
\newcommand{\rrhd}{\mathrel{%
    \ooalign{$\triangleright$\cr\hidewidth\scalebox{.65}[1]{$-$}\hidewidth\cr}%
    }}
%\def\rrhd{{\rhd\!\!\!\!\!-}}
\def\natohnenull{{ I\!\!N}}           % |N
\def\natnull{\natohnenull_{o}}           % |N
\def\oder{{\cap\!\cap}}
\def\und{{\cup\!\cup}}
\def\RETURN{{\keyword{return}}}

% Hier meine Makros
%--------------------------
\def\beztyp{\mathcal{B}}
\def\bez{B}
\def\oetyp{\mathcal{OE}}
\def\ftyp{\mathcal{F}}
\def\orgtyp{\mathcal{O}}
\def\oe{OE}
\def\OrgElementMenge{\mathcal{ORG}}
\def\Bezeichner{\mathcal{BEZ}}
\def\f{F}
\def\a{A}
\def\orginstanz{\mathcal{o}}
%\def\rel{\Gamma_{\oetyp,\oetyp}}
\def\reloetypoetyp1{\Theta_{\oetyp,\oetyp}}
\def\beispielende{$\Diamond$}
\def\relmenge{\mathfrak{R}}
\def\attribute{\mathcal{ATT}}
\def\zeitausdruck{ZA}
\def\za{za}
\def\zeit{T}
\def\ID{ID}
\def\id{id}
\def\domaene{\mathcal{DOM}}
\def\praedikat{\mathcal{P}}
\def\relname{\mathcal{RN}}
\def\rel{\Gamma}
\def\relationen{\mathcal{REL}}
\def\WerteMenge{\mathcal{W}}

%--------------------------------------------------------------------
% Definitionen der Relationssymbole
%--------------------------------------------------------------------

\def\typsymbol{\Upsilon}
\def\struktursymbol{s}
\def\benutzersymbol{b}
\def\instsymbol{A}
\def\OEFsymbol{E}
\def\FAsymbol{\f\a}
\def\relmengetyp{{\relmenge}^{\typsymbol}}
\def\typstrukturrelation{{\rel}^{\typsymbol}_{\struktursymbol}}
\def\typbenutzerrelationenmenge{{\relmenge}^{\typsymbol}_{\benutzersymbol}}
\def\typbenutzerrelation{{\rel}^{\typsymbol}_{\benutzersymbol}}
\def\relmengeinst{{\relmenge}^{\instsymbol}} %R^A
\def\relmengeinstOE_F{{\relmenge}^{\OEFsymbol}}
\def\relstrukturOE_F{{\rel}^{\OEFsymbol}_{\struktursymbol}}
\def\relmengebenutzerinstOE_F{{\relmenge}^{\OEFsymbol}_{\benutzersymbol}}
\def\relbenutzerinstOE_F{{\rel}^{\OEFsymbol}_{\benutzersymbol}}
\def\relmengeinstFA{{\relmenge}^{\FAsymbol}}
\def\relstrukturFA{{\rel}^{\FAsymbol}_{\struktursymbol}}
\def\relmengebenutzerFA{{\relmenge}^{\FAsymbol}_{\benutzersymbol}}
\def\relbenutzerFA{{\rel}^{\FAsymbol}_{\benutzersymbol}}


%----------------------------
\let\rsa=\rightarrow

\def\plus{{\sqcup\!\!\!\sqcup}}
\def\minus{{\backslash\!\backslash}}
\newtheorem{algorithm}{Algorithm}



\begin{document}

\chapterstyle{veelo}

\frontmatter

\pagestyle{empty}

% half-title page
\vspace*{3cm}
\begin{adjustwidth}{0cm}{-3cm}
	\begin{flushright}
		\LARGE\textsf {Standard for the Subject-oriented\\Specification of Systems}
	\end{flushright}
\end{adjustwidth}
\vspace*{\fill}
\cleardoublepage

% title page
\vspace*{0cm}
\begin{flushleft}
	\Large\textsf{Albert Fleischmann \textit{Editor}}\par
\end{flushleft}
\vspace{2cm}
\begin{flushleft}
	\Huge\textsf{Standard for the Subject-oriented\\Specification of Systems}\par
	\bigskip\bigskip
	\Large\textsf{Working Document}
\end{flushleft}
\vspace{2.5cm}
\begin{flushleft}
	%\normalsize\textsf{With 178 Figures}
\end{flushleft}
\vspace*{\fill}
\clearpage

% copyright page
\begingroup
\footnotesize
\setlength{\parindent}{0pt}
\setlength{\parskip}{\baselineskip}
% Platz
\vspace*{3cm}
% Autoren
\begin{flushleft}
	Egon B\"orger\\
	xyz\\
	\medskip
	Stephan Borgert\\
	xyz\\
	\medskip
	Matthes Elstermann\\
	xyz\\
	\medskip
	Albert Fleischmann\\
	InterAktiv Unternehmensberatung, Pfaffenhofen a.d. Ilm, Germany\\
	\medskip
	Reinhard Gniza\\
	xyz\\
	\medskip
	Herbert Kindermann\\
	xyz\\
	\medskip
	Florian Krenn\\
	xyz\\
	\medskip
	Thomas Schaller\\
	xyz\\
	\medskip
	Werner Schmidt\\
	Technische Hochschule, Ingolstadt, Germany\\
	\medskip
	Robert Singer\\
	FH JOANNEUM--University of Applied Sciences, Graz, Austria\\
	\medskip
	Christian Stary\\
	Johannes Kepler Universit\"at, Linz, Austria\\
	\medskip
	Florian Strecker\\
	xyz\\
	\medskip
	Andr\'e Wolski\\
	Technische Universit\"at, Darmstadt, Germany\\
	\medskip
	Conny Zebold\\
	Technische Hochschule, Ingolstadt, Germany\\
\end{flushleft}
\vspace*{\fill}
This work is subject to copyright. All rights are reserved, whether the whole or part of the material is concerned, specifically the rights of translation, reprinting, reuse of illustrations, recitation, broadcasting, reproduction on microfilm or in any other way, and storage in data banks. Duplication of this publication or parts thereof is permitted only under the provisions of the German Copyright Law of September 9, 1965, in its current version. Violations
are liable to prosecution under the German Copyright Law.\par
\textcopyright{ 2020} Institute of Innovative Process Management, Ingolstadt\par
The use of general descriptive names, registered names, trademarks, etc. in this publication does not imply, even in the absence of a specific statement, that such names are exempt from the relevant protective laws and regulations and therefore free for general use.\par
Typeset by the authors\par
Production and publishing: XYZ\par
ISBN: 978-3-123-45678-9 (dummy)
\endgroup
\clearpage

\pagestyle{companion}

\setupshorttoc
\tableofcontents
\cleardoublepage

\chapter{Commands for Reciew process}
For managing changes the package change is used. Following commands are essential:\\
\textbf{Mark text without id:}\\
\hl{
	This is the way  text is marked
}\\

\textbf{Add todo withaout id:}\\
Here is some text
\todo{The original todo note withouth changed colours.\newline Here's another line.} Here is some text
\newline
\newline
\newline
\newline
\textbf{Add todo withaout id and inline:}\\
Here is some text
\todo[inline]{The original todo note withouth changed colours.\newline Here's another line.} Here is some text
\newline

\textbf{Add text with id:}\\
\added[id=AF, comment=remarks to added text]{Added text}\\

\textbf{Delete command with id:}\\
\deleted[id=WS, comment=example for delete]{Here is the text to be deleted}
Here is some text\\
\newline
\textbf{Replace text with id:}\newline
Here is some text
\replaced[id=CS, comment=Remarks to Replace text]{new text}{old text}
Here is some text\\

\textbf{highlight text with id:}\\
Here is some text
\highlight[id=AF, comment=remarks]{highlighted text}\\

\textbf{Add todo with fancy line but without id:}\\
Here is some text \todo[color=green!40, fancyline]{The original todo note.\newline Here's another line with a fancy line}
 Here is some text
\newline 
\newline
\newline
\newline 
\newline
\newline
For todo commands ids are not allowed. For the add, highlight and
Documentation for using the changes package can be found:\\
http://ctan.ebinger.cc/tex-archive/macros/latex/contrib/changes/changes.english.pdf
\newline
The ids and colors for the various reviewers can be found in the preamble (line 29-33) of the text file.

\chapter{Preface}
This book is a collection of all the essential aspects of subject oriented modeling and programming. Many parts of this documents are copy pasts of various books and papers. Table \ref{tbl:sources} shows the sources for the various chapters and sections.\\
In chapter 1 an overview is given to subject orientated language PASS and the methods which are used to describe its structural and execution semantics. OWl (Web Ontology Language) \cite{Web:OWL} is used to describe the structural semantics and Abstract State Machines (ASM) \cite{book:ASM-2018}, \cite{book:ASM-2003} is applied to specify the execution demantics precisely.\\
In chapter 2 the structure of PASS specification is considered in detail and its semantic is defined precisely in a formal way using OWL and ASM. The semantic of the execution of PASS models is described in chapter 3. The execution semantic uses coreASM which is an executable extension of the ASM method. This means that models described in PASS can be executed by a coresponding interpreter.\\
Chapter 4 shows how the abstract implementation independent PASS models can be implemented using software components, physical components or human.  A formal language is described how the right ressources are assigned to the entities of the model.\\
In the last chapter some additional aspects of the subject oriented modeling approach is considered. Thes aspects extend the kernel which was defined in chapter 2, 3 and 4.\\
The appendix contains the details of the formal semantic of PASS. Appendix A contains the complete Ontology, Abbendix B defines the mapping of the ontology to the ASM definition of PASS and Appendix C containes the complete formal execution semantics.



\chapter{Preface}
\todo[inline]{The preface has to be more detailed}

Recent years have seen significant increases in the scope and complexity of digital systems - cf. hhe uprise of Cyber-Physical Systems. Developers, providers, and users have recognized the necessity to devote greater attention to the adoption of innovative technologies. It requires a novel type of preparedness, in particular to the process of design and dynamic adaptation to emergent systems. Since subject-oriented thinking and design supports a human-oriented way to structure and imoplement complex systems, its approach to modeling and execution should be 'standardized', as different teams or modelers could interpret subject oriented means of representation differently. Since subject-orientation is a behavior-centered approach to system understanding and development, non-standardized application could easily lead to non-intended behavior of systems or their components.

However, standardization efforts requires essential steps, among them (see also Figure \ref{fig:Standardisation_Process} )the process into seven steps.


\begin{figure}[h]
	\centering
	\includegraphics[width=0.7\linewidth]{Figures/Preface}
	\caption[Standardisation Process]{Standardisation Process}
	\label{fig:Standardisation_Process}
\end{figure}


\begin{enumerate}
	\item initiating the project

Although we have started in 2012 (Fleischmann et al., 2012) the needs for a standard has been identified 2 years ago. We started documenting (Referenz zu paper bei S-BPM 2019?) experiences from providers, users, researchers and system developers. Technological advancements in model execution have led to standardize the semantics, in order to reflect the novel opportunities. Once a need is identified, a proposal to create a new standard has to be put forward.

 \item  Mobilize a working group

Standards are created or reviewed by experts in the relevant field. They include researchers, providers, users and communities of practice, who form into a some technical committee, termed the Standardization Gang for subject orientation.

\item Balloting the standard by public review 

The technical committee conducts preliminary research and creates a draft outline of the new standard. Much of this initial work can be done remotely or in sub groups, however, needs to be consolidated at some point in time. This is where we are today when providing this edition of consolidated standard inputs.

\item Gain approval 

Once a draft is written, it requires approval for public review. This consensus is required in order to progress any further. The next step is to include all feedback and create a revised document for final public review. Thereby, anyone is welcome to provide feedback to improve the quality and ensure all relevant areas and perspectives are captured.

\item Publish and Maintain 

After public review, the standard goes back to the technical committee to make amendments it deems necessary based on the feedback received. The committee then approves the final version of the standard. After the revised standard receives that final approval from the technical committee, it is officially released. System developers or providers may incorporate it into their practice.

\end{enumerate}


The current version is intended for researcher who work in the area of software engineering and business process management. It helps understanding the approach through in-depth provision of the practical and theoretical aspects of subject oriented modeling and implementing systems.\\
The book is a collection of all the essential aspects of subject oriented modeling and programming. Many parts of this documents are reprints of various books and papers. Table \ref{tbl:sources} shows the sources for the various chapters and sections.\\
In chapter 1 an overview is given to the subject orientated language PASS and the methods which are used to describe its structural and execution semantics. OWl (Web Ontology Language) \cite{Web:OWL} is used to describe the structural semantics and Abstract State Machines (ASM) \cite{book:ASM-2018}, \cite{book:ASM-2003} is applied to specify the execution demantics precisely.\\
In chapter 2 the structure of PASS specification is considered in detail and its semantic is defined precisely in a formal way using OWL and ASM. The semantic of the execution of PASS models is described in chapter 3. The execution semantic uses coreASM which is an executable extension of the ASM method. This means that models described in PASS can be executed by a coresponding interpreter.\\
Chapter 4 shows how the abstract implementation independent PASS models can be implemented using software components, physical components or human.  A formal language is described how the right ressources are assigned to the entities of the model.\\
In the last chapter some additional aspects of the subject oriented modeling approach is considered. Thes aspects extend the kernel which was defined in chapter 2, 3 and 4.\\
The appendix contains the details of the formal semantic of PASS. Appendix A contains the complete Ontology, Abbendix B defines the mapping of the ontology to the ASM definition of PASS and Appendix C containes the complete formal execution semantics.



%\begin{longtable}
%	\footnotesize
%	\centering
%	\begin{tabular}[t]{@{}1 p{0.3\linewidth} p{0.3\linewidth} p{0.4\linewidth} @{}}
\todo[inline]{contact publisher for clarifying copyrights}
	\begin{longtable}[t]{ p{1 cm} p{4 cm} p{7 cm} }	
	\toprule
		\textbf{Chapter Nr.} & \textbf{Chapter title}  & \textbf{Source}
		\\
		\midrule
		1.1 \newline 2.1 \newline 3.1 & Subject Orientation and PASS \newline Informal Description \newline Informal Description of Subject Behaviour and its Execution & Fleischmann, Albert; Schmidt, Werner; Stary, Christian; Obermeier, Stefan; B\"orger, Egon; \newline \textit{Subject-Oriented Business Process Management,} \newline Springer Verlag Berlin 2012
		\\
		\midrule
		3.3 & ASM Definition of Subject Execution & 
		\\
		\midrule
		4.1 & People and Organisations & Fleischmann, Albert; Oppl, Stefan; Schmidt, Werner; Stary, Christian;\newline \textit{Contextual Process Digitalization: Changing Perspectives - Design Thinking - Value-Led Design} \newline Springer Verlag Berlin 2020
		\\
		\midrule
		5.1 & Subjects and Shared Input Pools & Fleischmann, A.; Stary, C.,\newline
			\textit{Dependable Data Sharing in Dynamic IoT-Systems - Subject-oriented Process Design, Complex Event Processing, and Blockchains;} \newline
			in Proceedings of S-BPM ONE 2019, 11th International Conference on Subject Oriented Business Process Management,\newline
			editors: Betz, S.;Elstermann, M.; Lederer, M; \newline
			ICPC published by Association of Computing Machinery (ACM) Digital Library; 2019
		\\
		\midrule
		5.2 & Subject-Phase Model based process specifications &  Fleischmann, A.,\newline
		\textit{Activity-Based Costing for S-BPM,}\newline
		Proceedings of the 5th International Conference S-BPM ONE 2013, Computer and Information Sciences (CCIS), No. 360,\newline
		editors: Fischer, H. and Schneeberger, J.\newline
		Springer 2013,
		\\
		\midrule
		5.3 & Hierarchies in Communication Oriented Business Process Models & Elstermann, M and Fleischmann, A.,\newline
		\textit{Modeling Complex Process Systems with Subject Oriebted Means,}\newline
		in Proceedings of S-BPM ONE 2019, 11th International Conference on Subject Oriented Business Process Management,\newline
		editors: Betz, S.;Elstermann, M.; Lederer, M; \newline
		ICPC published by Association of Computing Machinery (ACM) Digital Library; 2019
		\\
		\midrule
		5.4 & Business Activity Monitoring for S-BPM & Schmidt, W.; Fleischmann, A.;\newline
		\textit{Business Process Monitoring with S-BPM,}\newline
		Proceedings of the 5th International Conference S-BPM ONE 2013, Computer and Information Sciences (CCIS), No. 360,\newline
		editors: Fischer, H. and Schneeberger, J.\newline
		Springer 2013,
		\\
		\midrule
		5.5 & Subject Oriented Project Management & Albert Fleischmann, Werner Schmidt, Christian Stary;\newline
		\textit{Subject Oriented Project Management,}\newline
		published in SEAA '14: Proceedings of the 2014 40th EUROMICRO Conference on Software Engineering and Advanced Applications,\newline
		IEEE Computer Society, 2014
		\\
		\midrule
		5.6 & Subject Oriented Fog Computing & Stary, C. ; Fleischmann, A. ; Schmidt, W., \newline
		\textit{Subject-oriented Fog Computing: Enabling Stakeholder Participation in Development,} \newline
		Proceedings of the 4th IEEE World Forum on Internet of Things (WF-IoT), Singapure \newline
		IEEE Xplore Digital Library, DOI 10.1109/WF-IoT.2018.8355167, 2018
		\\
		\midrule
		5.6 & Activity Based Costeing &  Zehbold, C.; Schmidt, W.; Fleischmann, A.,\newline
		\textit{Activity-Based Costing for S-BPM,}\newline
		Proceedings of the 5th International Conference S-BPM ONE 2013, Computer and Information Sciences (CCIS), No. 360,\newline
		editors: Fischer, H. and Schneeberger, J.\newline
		Springer 2013,
		\\
\bottomrule
%\end{tabular}
\caption{Main sources of the various chapters and sections}
\label{tbl:sources}
\end{longtable}




\cleardoublepage

\setupparasubsecs
\setupmaintoc
\tableofcontents
\setlength{\unitlength}{1pt}
\cleardoublepage

%\listoffigures
%\clearpage
%\listoftables

%\lipsum{2}

%===================================

\mainmatter

% !TeX spellcheck = en_US

% Review Christia Stary January 2020

\chapter{Overview}

%\todo[inline]{
%Review CS: Chapter One should be a motivation including history of developements, \newline rationale and target reader groups (sntadarization bodies, developers, practitioners, researchers).
%\newline
%AF: See preface\\
%CS: Hochkommas und Bindestriche im pdf nicht alle lesbar, oft mt Umlauten gedruckt!
%}

To facilitate the understanding of the following sections we will introduce the concept of subject-orienting modeling which is based on the Parallel Activity Specification Scheme (PASS). Additionally, we will give a short introduction to ontologies---especially the Web Ontology Language (OWL)---, and to Abstract State Machines (ASM) as underlying concepts of this standard document.

\section{Subject Orientation and PASS }
\label{SubjectOrient}

%\sidepar{Text in the sidebar}

In this section, we lay the ground for PASS as a language for describing processes in a subject-oriented way. This section is not a complete description of all PASS features, but it gives the first impression about subject-orientation and the specification language PASS. The detailed concepts are defined in the upcoming chapters.

The term subject has manifold meanings depending on the discipline. In philosophy, a subject is an observer and an object is a thing observed. In the grammar of many languages, the term subject has a slightly different meaning. "According to the traditional view, the subject is the doer of the action (actor) or the element that expresses what the sentence is about (topic)."~\cite{Keenan:1976aa}. In PASS the term subject corresponds to the doer of an action whereas in ontology description languages, like RDF (see section \ref{IntroOntology}), the term subject means the topic what the "sentence" is about.

\subsection{Subject-driven Business Processes}

Subjects represent the behavior of an active entity. A specification of a subject does not say anything about the technology used to execute the described behavior. This is different to other encapsulation approaches, such as multi-agent systems.

Subjects communicate with each other by exchanging messages. Messages have a name and a payload. The name should express the meaning of a message informally and the payloads are the data (business objects) transported. Internally, subjects execute local activities such as calculating a price, storing an address, etc.

A subject sends messages to other subjects, expects messages from other subjects, and executes internal actions. All these activities are done in sequences which are defined in a subject's behavior specification. Subject-oriented process specifications are always embedded in a context. A context is defined by the business organization and the technology by which a business process\todo{Add glossary to the document} is executed.

Subject-oriented system development integrates established theories and concepts. It has been inspired by various process algebras (see e.g. [2], [3], [4]), by the basic structure of nearly all natural languages (Subject, Predicate, Object) and the systemic sociology developed by Niklas Luhmann (an introduction can be found in [5]). According to the organizational theory developed by Luhmann, the smallest organization consists of communication executed between at least two information processing entities [5]. The integrated concepts have been enhanced and adapted to organizational stakeholder requirements, such as providing a simple graphical notation, as detailed in the following sections.

\subsection{Subject Interaction and Behavior}

We introduce the basic concepts of process modeling in PASS using a simple order process. A customer sends an order to the order handling department of a supplier. He is going to receive an order confirmation and the ordered product by the shipment company. Figure \ref{fig:ordercomstructure} shows the communication structure of that process. The involved subjects and the messages they exchange can easily be grasped. 

%\strictpagecheck
\begin{figure}[htbp]
	\centering
	\includegraphics[width=0.7\linewidth]{Figures/Ontology/SubjectExecution/OrderComStructure}
	\caption[The Communication Structure in the Order Process]{The Communication Structure in the Order Process}
	\label{fig:ordercomstructure}
\end{figure}

Each subject has a so-called input pool which is its mailbox for receiving messages. This input pool can be structured according to the business requirements at hand. The modeler can define how many messages of which type and/or from which sender can be deposited and what the reaction is if these restrictions are violated. This means the synchronization through message exchange can be specified for each subject individually.

Messages have an intuitive meaning expressed by their name. A formal semantics is given by their use and the data which are transported with a message. Figure \ref{fig:ordercustomerorderhandling} depicts the behavior of the subjects "customer" and "order handling".

%\strictpagecheck
\begin{figure}[htbp]
	\centering
	\includegraphics[width=0.9\linewidth]{Figures/Ontology/SubjectExecution/OrderCustomerOrderHandling}
	\caption[The Behavior of Subjects]{The Behavior of Subjects}
	\label{fig:ordercustomerorderhandling}
\end{figure}

In the first state of its behavior, the subject "customer" executes the internal function "Prepare order". When this function is finished the transition "order prepared" follows. In the succeeding state "send order" the message "order" is sent to the subject "order handling". After this message is sent (deposited in the input pool of subject "order handling"), the subject "Customer" goes into the state "wait for confirmation". If this message is not in the input pool the subject stops its execution until the corresponding message arrives in the input pool. On arrival, the subject removes the message from the input pool and follows the transition into state "Wait for product" and so on.

The subject "Order Handling" waits for the message "order" from the subject "customer". If this message is in the input pool it is removed and the succeeding function "check order" is executed and so on.

The behavior of each subject describes in which order it sends messages, expects (receives) and performs internal functions. Messages transport data from the sending to the receiving subject and internal functions operate on internal data of a subject. These data aspects of a subject are described in section \ref{SUbjects-Objects}. In a dynamic and fast-changing world, processes need to be able to capture known but unpredictable events. In our example let us assume that a customer can change an order. This means the subject "customer" may send the message "Change order" at any time. Figure\todo{AF: Bild und Text passen nicht Change order} \ref{fig:ordercomstructure} shows the corresponding communication structure, which now contains the message "change order".

%\strictpagecheck
\begin{figure}[htbp]
	\centering
	\includegraphics[width=0.7\linewidth]{Figures/Ontology/SubjectExecution/OrderComStructure}
	\caption[The Communication Structure with Change Message]{The Communication Structure with Change Message}
	\label{fig:ordercomstructure}
\end{figure}

Due to this unpredictable event, the behavior of the involved subjects needs also to be adapted. Figure \ref{fig:ordercustomerchange} illustrates the respective behavior of the customer. 

%\strictpagecheck
\begin{figure}[htbp]
	\centering
	\includegraphics[width=0.9\linewidth]{Figures/Ontology/SubjectExecution/OrderCustomerChange}
	\caption[Customer is allowed to Change Orders]{Customer is allowed to Change Orders}
	\label{fig:ordercustomerchange}
\end{figure}

The subject "customer" may have the idea to change its order in the state "wait for confirmation" or in the state "wait for product". The flags in these states indicate that there is a so-called behavior extension described by a so-called nondeterministic event guard [12, 22]. The non-deterministic event created in the subject is the idea "change order". If this idea comes up, the current states, either "wait for confirmation" or "wait for product", are left, and the subject "customer" jumps into state "change order" in the guard behavior. In this state, the message "change order" is sent and the subject waits in the state "wait for reaction". In this state, the answer can either be "order change accepted" or "order change rejected". Independently of the received message the subject "customer" moves to the state "wait for product". The message "order change accepted" is considered as confirmation, if a confirmation has not arrived yet (state "wait for confirmation"). If the change is rejected the customer has to wait for the product(s) he/she has ordered originally. Similar to the behavior of the subject "customer" the behavior of the subject "order handling" has to be adapted.

\subsection{Subjects and Objects}
\label{SUbjects-Objects}

Up to now, we did not mention data or the objects with their predicates, to get complete sentences comprising subject, predicate, and object. Figure\todo{CS:Wrong reference link } \ref{fig:subjectobject} displays how subjects and objects are connected. The internal function "prepare order" uses internal data to prepare the data for the order message. This order data is sent as the payload of the message "order".

%\strictpagecheck
\begin{figure}[htbp]
	\centering
	\includegraphics[width=0.9\linewidth]{Figures/Ontology/SubjectExecution/SUbjectObject}
	\label{fig:subjectobject}
	\caption[Subjects and Objects]{Subjects and Objects}
\end{figure}

The internal functions in a subject can be realized as methods of an object or functions implemented in a service if a service-oriented architecture is available. These objects have an additional method for each message. If a message is sent, the method allows receiving data values sent with the message, and if a message is received the corresponding method is used to store the received data in the object [22]. This means either subject are the entities which use synchronous services as an implementation of functions or asynchronous services are implemented through subjects or even through complex processes consisting of several subjects. Consequently, the concept Service Oriented Architecture (SOA) is complementary to S-BPM: Subjects are the entities which use the services offered by SOAs (cf. [25]).

\section{Introduction to Ontologies and OWL }
\label{IntroOntology}

This short introduction to ontology, the Resource Description Framework and Web Ontology Language (OWL), should help to get an understanding of the PASS ontology outlined in section \ref{PASSStruct} and \ref{PASSExec}.

Ontologies are a formal way to describe taxonomies and classification networks, essentially defining the structure of knowledge for various domains: the nouns representing classes of objects and the verbs representing relations between the objects of classes.

In computer science and information science, an ontology encompasses a representation, formal naming, and definition of the classes, properties, and relations between the data, and entities that substantiate considered domains.

The Resource Description Framework (RDF) provides a graph-based data model or framework for structuring data as statements about resources. A "resource" may be any "thing" that exists in the world: a person, place, event, book, museum object, but also an abstract concept like data objects. Figure \ref{fig:classes-properties}  shows an RDF graph.

%\strictpagecheck
\begin{figure}[h]
	\centering
	\includegraphics[width=0.6\linewidth]{Figures/Ontology/Introduction/Classes-Properties}
	\caption[RDF graphic]{RDF graphic}
	\label{fig:classes-properties}
\end{figure}

RDF is based on the idea of making statements about resources (in particular web resources) in expressions of the form subject–predicate–object, known as triples. The subject denotes the resource, and the predicate denotes traits or aspects of the resource and expresses a relationship between the subject and the object. In the context of ontology, the term subject expresses what the sentence is about (topic) (see \ref{SubjectOrient}).

For describing ontologies several languages have been developed. One widely used language is OWL (worldwide web ontology language) which is based on the Resource Description Framework (RDF).

OWL has classes, properties, and instances. Classes represent terms also called concepts. Classes have properties and instances are individuals of one or more classes.

A class is a type of thing. A type of "resource" in the RDF sense can be
person, place, object, concept, event, etc.. Classes and subclasses form a hierarchical taxonomy and members of a subclass inherit the characteristics of their parent class (superclass). Everything true for the parent class is also true for the subclass.

A member of a subclass "is a", or "is a kind of" its parent class. Ontologies define a set of properties used in a specific knowledge domain. In an ontology context, properties relate members of one class to members of another class or a literal.

Domains and ranges define restrictions on properties. A domain restricts what kinds of resources or members of a class can be the subject of a given property in an RDF triple. A range restricts what kinds of resources/members of a class or data types (literals) can be the object of a given property in an RDF triple.

Entities belonging to a certain class are instances of this class or individuals. A simple ontology with various classes, properties and individual is shown below:

Ontology statement examples:

\begin{itemize}
	\item \textbf {Class definition statements:}
	\begin{itemize}
		\item Parent \texttt{isA} Class
		\item Mother \texttt{isA} Class
		\item Mother \texttt{subClassOf} Parent
		\item Child \texttt{isA} Class
	\end{itemize}
	\item \textbf {Property definition statement:}
	\begin{itemize}
		\item \texttt{isMotherOf} is a relation between the classes Mother and Child
	\end{itemize}
	\item \textbf{Individual/instance statements:}
	\begin{itemize}
		\item MariaSchmidt \texttt{isA} Mother
		\item MaxSchmidt \texttt{isA} Child
		\item MariaSchmidt \texttt{isMotherOf} MaxSchmidt
	\end{itemize}
\end{itemize}

\section{Introduction to Abstract State Machines }

An abstract state machine (ASM) is a state machine operating on states that are arbitrary data structures (structure in the sense of mathematical logic, that is a nonempty set together with several functions (operations) and relations over the set).

The language of the so-called Abstract State Machine uses only elementary If-Then-Else-rules which are typical also for rule systems formulated in natural language, i.e., rules of the (symbolic) form 

\medskip
\textbf{if} \textit{Condition} \textbf{then} \textit{ACTION}
\medskip

with arbitrary \textit{Condition} and \textit{ACTION}. The latter is usually a finite set of assignments of the form \textit{f (t1, ..., tn) := t}. The meaning of such a rule is to perform in any given state the indicated action if the indicated condition holds in this state.

The unrestricted generality of the used notion of Condition and \textit{ACTION} is guaranteed by using as ASM-states the so-called Tarski structures, i.e., arbitrary sets of arbitrary elements with arbitrary functions and relations defined on them. These structures are updatable by rules of the form above. In the case of business processes, the elements are placeholders for values of arbitrary type and the operations are typically the creation, duplication, deletion, or manipulation (value change) of objects. The so-called views are conceptually nothing else than projections (read: substructures) of such Tarski structures.

An (asynchronous, also called distributed) ASM consists of a set of agents each of which is equipped with a set of rules of the above form, called its program. Every agent can execute in an arbitrary state in one step all its executable rules, i.e., whose condition is true in the indicated state. For this reason, such an ASM, if it has only one agent, is also called sequential ASM. In general, each agent has its own "time" to execute a step, in particular, if its step is independent of the steps of other agents;  in special cases, multiple agents can also execute their steps simultaneously (in a synchronous manner).

Without further explanations, we adopt usual notations, abbreviations, etc., for example:

% for the typesetting of ASM code some work is required!!

%\lstdefinelanguage{ASM}{C}
%\lstset{language=ASM}

\begin{lstlisting}
if Cond then M1 else M2
\end{lstlisting}

instead of the equivalent ASM with two rules:

\begin{lstlisting}
if Cond then M1
if not Cond then M2
\end{lstlisting}

Another notation used below is

\begin{lstlisting}
let x=t in M
\end{lstlisting}

for $M(x/a)$, where $a$ denotes the value of $t$ in the given state and $M(x/a)$ is obtained from $M$ by substitution of each (free) occurrence of $x$ in $M$ by $a$.

For details of a mathematical definition of the semantics of ASMs which justifies their intuitive (rule-based or pseudo-code) understanding, we refer the reader to the AsmBook Börger, E., Stärk R. Abstract State Machines. A Method for High-Level System Design and Analysis. Springer, 2003.






\chapter{Classes and Property of the PASS Ontology}


\section{All Classes (95)}

\begin{itemize}
\item PASSProcessModelElement
\begin{itemize}
	\item BehaviorDescribingComponent
	\begin{itemize}
		\item Action
		\item DataMappingFunction
		\begin{itemize}
			\item DataMappingIncomingToLocal
			\item DataMappingLocalToOutgoing
		\end{itemize}
		\item FunctionSpecification
		\begin{itemize}
			\item CommunicationAct
			\begin{itemize}
				\item ReceiveFunction
				\item SendFunction
			\end{itemize}
			\item DoFunction
		\end{itemize}
		\item ReceiveType
		\item SendType
		\item State
		\begin{itemize}
			\item ChoiceSegment
			\item ChoiceSegmentPath
			\begin{itemize}
				\item MandatoryToEndChoiceSegmentPath
				\item MandatoryToStartChoiceSegmentPath
				\item OptionalToEndChoiceSegmentPath
				\item OptionalToStartChoiceSegmentPath
			\end{itemize}
			\item EndState
			\item GenericReturnToOriginReference
			\item InitialStateOfBehavior
			\item InitialStateOfChoiceSegmentPath
			\item MacroState
			\item StandardPASSState
			\begin{itemize}
				\item DoState
				\item ReceiveState
				\item SendState
			\end{itemize}
			\item StateReference
		\end{itemize}
		\item Transition
		\begin{itemize}
			\item CommunicationTransition
			\begin{itemize}
				\item ReceiveTransition
				\item SendTransition
			\end{itemize}
			\item DoTransition
			\item SendingFailedTransition
			\item TimeTransition
			\begin{itemize}
				\item ReminderTransition
				\begin{itemize}
					\item CalendarBasedReminderTransition
					\item TimeBasedReminderTransition
				\end{itemize}
				\item TimerTransition
				\begin{itemize}
					\item BusinessDayTimerTransition
					\item DayTimeTimerTransition
					\item YearMonthTimerTransition
				\end{itemize}
				\item UserCancelTransition
			\end{itemize}
			\item TransitionCondition
			\begin{itemize}
				\item DoTransitionCondition
				\item MessageExchangeCondition
				\begin{itemize}
					\item ReceiveTransitionCondition
					\item SendTransitionCondition
				\end {itemize}
				\item SendingFailedCondition
				\item TimeTransitionCondition
				\begin{itemize}
					\item ReminderEventTransitionCondition
%					\begin{itemize}
%						\item CalendarBasedReminderTransitionCondition
%						\item TimeBasedReminderTimeOutTransitionCondition
%					\end{itemize}
					\item TimerTransitionCondition
%					\begin{itemize}
%						\item BusinessDayTimerTransitionCondition
%						\item DayTimeTimerCondition
%						\item YearMonthTimerTransitionCondition
%					\end{itemize}
				\end{itemize}
			\end{itemize}
		\end{itemize}
	\end{itemize}		
			
	\item DataDescribingComponent
	\begin{itemize}
		\item DataObjectDefinition
		\begin{itemize}
			\item DataObjectListDefintion
			\item PayloadDataObjectDefinition
			\item SubjectDataDefinition
		\end{itemize}
		\item DataTypeDefinition
		\begin{itemize}
			\item CustomOrExternalDataTypeDefinition
			\begin{itemize}
					\item JSONDataTypeDefinition
					\item OWLDataTypeDefinition
					\item XSD-DataTypeDefinition
			\end{itemize}
			\item ModelBuiltInDataTypes
		\end{itemize}
		\item PayloadDescription
		\begin{itemize}
			\item PayloadDataObjectDefinition
			\item PayloadPhysicalObjectDescription
		\end{itemize}
	\end{itemize}
	\item InteractionDescribingComponent
	\begin{itemize}
		\item InputPoolConstraint
		\begin{itemize}
			\item MessageSenderTypeConstraint
			\item MessageTypeConstraint
			\item SenderTypeConstraint
		\end{itemize}
		\item InputPoolContstraintHandlingStrategy
		\item MessageExchange
		\item MessageExchangeList
		\item MessageSpecification
		\item Subject
		\begin{itemize}
			\item FullySpecifiedSubject
			\item InterfaceSubject
			\item MultiSubject
			\item SingleSubject
			\item StartSubject
		\end{itemize}
	\end{itemize}
				
	\item PASSProcessModel
	\item SubjectBehavior
	\begin{itemize}
		\item GuardBehavior
		\item MacroBehavior
		\item SubjectBaseBehavior
	\end{itemize}
\end{itemize}
	
\item SimplePASSElement
\begin{itemize}
	\item CommunicationTransition
	\begin{itemize}
		\item ReceiveTransition
		\item SendTransition
	\end{itemize}
	\item DataMappingFunction
	\begin{itemize}
		\item DataMappingIncomingToLocal
		\item DataMappingLocalToOutgoing
	\end{itemize}
	\item DoTransition
	\item DoTransitionCondition
	\item EndState
	\item FunctionSpecification
	\begin{itemize}
		\item CommunicationAct
		\begin{itemize}
			\item ReceiveFunction
			\item SendFunction
		\end{itemize}
		\item DoFunction
	\end{itemize}
	\item InitialStateOfBehavior
	\item MessageExchange
	\item MessageExchangeCondition
	\begin{itemize}
		\item ReceiveTransitionCondition
		\item SendTransitionCondition
	\end{itemize}
	\item MessageExchangeList
	\item MessageSpecification
	\item ModelBuiltInDataTypes
	\item PayloadDataObjectDefinition
	\item StandardPASSState
	\begin{itemize}
		\item DoState
		\item ReceiveState
		\item SendState
	\end{itemize}
	\item Subject
	\begin{itemize}
		\item FullySpecifiedSubject
		\item InterfaceSubject
		\item MultiSubject
		\item SingleSubject
		\item StartSubject
	\end{itemize}
	\item SubjectBaseBehavior
\end{itemize}
\end{itemize}	

			
			
			
			










\section{Data Properties (27)}

hasBusinessDayDurationTimeOutTime
hasCalendarBasedFrequencyOrDate
hasDataMappingString
hasDayTimeDurationTimeOutTime
hasDurationTimeOutTime
hasFeelExpressionAsDataMapping
hasGraphicalRepresentation
hasKey
hasLimit
hasMaximumSubjectInstanceRestriction
hasMetaData
hasModelComponentComment
hasModelComponentID
hasModelComponentLabel
hasPriorityNumber
hasReoccuranceFrequenyOrDate
hasSVGRepresentation
hasTimeBasedReoccuranceFrequencyOrDate
hasTimeValue
hasToolSpecificDefinition
hasValue
hasYearMonthDurationTimeOutTime
isOptionalToEndChoiceSegmentPath
isOptionalToStartChoiceSegmentPath
owl:topDataProperty
PASSModelDataProperty
SimplePASSDataProperties




\section{Object Properties (42)}

\begin{landscape}
\begin {longtable} { l | p{4 cm}  l |  l |l |}
\hline
Property name & Domain & Range & Reference\\
\toprule
\endhead
\hline
belongsTo & & &\\
\hline
contains & & &\\
\hline	
belongsTo & PASSProcessModelElement & PASSProcessModelElement & \\
\hline
contains & PASSProcessModelElement& PASSProcessModelElement & \\
\hline
containsBaseBehavior & Subject & SubjectBehavior & \\
\hline
containsBehavior & Subject & SubjectBehavior & \\
\hline
containsPayloadDescription & MessageSpecification & PayloadDescription & \\
\hline
guardedBy & State, Action & GuardBehavior & \\
\hline
guardsBehavior &GuardBehavior &SubjectBehavior &  \\
\hline
guardsState & State, Action & guardedBy & \\
\hline
hasAdditionalAttribute & PASSProcessModelElement& AdditionalAttribute&  \\
\hline
hasCorrespondent & & Subject & \\
\hline
hasDataDefinition & & DataObjectDefinition & \\
\hline
hasDataMappingFunction &state, SendTransition,\ ReceiveTransition & DataMappingFunction &  \\
\hline 
hasDataType & PayloadDescription or \ DataObjectDefinition & DataTypeDefinition &  \\
\hline
hasEndState & SubjectBehavior \ or ChoiceSegmentPath & State, not SendState &  \\
\hline
hasFunctionSpecification & State& FunctionSpecification&  \\
\hline
hasHandlingStrategy &InputPoolConstraint & InputPoolContstraintHandlingStrategy &  \\
\hline
hasIncomingMessageExchange & Subject& MessageExchange &  \\
\hline
hasIncomingTransition &State &Transition &  \\
\hline
hasInitialState & SubjectBehavior or ChoiceSegmentPath &State &  \\
\hline
hasInputPoolConstraint &Subject &InputPoolConstraint &  \\
\hline
hasKeyValuePair & & &  \\
\hline
hasMessageExchange & Subject & &  \\
\hline
hasMessageType & MessageTypeConstraint or\ MessageSenderTypeConstraint or \ MessageExchange
 &MessageSpecification &  \\
\hline
hasOutgoingMessageExchange & Subject& MessageExchange&  \\
\hline
hasOutgoingTransition &State & Transition&  \\
\hline
hasReceiver &MessageExchange & Subject & \\
\hline
hasRelationToModelComponent & PASSProcessModelElement& PASSProcessModelElement &  \\
\hline
hasSender &MessageExchange & Subject & \\
\hline
hasSourceState & Transition& State&  \\
\hline
hasStartSubject & PASSProcessModel& StartSubject&  \\
\hline
hasTargetState & Transition& State&  \\
\hline
hasTransitionCondition &Transition &TransitionCondition &  \\
\hline
isBaseBehaviorOf &SubjectBaseBehavior & &  \\
\hline
isEndStateOf & State and not SendState & SubjectBehavior or ChoiceSegmentPath &  \\
\hline
isInitialStateOf & State& SubjectBehavior or ChoiceSegmentPath &  \\
\hline
isReferencedBy & & &  \\
\hline
references & & &  \\
\hline
referencesMacroBehavior &MacroState &MacroBehavior &  \\
\hline
refersTo & CommunicationTransition& MessageExchange&  \\
\hline
requiresActiveReceptionOfMessage &ReceiveTransitionCondition &MessageSpecification &  \\
\hline
requiresPerformedMessageExchange & MessageExchangeCondition&MessageExchange &  \\
\hline
SimplePASSObjectPropertie & & &  \\
\hline
\end{longtable}
\end {landscape}



\chapter{Execution of a PASS Model}

\section{Informal Description of Subject Behavior and its Execution}
The excution of subject means sending and reveiving messages and executing internal activities in the defined order. In the following sections it is described what sending and receiving messages and executing internal functions means.

\subsection{Sending Messages}
Before sending a message, the values of the parameters to be transmitted need to be determined. In case the message parameters are simple data types, the required values are taken from local variables or business objects of the sending subject, respectively. In case of business objects, a current instance of a business object is transferred as a message parameter.\\
The sending subject attempts to send the message to the target subject and store it in its input pool. Depending on the described configuration and status of the input pool, the message is either immediately stored or the sending subject is blocked until a delivery of the message is possible.\\
In the sample business trip application, employees send completed requests using the message ‘send business trip request’ to the manager’s input pool. From a send state, several messages can be sent as an alternative. The following example shows a send state in which the message M1 is sent to the subject S1, or alternatively the message M2 is sent to S2, therefore referred to as alternative sending (see Figure \ref{fig:sendstate}). It does not matter which message is attempted to be sent first. If the send mechanism is successful, the corresponding state transition is executed. In case the message cannot be stored in the input pool of the target subject, sending is interrupted automatically, and another designated message is attempted to be sent. A sending subject will thus only be blocked if it cannot send any of the provided messages.

\begin{figure}[ph]
	\centering
	\includegraphics[width=0.7\linewidth]{20181026-Ontologie-Bilder/Grafiken-Ontologie/SUbjectExecution/sendState}
	\caption[Example of alternative sending]{Example of alternative sending}
	\label{fig:sendstate}
\end{figure}

By specifying priorities, the order of sending can be influenced. For example, it can be determined that the message M1 to S1 has a higher priority than the message M2 to S2. Using this specification, the sending subject starts with sending message M1 to S1 and then tries only in case of failure to send message M2 to S2. In case message M2 can also not be sent to the subject S2, the attempts to send start from the beginning.

The blocking of subjects when attempting to send can be monitored over time with the so-called timeout. The example in Figure \ref{fig:sendstatetimer} shows with ‘Timeout: 24 h’ an additional state transition which occurs when within 24 hours one of the two messages cannot be sent. If a value of zero is specified for the timeout, the process immediately follows the timeout path when the alternative message delivery fails completely.

\begin{figure*}[ph]
	\centering
	\includegraphics[width=0.7\linewidth]{20181026-Ontologie-Bilder/Grafiken-Ontologie/SUbjectExecution/SendSTateTimer}
	\caption[Send using time monitoring]{Send using time monitoring}
	\label{fig:sendstatetimer}
\end{figure*}


\subsection{Receiving Messages}
Analogously to sending, the receiving procedure is divided into two phases, which run inversely to send.

The first step is to verify whether the expected message is ready for being picked up. In case of synchronous messaging, it is checked whether the sending subject offers the message. In the asynchronous version, it is checked whether the message has already been stored in the input pool. If the expected message is accessible in either form, it is accepted, and in a second step, the corresponding state transition is performed. This leads to a takeover of the message parameters of the accepted message to local variables or business objects of the receiving subject. In case the expected message is not ready, the receiving subject is blocked until the message arrives and can be accepted.

In a certain state, a subject can expect alternatively multiple messages. In this case, it is checked whether any of these messages is available and can be accepted. The test sequence is arbitrary, unless message priorities are defined. In this case, an available message with the highest priority is accepted. However, all other messages remain available (e.g., in the input pool) and can be accepted in other receive states.

Figure \ref{fig:receivestate} shows a receive state of the subject ‘employee’ which is waiting for the answer regarding a business trip request. The answer may be an approval or a rejection.

\begin{figure}[ph]
	\centering
	\includegraphics[width=0.7\linewidth]{20181026-Ontologie-Bilder/Grafiken-Ontologie/SUbjectExecution/ReceiveState}
	\caption[Example of alternative receiving]{Example of alternative receiving}
	\label{fig:receivestate}
\end{figure}

Just as with sending messages, also receiving messages can be monitored over time. If none of the expected messages are available and the receiving subject is therefore blocked, a time limit can be specified for blocking. After the specified time has elapsed, the subject will execute the transition as it is defined for the timeout period. The duration of the time limit may also be dynamic, in the sense that at the end of a process instance the process stakeholders assigned to the subject decide that the appropriate transition should be performed. We then speak of a manual timeout.

\begin{figure}[ph]
	\centering
	\includegraphics[width=0.7\linewidth]{20181026-Ontologie-Bilder/Grafiken-Ontologie/SUbjectExecution/ReceiveStateTimer}
	\caption[Time monitoring for message reception]{Time monitoring for message reception}
	\label{fig:receivestatetimer}
\end{figure}

\newpage
Figure \ref{fig:receivestatetimer} shows that, after waiting three days for the manager’s answer, the employee sends a corresponding request.

Instead of waiting for a message for a certain predetermined period of time, the waiting can be interrupted by a subject at all times. In this case, a reason for abortion can be appended to the keyword ‘breakup’. In the example shown in Figure \ref{fig:receivestatebreak}, the receive state is left due to the impatience of the subject.

\begin{figure}[ph]
	\centering
	\includegraphics[width=0.7\linewidth]{20181026-Ontologie-Bilder/Grafiken-Ontologie/SUbjectExecution/ReceiveStateBreak}
	\caption[Message reception with manual interrupt]{Message reception with manual interrupt}
	\label{fig:receivestatebreak}
\end{figure}
\newpage

\subsection{Subject Behavior}
The possible sequences of a subject’s actions in a process are termed subject behavior. States and state transitions describe what actions a subject performs and how they are interdependent. In addition to the communication for sending and receiving, a subject also performs so-called internal actions or functions.

States of a subject are therefore distinct: There are actions on the one hand, and communication states to interact with other subjects (receive and send) on the other. This results in three different types of states of a subject. Figure \ref{fig:behavior-symbole} shows the different types of states with the coresponding symbols.

\begin{figure}[ph]
	\centering
	\includegraphics[width=0.5\linewidth]{20181026-Ontologie-Bilder/Grafiken-Ontologie/SUbjectExecution/Behavior-Symbole}
	\caption[State types and coresponding symbols]{State types and coresponding symbols}
	\label{fig:behavior-symbole}
\end{figure}

In S-BPM, work performers are equipped with elementary tasks to model their work procedures: sending and receiving messages and immediate accomplishment of a task (function state).
In case an action associated with a state (send, receive, do) is possible, it will be executed, and a state transition to the next state occurs. The transition is characterized through the result of the action of the state under consideration: For a send state, it is determined by the state transition to which subject what information is sent. For a receive state, it becomes evident in this way from what subject it receives which information. For a function state, the state transition describes the result of the action, e.g., that the change of a business object was successful or could not be executed.

The behavior of subjects is represented by modelers using Subject Behavior Diagrams (SBD). Figure \ref{fig:vollst-beispiel} shows the subject behavior diagram depicting the behavior of the subjects ‘employee’, ‘manager’, and ‘travel office’, including the associated states and state transitions. 

\begin{landscape}
\begin{figure}[ph]
	\centering
	\includegraphics[width=0.7\linewidth]{20181026-Ontologie-Bilder/Grafiken-Ontologie/SUbjectExecution/Vollst-Beispiel}
	\caption[Subject behavior diagram for the subjects ‘employee’, ‘manager’, and ‘travel office’]{Subject behavior diagram for the subjects ‘employee’, ‘manager’, and ‘travel office’}
	\label{fig:vollst-beispiel}
\end{figure}
\end{landscape}
\newpage



\section{Ontology of Subject Behavior Description}

\begin{landscape}
\begin{figure}[ph]
	\centering
	\includegraphics[width=1.0\linewidth]{20181026-Ontologie-Bilder/Grafiken-Ontologie/SUbjectExecution/20181218-SubjectBehavior}
	\caption[Behavior of subjects]{Behavior of subjects}
	\label{fig:20181218-subjectbehavior}
\end{figure}
\end{landscape}


\section{ASM Definition of Subject Execution}

\chapter{Implementation of Subject Oriented Models}

SUbject oriented models address the internal aspects and structures of a system. They are essentially models of the internal structure of a system and cover organizational and technical aspects. When implementing the models, it is now necessary to establish the relationship between the process model and the available resources. Figure \ref{fig:Implementation-steps} shows the individual steps from a process model to the executable process instance.

\begin{figure}[h]
	\centering
	\includegraphics[width=0.9\linewidth]{Figures/Implementation/Implementation-steps.jpg}
	\caption[Implementation steps]{Implementation steps}
	\label{fig:Implementation-steps}
\end{figure}
 
In a system model, the actors, the actions, their sequences and the objects manipulated by the actions are described. Actions (activities) can be performed by humans, software systems, physical systems or a combination of these basic types of actors. We call them the task holders. For example, a software system can automatically perform the "tax rate calculation" action, while a person uses a software program to perform the "order entry" activity. The person enters the order data via a screen mask. The software checks the entered data for plausibility and saves it. However, activities can also be carried out purely manually, for example when a warehouse worker receives a picking order on paper, executes it, marks it as executed on the order form and returns it to the warehouse manager.\\ 
When creating a system model, it is often not yet known which types of actors execute which actions. Therefore, it can be useful to abstract from said model when starting to describe processes by introducing abstract actors. A modeling language should allow the use of such abstractions. This means that when defining the process logic, no assertion should have to be made about what type of actor is realized. In S-BPM, the subjects represent abstract actors. \\
In the description of the control logic of a process, the individual activities are also described independently of their implementation. For example, for the action "create a picking order" it is not specified whether a human actor fills in a paper form or a screen mask, or whether a software system generates this form automatically. Thus, with activities the means by which something happens is not described, but rather only what happens.\\ 
The means are of course related to the implementation type of the actor. As soon as it has been defined which types of actors are assigned to the individual actions, the manner of realization of an activity has also been defined. In addition, the logical or physical object on which an action is executed also needs to be determined. Logical objects are data structures whose data is manipulated by activities. Paper forms represent a mixture between logical and physical objects, while a workpiece on which the "deburring" action takes place is a purely physical object. Therefore, there is a close relationship between the type of task holder, the actions and the associated objects actors manipulate or use when performing actions.\\
A system model can be used in different areas. The process logic is applied unchanged in the respective areas. However, it may be necessary to implement the individual actors and actions differently. Thus, in one environment certain actions could be performed by humans and in another the same actions could be performed by software systems. In the following, we refer to such different environments of use for a system model as context. Hence, for a process model, varying contexts can exist, in which there are different realization types for actors and actions.\\ 
In Subject Oriented Modeling, actors are not assigned to individual activities, but rather the actor type is assigned to an entire subject. This assignment is not part of the process logic, but in the most simple way it is done instead for each process in a separate two-column table. The left column contains the subject name and the right col-umn the implementation type. If there are several contexts for a model, a separate assignment table is created for each of them.
The assignment of the implementation type forms the transition between the system logic and its implementation. Subsequently, it has to be defined which persons, software systems and physical systems represent the actors and how the individual actions are concretely realized. These aspects are described in detail in the following subsections.

\section{People and organizations}


\section{Physical infrastructure}

\section{IT-Systems and Software}


% Review Christian Stary January 2020

\chapter{Aspects for further standardisation activities}

CS: Hochkommas und - im pdf falsch gedruckt.

In this chapterr various aspects of the subject oriented modelling and programming concept are outlined. These aspects have already been published on different conferences. The following sections are based on these publications. CS: They contain original text parts and thus, the conclusions need to be aligned to the standardization effor, as tried for Fog Computing.
The concepts described in theses sections will be part of future standardisation activities.
The following sections are based on following publications:\\
\begin{list}{-}
	\item Subjects and Shared Input Pools:
	\item Hierarchies in Communication Oriented Business Process Models:
	\item Business Activity Monitoring for S-BPM: \cite{article:SubProcessMon}
	\item Subject Oriented Project Management:
	\item Subject-oriented Fog Computing: \cite{article:FogComp}
	\item Activity based Costing \cite{article:SBPMCosting}
\end{list}

\section{Subjects and Shared Input Pools}

Shared input pools have the same structure like subject-specific ones, and thus, the same properties like the standard input pool. The only difference is that different subjects can deposit in or remove messages from a shared input pool. Subjects that want to send a message via a shared input pool do not use a subject name as addressee of a message, but the name of a shared input pool instead. In a distributed system several shared input pools for different purposes can be used. Figure 7 shows the slightly changed structure of the traffic management system when operating it with a shared input pool CS: hier fehlt das originäre Beispiel als Bezugspunkt. The subject "Car detection" represents the shared input pool.


\begin{figure}[htbp]
	\centering
	\includegraphics[width=0.7\linewidth]{Figures/Chapter5/figuresshared/SharedInputPoolExample.jpg}
	\caption[Traffic Management System with Shared Input Pool]{Traffic Management System with Shared Input Pool}
	\label{fig:SharedInputPooTraffic}
\end{figure}


Shared input pools make a distributed system more flexible when additional participants or nodes are added. For instance, a third intersection control could be added to the traffic management system without much effort. In this case, only the additional detectors and the components for controlling the intersection have to be complemented and linked to the shared input pool. The extension would have no impact on the behavior of the other subjects and their behavior in that system.
There is one additional attribute for shared input pool: It defines whether a message will be removed from the input pool once a message has been picked up by a receiving subject. This mechanism is required, since several subject may need to process a particular message. In addition, it allows keeping historical information in the input pool, in particular for analyzing the content of an input pool independently of the behavior of interacting subjects. 
The messages of an input pool can be analyzed with respect to certain patterns of its messages. In order to perform such an analysis, Complex Event Processing (CEP) concepts can be applied. Complex Event Processing can be encapsulated in a subject. A subject of this kind scans the messages of a shared input pool and checks whether patterns of interest can be found. Once such a pattern is identified, a message including the discovered pattern can be sent to other participants, and initiate further activities. Figure 8 shows the traffic management example enriched with subjects processing complex events.
In the example, the subject "CEP pollution analyzer" can analyze the time between cars passing the intersection in a certain time period. It can identify the events "low traffic" or "high traffic" and send it to the subject "Environment management". In case of tunnels, the subject "Environment management" might react to this information in a different way compared to open air settings. 


\begin{figure}[htbp]
	\centering
	\includegraphics[width=0.7\linewidth]{Figures/Chapter5/figuresshared/SharedInputPoolEvent.jpg}
	\caption[Shared Input Pools and Complex Event Processing]{Shared Input Pools and Complex Event Processing}
	\label{fig:sharedInputPoolEvents}
\end{figure}

\subsection{Implementing Shared Input Pools}
As mentioned earlier, shared data repositories represent a single point of failure of a distributed system. A malfunction of a shared data storage component or device may have a significant impact on the functionality of the whole distributed system. If a subject or a communication line is disturbed, only a small part of a system may be concerned but if a shared data store is down this has an impact on all subjects accessing this input pool.
\\
In addition to this operational problem it must be decided in the course of organizational implementation which organization is held responsible for running and maintaining the system hosting the shared data. Such issues become prominent, if a distributed system is connecting several independent organizations, e.g., different companies in a supply chain. Distributed systems run by independent organizations may also have to deal with several changes dynamically, affecting the data quality and system stability. Even companies can be replaced by other organizations. If only functional subjects are concerned, such a change can be managed without affecting the operation of the entire system: The execution of a subject is just assigned to the new actor. The problem is more serious if a company leaving a distributed system is responsible for running the system with shared data, as other participants of the shared system are affected. Then, a new company still part of the distributed system must take over the responsibility for the shared data. The migration of these data from one company to another can become very cumbersome from the business point of view and from a technological perspective, too.
One way to solve these problems is implementing shared input pools with blockchain technology. A blockchain is an open, distributed ledger that can record transactions efficiently in a verifiable and permanent way. Blockchains allow to achieve the integrity of a collection of data in a distributed peer-to-peer system, whereas the number of the peers is unknown and an unknown number of them are not reliable and trustworthy \cite{book:Blockchainbasics}. 
\\
Today, blockchains are mainly used for managing the ownership of money, goods, real estates, etc. Each participant in a distributed system may have a copy of a blockchain. Changes in a blockchain follow a mechanism which manage changes in a consistent way and the change protocol guarantees that any participant will have again a consistent copy after a change. A change of a blockchain means that a new data record is added, and nothing can be removed from a block chain. Adding a new block to a block chain requires some effort from parties involved in a blockchain. This effort is rewarded by adding crypto money to the party when having accomplished the task successfully. These rewards serve as an incentive for the creators of blocks. 
\\
Although heavily questioned with respect to effort and gains by practitioners \cite{article:BlockchainUniverse} blockchain technology provides concepts ensuring the trustworthiness of system components. The latter becomes crucial when operating sensitive distributed systems, such as public transportation and healthcare, in particular when event-based data fusion is needed, where nodes of various type (sensor systems, vendor-specific monitoring systems. user devices, household items, etc.) exchange notifications of events and decision-relevant data with each other. In such settings, not only notification mechanisms needs to be streamlined in case of heterogeneity of nodes, but also data source trust is important for further processing and system behavior \cite{article:EventbasedSensor}.
\\ 
In order to ensure dependable sharing of data, these basic properties of blockchains need to be adapted to the requirements of a shared input pool. Hence, a blockchain-oriented implementation of a shared input pool must meet several requirements:
\begin{enumerate}
	\item Subjects can subscribe for the access to a shared input pool.
	\item Subjects subscribed for an input pool may deposit or read events from that input pool. 
	\item Events can be marked as removed from a shared input pool.
	\item Subjects may analyze the content of a blockchain, e.g., when processing complex events.
	\item There must be a mechanism that a block chain can be deleted, once all involved parties agree on that.
\end{enumerate}

Traditionally data received from "things" are not very complex. These data are mainly values as measured by sensors, or binary signals. This may lead to a paradox situation: If such simple data are to be stored in a blockchain, the fee to be paid for adding blocks containing simple data is larger than the value being transferred. 
One way to solve the resulting incentive problem is to use permissioned block chains instead of open block chains: Blockchains for dedicated distributed application are not open blockchains like the ones implementing the management of digital currencies. 
\\
For the implementation of shared input pools, we suggest managed or permissioned blockchains. For instance, Hyperledger Fabric \cite{article:hyperledger} is an open source implementation of a permissioned blockchain. Unlike to a public permissionless network, the participants are known to each other, rather than staying anonymous and interacting untrusted. It means, while the participants may not fully trust one another, e.g., in case of being competitors in the same industry sector, a network can be operated under a governance model that is built on the extent of trust existing between participants, such as a legal agreement or framework for handling disputes. When building a business process with known participants, such type of a blockchain implementation would be sufficient. Consensus algorithms for permissioned blockchains are faster and do need much less energy than permissionless blockchain networks. 
\\
In \cite{article:Blockbench} it is reported that hyperledger fabric is the fastest available permissioned blockchain. The transaction throughput could even be increased from 3,000 to 20,000 transactions per second \cite{article:hyperledgerfabric}.
\\
When using Hyperledger to create blockchain networks of that kind, a hyperledger blockchain network provides a technical infrastructure offering ledger and smart contract (chaincode) services to applications. Primarily, smart contracts are used to generate transactions which are subsequently distributed to each peer node in the network where they are immutably recorded on their copy of the ledger. The users of applications can be users of client applications or blockchain network administrators.
\\
Subject add messages to the shared input pool and other subjects want to read these messages. If a shared input pool is implemented as a blockchain it is necessary that the chain code (smart contract in Ethereum) realizing the functions of the shared input pool must interact with the world outside the block chain. In hyper ledger fabric (including Ethereum), this problem is solved by so called oracles. We suggest using the blockchain patterns Oracle and Reverse Oracle as described in \cite{book:Blockchainapplications}. For flexibility reasons we prefer off chain oracles - see figure \ref{fig:sharedblockchain}.



\begin{figure}[htbp]
	\centering
	\includegraphics[width=0.7\linewidth]{Figures/Chapter5/figuresshared/Block-Chain.jpg}
	\caption[Utilizing block chain patterns Oracle and Reverse Oracle]{Utilizing block chain patterns Oracle and Reverse Oracle}
	\label{fig:sharedblockchain}
\end{figure}

\subsection {Conclusion}
The more the Internet of Things (IoT) propagates into domain-specific applications, the more stakeholders get involved with respect to business and user requirements. They expect omnipresent use and adaptation on demand. Ensuring robust and semantically correct operation in dynamically networked IoT environments requires tools and development methods to handle complex patterns of interactions due to the different components and capabilities of actors.\\
These patterns refer to the (reactive) flow of control and correct exchange of data. We have proposed an integrated approach based on subject-oriented process models. These role-specific representations allow behavior abstractions on various levels of granularity and can be enriched with a mechanism for handling complex events and sharing data. The data handling mechanism is bound to exchanging messages and a blackboard-like structure. Its behavior can be implemented through blockchain technologies, in case a single point of failure in system operation should be inhibited. The latter is of crucial importance, once the data exchange between IoT-system elements should be trustworthy and traceable.\\
The presented approach should facilitate transparent development and stakeholder understanding of (complex) IoT systems in dynamic settings, due to the implementation-independent representation on a mainly diagrammatic level based on a minimalistic notation, stemming from subject-oriented modeling. Abstractions and decomposition into IoT system components encapsulate behavior. The overall behavior of an IoT system is determined by a set of interactions that integrates the control flow with data exchange patterns from a semantic process perspective. Application design can be understood as top-down approach with the functionality specific to the IoT application residing on an edge operating system. Platform services implement all functional requirements, and are backed by communication and information processing technologies. Cross-functional issues, such as secure operation, business-relevant standardization, and critical event handling can be explicated on an implementation-independent level due to the semantic process representation scheme. The resulting models are executable and thus, can be adapted dynamically.


\subsection{Future Work}
Due to the novel conceptual integration addressed, several aspects and topics need to be addressed by future research:
\begin{list}{-}{spacing}
\item From an application perspective, the results need to be aligned with novel industry 4.0 concepts (cf. [29]), since there not only existing standards are framed by business processes, but also distributed operation of production-relevant processes and real-time sharing of data.
\item From an implementation perspective, our approach requires a (prototypical) realization of an appropriate block chain mechanism for managing shared input pools meeting all requirements in section 4.
\item From an industry perspective, performance evaluations might lead to reconsider our conceptual findings, e.g., how to manage a shared input pool of a distributed system in real time.
\item Definition of structural semantics in OWL
\item Definition of execution semantics in ASM
\end{list}

\section{Hierarchies in Communication Oriented Business Process Models}
PASS  offers powerful possibilities for structuring complex process systems. The ways to do that are demonstrated with an example.
As an example we will consider a process for realizing a car break down service. This service consists of several connected processes. There is the main process for handling the car accident and supporting e.g. processes for organising towing and repair shop services. Insurance companies may be involved for covering damages, the customer gets an invoice, uses money transfer services or banks for paying the invoice. These processes are executed by various organisations like help desk service companies, towing service companies, car repair workshops banks etc.. In most business process projects not all parts of processes are described in detail. Only a certain part is considered, e.g. only the help desk process has to be considered in detail. In order to do so we have to consider the whole environment in which a considered process is embedded. We have to know which relations exists to these other processes. It is necessary to know which inputs are rquired by neighbour processes and which results they deliver. A help desk process which organizes the towing services has to know how the towing service is requested and which further interactions are required. For instance it must be agreed whether the towing service informs the client about the arrival time of the towing truck or the help desk does it.


\subsection{Process Architecture}

Rectangles represent processes. Each process has a name. Processes consists of other processes and/or subjects. The lines between the rectangles represent the communication channels between processes. Each communication channel has a nameand can contain other communication chan-nels and/or messages.

Figure \ref{fig:car-service-level1} shows the highest process level of the car break down service. In the "car use" process the event "car break down" happens. In order to organize support an interaction is initiated with process "car break down service" . Between these processes messages are exchanged which are elements of the communication channel "Car break down handling".\\


\begin{figure}[htbp]
	\centering
	\includegraphics[width=0.7\linewidth]{Figures/Chapter5/figures-hierarchy/Car-Service-Level1.jpg}
	\caption[High level structure of car break down service]{High level structure of car break down service}
	\label{fig:car-service-level1}
\end{figure}



Figure \ref{fig:car-service-leve2} shows the next process structure level of the process "car break down ser-vice". In this level the process "Car break down service" is spltid in 10 processes. The processes "Bank", "Insurance service", "Car repair workshop", "Incident Management","Mobility Manage-ment" and "Towing Management" have a communication channel to the prcess "Car usage". This means the communication channel "Car break down handling" is split into five communica-tion channels. Each of them covers the communication with the relared process, e.g. the communi-cation channel "Accident notification Car break down" is the communication channel between the processes "Car usage" and "Incident Management".\\


\begin{figure*}[htbp]
	\centering
	\includegraphics[width=0.8\linewidth]{Figures/Chapter5/figures-hierarchy/Car-Service-Leve2}
	\caption[Structure of the Emmergency Call Handling Process]{Structure of the Emmergency Call Handling Process}
	\label{fig:car-service-leve2}
\end{figure*}



Inside a process there can be also processes. This means that levels of processes can be built. Figure \ref{fig:car-service-lev3} shows the next deeper level of our process hierarchy. The process "Car repair workshop" is structured in six processes. According to this separation the communication sets are also splitted e.g. the communication set "Handling repair service" is splitted into three parts, one part is han-dled by the process "Service scheduling" the other by the process "Car droping" and the third one by the process "Customer Satisfaction".\\

\begin{figure*}[htbp]
	\centering
	\includegraphics[width=0.8\linewidth]{Figures/Chapter5/figures-hierarchy/Car-Service-Lev3}
	\caption[Details of the "Car repair workshop" Process]{Details of the "Car repair workshop" Process}
	\label{fig:car-service-lev3}
\end{figure*}

As already mentioned, processes cannot communicate directly with each other. The active entities of a process, the subjects communicate with each other. This means messages from one process are sent to an other process are reveived by a subject inside of that process. Messages belonging to a channel are assigned to a sending or receiving subject at the lowest level of a process architecture. This lowest level of a process description is the subject interaction diagram (SID) which shows the involved subjects of a process and the messages they exchange. In the following we consider the process incident management in more detail. This process does not contain other processes like the process "Car Repair Shop". The process "Incident management" contains a Subject Interaction Diagram. Some of the subjects of a process communicate with subjects in other processes. These subjects are called border subjects because they are at the border of a process to other prcesses. Figure \ref{fig:car-service-lev4} shows the process "Incident management" with its border subjects. There is a border subject "Help agent" which communicates with the processes "Towing service", "Mobility ser-vice"and  "Car repair workshop", precisely it communicates with a subject in one of these processes. Another border subject of the process "Incident management" which is called "Help desk"communicates with the process "Car usage".\\

\begin{figure}[htbp]
	\centering
	\includegraphics[width=0.9\linewidth]{Figures/Chapter5/figures-hierarchy/Car-Service-Lev4}
	\caption[Neighbors of the "Incident Manaement Process"]{Neighbors of the "Incident Manaement Process"}
	\label{fig:car-service-lev4}
\end{figure}

The border subjects of the process "Incident management" must have a coresponding border sub-ject at the neighbour processes. The border subjects "Call agent" communicates with the border subject "Help requestor" of process "Car usage" and the border subject "Help agent" communi-cates with border subjects of the processes "Car repair workshop", "Towing service and "Mobility service". The process "Incident management" with all the border subjects is shown in  figure \ref{fig:car-service-lev5}.\\

\begin{figure}[htbp]
	\centering
	\includegraphics[width=0.9\linewidth]{Figures/Chapter5/figures-hierarchy/Car-Service-Lev5}
	\caption[Border subjects of the "Incident Management" Process]{Border subjects of the "Incident Management" Process}
	\label{fig:car-service-lev5}
\end{figure}

The border subjects of the processes "Mobility service", "Towing service" and "Car repair work-shop" have the same name “Service agent” but these are different subjets because they belong to different processes. Because the process "Car repair workshop" consists of several layers the corre-sponding border subject can be in a process which is part of process "Car repair workshop" in a lower level.\\
From the perspective of the subjects inside of the process "Incident managent" are the border subjects of the processes "Mobility service", "Towing service" and "Car repair workshop" interfaces to these processes, therefore they are called interface subjects in the Subject Interaction Diegram of a process. Figure \ref{fig:car-service-lev6}shows the Subject Interaction Diagram of the process incident management.\\


\subsection{Behavioral Interface}
Processes to which a considered process has communication relationships are called process neighbours or for short neighbours. Now we want to consider the details of the communication relationships between two neighbours. The interface between two processes is defined by the related border subjects and the allowed sequences in which the messages are exchanged between them in a communication channel. As already described above each message is defined by a name and the data which are transported the so-called payload. A border subject observes the behavior of the border subject of the neighbour process and vice versa. Figure \ref{fig:car-service-lev8} shows the border subject "Help desk" of the processes "Incident Management" which communicates with the border subject of process "Car usage".\\

\begin{figure}[htbp]
	\centering
	\includegraphics[width=1.0\linewidth]{Figures/Chapter5/figures-hierarchy/Car-Service-Lev6}
	\caption[Subject Interaction Diagram of the Process "Incident Management"]{Subject Interaction Diagram of the Process "Incident Management"}
	\label{fig:car-service-lev6}
\end{figure}

Because we consider the process "Incident management" the border subject "Caller" of the process "Car usage" becomes an interface subject in the SID (details about interface subjects can be found in \cite{Flei12}) of the process “Incident Management”. Figure \ref{fig:car-service-lev8} shows the detailed Subject Interaction Diagram around the subject help desk. \\

\begin{figure}[htbp]
	\centering
	\includegraphics[width=0.9\linewidth]{Figures/Chapter5/figures-hierarchy/Car-Service-Lev8}
	\caption[Subject Interaction around the subject “Help desk”]{Subject Interaction around the subject “Help desk”}
	\label{fig:car-service-lev8}
\end{figure}

Instead of the channels the messages required for a towing service request are shown. A message "Request towing service" comes from the interface subject. This message is accepted by the subject "help desk". The subject help desk checks the customer data received with this message by sending a corresponding the message "Get customer data" to the subject "Customer data management". This subject send the complete customer data back to the subject "Help desk" via the message "Customer data". The subject "Help desk" checks the customer data. If the data are invalid a message "Invalid customer data" is sent to the subject "Caller" and the process is finished.\ 
If the customer data are valid with that data the subject "Help desk" creates a trouble ticket which is sent to the subject "Ticket management". After that the message "Towing service requested" is sent to the help agent which organizes the towing service. The part of the communication structure of the subject "Help agent" in order to organize the towing service is not shown in figure \ref{fig:car-service-lev9}. We only see that subject "Help agent" sends the message "Towing service data" to the subject "Help desk". This message contains all the data about the service e.g. name of the towing company and arrival time. The subject "Help desk" forwards that data to the interface subject "Caller". This behavior is shown in figure \ref{fig:car-service-lev8}.\\



\begin{figure}[htbp]
	\centering
	\includegraphics[width=0.7\linewidth]{Figures/Chapter5/figures-hierarchy/Car-Service-Lev9}
	\caption[Part of the Behavior Diagramm of the subject “Help desk”]{Part of the Behavior Diagramm of the subject “Help desk”}
	\label{fig:car-service-lev9}
\end{figure}


The behavior described in the figure above contains the communication with all neighbor subjects of subject "Help desk" including the communication with the interface subject "Caller". From the perspective of this subject the communication of the subject "Help desk" with its other neighbor subjects is not relevant. For the subject "Caller" only the commumication sequence between itself and the subject "Help desk" is relevant. These allowed communication sequences are called the behavioral interface.\\
The behavioral interface between two subjects can be derived from the complete behavior of a subject by deleting the interactions with all the other subjects . Figure \ref{fig:car-service-lev10} shows how the communication sequence relevant for the communication be-tween the subject "Help Desk" and "Caller" is derived from the complete behavior of subject "Help desk".

\begin{figure}[htbp]
	\centering
	\includegraphics[width=1.0\linewidth]{Figures/Chapter5/figures-hierarchy/Car-Service-Lev10}
	\caption[Deriving the Behavioral Interface from the Subject Behavior]{Deriving the Behavioral Interface from the Subject Behavior}
	\label{fig:car-service-lev10}
\end{figure}

A behavoral interface is always relative to a communication partner. In figure \ref{fig:car-service-lev10} the behavioral interface is relative to the interface subject "Caller". The behavioral interface to the subject "Ticket Management" is different because only the communication activities with this subject are considered.This behavioral interface would be very simple. It consists of only one send activity, sending the message "Store ticket".\\
The behavioral interface relative to a partner subject can be automatically derived from the complete behavior of a subject
(see \cite{article:jCPEX}).

\subsection{Future Work}

Due to the novel conceptual integration addressed, several aspects and topics need to be addressed by future research:
\begin{list}{-}{spacing}
	\item Clarify terminology e.g. using the term interface subject, system interface, implementation
	\item Definition of structural semantics in OWL
	\item Definition of execution semantics in ASM. The semantic of the behaiour interface and its relation to the behavior of the related subject has to be described.
\end{list}


\section{Business Activity Monitoring for S-BPM}\label{sec:BAMinSubjectOrientation}

Monitoring of Business Process looks at running instances. For those it measures metrics, aggregates them to Process Performance Indicators (PPIs) as a business process-related form of Key Performance Indicators (KPIs), reveals deviations (as-is vs. to-be) and report and presents results to people in charge or interested in the value of the PPI. Thus monitoring lays ground for the performance analysis in the key dimensions quality, time and costs of processes and helps identifying weaknesses and opportunities for improvement \cite{book:UntPerform}.
By feeding back information for completed and running instances to analysis monitoring fosters organizational learning, forms an important part of the Business Process Management (BPM) lifecycle \cite{article:SUbjetorientiertBPM} and thus helps implementing the operational level in the closed-loop approach to enterprise performance management \cite{book:processmonitoring} (see figure \ref{fig:Approach-Performance}).
\\


\begin{figure}[htbp]
	\centering
	\includegraphics[width=0.8\linewidth]{Figures/Chapter5/Monitoring/Approach-Performance-Mgmt.jpg}
	\caption[Closed-loop Approach to Performance Management]{Closed-loop Approach to Performance Management \cite{book:AnalytInfSys}}
	\label{fig:Approach-Performance}
\end{figure}



\subsection{Architecture }  
A Business Activity Monitoring (BAM) environment supported by Complex Event Processing consists of several elements necessary at build time and at runtime (see figure \ref{fig:BAMArchitecture}) and \cite{book:processmonitoring}, \cite{book:CEPinAction} , \cite{article:BlueprintEventBPM}). At build time a modeling environment should provide tools for designing processes (e.g. Metasonic Build) and defining process performance indicators (PPIs), BAM events, rules, thresholds etc. as well as parameters for their visualization in report and on dashboards. At runtime there are (1) event producers like a process engine (e.g. Metasonic Flow) or an ERP system (e.g. SAP) which feed events into an event cloud or stream (chronologically ordered). (2) Event Processing Agents (EPA) form the event processing logic. They process events based on metrics, event patterns, rules and other parameters specified at design time. Their basic logical functions include filtering and transforming events and detecting patterns among them. Global state elements allow them accessing data from outside the application (e.g. from an ERP system). EPAs put the results of their processing (also to be understood as events) out to Event Consumers (3) like dashboards or process engines. Input and Output Adapters (IA, OA) transform event data between different formats of system elements as necessary. All system elements involved form an Event Processing Network (EPN), in which events are exchanged by communication mechanisms.

\begin{figure}[htbp]
	\centering
	\includegraphics[width=0.9\linewidth]{Figures/Chapter5/Monitoring/Integrated-BAM-CEP-Architecture-27.jpg}
	\caption[Integrated BAM/CEP Architecture 27]{Integrated BAM/CEP Architecture \cite{book:processmonitoring}}
	\label{fig:BAMArchitecture}
\end{figure}



\subsection{Modeling BAM Parameters at Build Time}
As mentioned in the last section it is necessary not only to model the processes, but also numerous pieces of information relevant for a sound process monitoring in the sense of Business Activity Monitoring (BAM model). These can be derived from answers to questions like what, when, how and how often should be measured by whom \cite{book:ProzesseSchmelzer}. The information should also include how single metrics are to be aggregated in order to determine Process Performance Indicators (PPIs). For systematically collecting and documenting the necessary information fact sheets or templates for metrics and performance indicators have been developed \cite{book:KennzahlenIT}, \cite{book:ITControlling}. Figure \ref{tbl:Fact-Sheet}  shows an extract of a sample fact sheet defined for the average processing time of activities (see also \cite{article:SBPMCosting}, \cite{book:MonitoringSubjekt} ).


\begin{table}[htbp]
	\footnotesize
	\centering
	\begin{tabular}[t]{@{}l p{0.5\linewidth} p{0.3\linewidth} @{}}
		\toprule
		\textbf{Attribute} & \textbf{Content}  \\
		\midrule
		 & \textbf{Characteristics}
		\\
		Identifier & Average activity time
		\\
		Description & Average time of a process activity within a certain period
		\\
		To-be value/unit & tbd specifically (min.)
		\\
		Tolerance range/unit & tbd specifically (\%)
		\\
		Escalation Rules/ Actions & In case of violation alert the process owner and start escalation process (tbd specifically)
		\\
		Addressees & Process Owner, Middle Mangement, Accountants (tbd specifically)
		Responsibility	Process Owner (tbd specifically)
		\\
		&  &
		\\
		& \textbf{Measuring and Computing}
		\\
		Measuring Object & All instances of the process 'Purchase Order'
		\\
		(Single) Metrics & Start time and end time of all activities of the process
		\\
		Measuring Method & Read time stamps for beginning and end of activities written by Metasonic Flow 
		\\
		Measuring Frequency & For every single instance as it occurs
		\\
		Algorithms & For computing period: Sum of processing time of all activities divided by number of instances

		\\
		Data Sources (general) & Tables in the database of Metasonic Suite:
		RT\_PROCDESC, RT\_PROCINST, REC\_PARADESC, REC\_RECTRANS, UM\_USER
		\\
		Data Sources (specific) & Activity processing time (for one instance):\newline
		\textbf{SELECT} TIMESTAMP1  \newline
		(\textbf{SELECT} STARTTIME \newline
		\textbf{FROM} RT\_PROCINST \newline
		\textbf{WHERE} RT\_PROCDESC = \textit{process (purchase order)}\newline
		\textbf{AND} ID = \textit{instance (9)}\newline
		\textbf{FROM} REC\_RECTRANS\newline
		\textbf{WHERE} RT\_STDESC = \textit{\textit{state (fill\_in\_form)}}\newline
		\textbf{AND} RT\_PROCINST = \textit{instance (9)}
		Completed instances: see separate fact sheet .
		\\
		Computing Period (time, no. of inst.) & Daily
		\\
		& \textbf{Presentation}
		\\
		Presentation Style & As-is value and to-be value in combination with a spark line showing the historical development, deviation from to-be value in \%
		\\
		Presentation Frequency & Weekly and in case of escalation
		\\
		Archiving & Stored in additional database table, linked with RT\_PROCDESC
		\\
		
\bottomrule
\end{tabular}
\caption{Fact Sheet for a PPI (extract)}
\label{tbl:Fact-Sheet}
\end{table}

Replacing the content column by more formal ontology-based linguistic patterns as suggested by Del-Rio-Ortega et al. (see table \ref{tbl:Fact-Sheet-PPI}) could help relating PPIs to elements of the process model, performing automated analysis \cite{article:ProcessPerfInd} and implementing the measurement at runtime. 

\begin{table}[htbp]
	\footnotesize
	\centering
	\begin{tabular}[t]{@{}l p{0.3\linewidth} p{0.4\linewidth} p{0.5\linewidth} @{}}
		\toprule
		\textbf{Attribute} & \textbf{Linguistic Pattern}  & \textbf{Example}\\
		\midrule
		PPI-<ID> & <PPI descriptive name> & PPI-001 Average time of RFC analysis
		\\
		Process	& <process ID the PPI is related to> & Request for change (RFC)
		\\
		Goals & <strategic or operational goals the PPI is related to> & BG-002: Improve customer satisfaction \newline
		BG-014: Reduce RFC time to response
		\\
		Definition & The PPI is defined as { \newline
			<DurationMeasure> | <CountMeasure> | <ConditionMeasure> |
			<DataMeasure> | <DerivedMeasure> | <AggregatedMeasure> }
		[expressed in <unit of measure>] & The PPI is defined as the average of Duration of Analyse RFC activity
		\\
		Target & The PPI value must { \newline
			be {greater | lower} than [or equal to] <bound> | \newline
			be between <lower bound> and <upper bound> [inclusive] |\newline
			fulfil the following constraint: <target constraint> } & The PPI value must be slower than or equal to 1 working day
		\\
		Scope & The process instances considered for this PPI are {
			the last <n> ones |
			those in the analysis period <AP-x> } & The process instances considered for this PPI are the last 100 ones
		\\
		Source & <source from which the PPI measure can be obtained> &	Event logs of BPMS
		\\
		Responsible & { <role> | <department> | <organization> | <person> } &	Planning and quality manager
		\\
		Informed &{ <role> | <department> | <organization> | <person> } & CIO
		\\
		Comments & <additional comments about the PPI> & Most RFCs are created after 12:00
		\\
\bottomrule
\end{tabular}
\caption{PPI Template based on Linguistic Patterns \cite{article:ProcessPerfInd}}
\label{tbl:Fact-Sheet-PPI}
\end{table}
Friedenstab et al. argue that such linguistic patterns do not fit to the usually graphical modeling of processes which makes integration difficult \cite{article:BPMNActivityMon}. The authors discuss some more approaches to BAM modeling. With regard to the limitations revealed, they present a BAM-related extension of the graphical Business Process Model Notation (BPMN) \cite{article:BPMNActivityMon}.
Using an abstract language syntax based on the Unified Modeling Language (UML) they started defining meta models for language constructs needed for BAM as there are Duration, Frequency, Composed Basic Measure, Aggregated Measure, Filter, Target Definition, Actions, Measure-based Expressions and Dashboard. Figure \ref{fig:Meta-Model} depicts the example for the duration of elements on different levels of detail, where the grey colored parts indicate references to the BPMN specification.

\begin{figure}[htbp]
	\centering
	\includegraphics[width=0.9\linewidth]{Figures/Chapter5/Monitoring/Meta-Mode-fo-Duration-relate-to-BPMN-1.jpg}
	\caption[Meta Model for Duration (related to BPMN) 12]{Meta Model for Duration (related to BPMN) \cite{article:BPMNActivityMon}}
	\label{fig:Meta-Model}
\end{figure}


In a second step Friedenstab et al. developed a concrete syntax allowing for modeling the abstract language elements with graphical symbols and text labels. Parts of it are visible in figure \ref{fig:Model-Cycle-Times}. The example shows the BAM model for determining the cycle times of a purchase order process modeled in BPMN (lower part). The upmost part for example expresses the fact that the overall cycle time (Duration) for the last 50 instances (Filter) has to be determined and displayed on the dashboard (Dashboard). Monitoring the average of the overall cycle time for completed instances controls the modeled business logic of the process. If it is above 48 hours goods are delivered with an express shipping if the average cycle time is more than. Otherwise standard shipping is carried out. A deviation also leads to an alert sent to the process owner, while in any case the average is to be presented on the dashboard. The latter is also valid for the third time-related metric in the example, the partial cycle-time for the company-internal part of the process, which is set into relation with the overall cycle time.

\begin{figure}[htbp]
	\centering
	\includegraphics[width=0.9\linewidth]{Figures/Chapter5/Monitoring/BAM Model for Cycle Times.jpg}
	\caption[BAM Model for Cycle Times of a Purchase Order Process based on BPMN 12]{BAM Model for Cycle Times of a Purchase Order Process based on BPMN \cite{article:BPMNActivityMon}}
	\label{fig:Model-Cycle-Times}
\end{figure}


The concept presented by Friedenstab et al. is thoroughly thought-out and clearly and precisely elaborated. The idea now is to adapt it to Subject-oriented Business Process Management and relate the abstract syntax to the S-BPM meta model instead of BPMN. Due to S-BPM being a more precise and comprehensive notation than BPMN \cite{article:BPMNYAWLPatterns} the mapping should be possible without problems. Table \ref{tbl:MonBPMNSBPM} compares the BPMN specification elements used by \cite{article:BPMNActivityMon} with the ones appropriate in S-BPM \cite{Flei12}.


\begin{table}[htbp]
	\footnotesize
	\centering
	\begin{tabular}[t]{@{}l p{0.3\linewidth} p{0.4\linewidth} p{0.5\linewidth} @{}}
	\toprule
	\textbf{BAM Language Syntax Construct} & \textbf{Used BPMN Specification Element}  & \textbf{Suitable S-BPM Specification Element}\\
	\midrule\\
	Duration (Time-Consuming Element) &	Process, Activity, Flow Nodes&	Process, Subject Behaviour States (Function, Send, Receive, Start, End)
	\\
	Frequency
	(Countable Element)&	Process, Activity, Data Objects, Data States &	Process, Subject Behaviour States (Function, Send, Receive), Business Objects and their States 
	\\
	Actions &	Process	 & Process
	\\
	Measure-based Expressions &	Expression, Sequence Flow &	Incoming Message
	\\
	\bottomrule
\end{tabular}
\caption{BPMN and S-BPM Specifications used in BAM Constructs}
\label{tbl:MonBPMNSBPM}
\end{table}

The remaining constructs as well as the extensions do not depend on the process modeling language and thus are not included in the table.
On the other hand S-BPM, following its paradigm of regarding subjects, predicates and objects as equally important parts of a process, offers the subject as an additional specification element to add . In figure \ref{fig:Meta-Model-S_BPM} we modified the picture of figure \ref{fig:Meta-Model} by replacing the BPMN by S-BPM elements and adding the subject. This allows modeling the determination of the overall time a subject (respectively the allocated resource(s)) spends on working on a process instance. This is of interest for cost-related analysis.

\begin{figure}[htbp]
	\centering
	\includegraphics[width=0.9\linewidth]{Figures/Chapter5/Monitoring/Meta-Mode-fo-Duration-relate- to-SBPM.jpg}
	\caption[Meta Model for Duration (related to S-BPM)]{Meta Model for Duration (related to S-BPM)}
	\label{fig:Meta-Model-S_BPM}
\end{figure}


In order to show how the BAM language syntax constructs can be related to subject-oriented models we designed the purchase order process in S-BPM. Due to missing information in the BPM model some assumptions were necessary like who performs the process steps (subjects). We then added the BAM modeling symbols to create a monitoring model similar to that in figure \ref{fig:Meta-Model-S_BPM}.
The result is depicted in the following graph. In the lower part it includes the subject interaction diagram (SID) of the process. The SID shows the subjects involved and how they coordinate themselves in the course of action by exchanging messages. In the monitoring model in the upper part a difference can be seen. The partial cycle time for the company-internal activities can be modeled by just relating the clock symbol to the subject "Sales". In the example this subject represents all steps carried out within the organization. In the same way we can determine the cycle time for the other subjects.

\begin{figure}[htbp]
	\centering
	\includegraphics[width=0.9\linewidth]{Figures/Chapter5/Monitoring/BAM-Model-fo- Cycle-Times-of-a-Purchase-Order-Process-based-on-S-BPM.png}
	\caption[BAM Model for Cycle Times of a Purchase Order Process based on S-BPM]{BAM Model for Cycle Times of a Purchase Order Process based on S-BPM}
	\label{fig:Cycle-Time-SBPM}
\end{figure}

Given a special information demand a more granular modeling of BAM parameters is possible on the subject behavior level. Figure \ref{fig:BAM-Cycle-Time} for example details the behavior of "Sales" including all receive, send and functional states walked through by the subject. The symbols indicate that the average cycle time between order reception and confirming the order to the customer should be measured. In the same way cycle times between states in behaviours of different subjects can be modelled.


\begin{figure}[htbp]
	\centering
	\includegraphics[width=0.9\linewidth]{Figures/Chapter5/Monitoring/BAM-Model-for-Cycle-Time-of-a-Process-Section-based-on-S-BPM.jpg}
	\caption[BAM Model for Cycle Time of a Process Section based on S-BPM]{BAM Model for Cycle Time of a Process Section based on S-BPM}
	\label{fig:BAM-Cycle-Time}
\end{figure}

Back on the level of subject interaction diagram we could also model to determine the overall time for receiving (waiting), sending and doing, both by process and by subject. Modeling on the two diagram levels reduces complexity.

\subsection {Conclusion and future Work}
This contribution systematized Business Process Monitoring and shed some light on the current state of monitoring in the context of S-BPM. Starting there we emphasized Business Activity Monitoring and took a closer look to the modelling of BAM parameters. We showed that the approach for BPMN presented by Friedenstab et al. can be adapted to S-BPM with little effort and that S-BPM shows additional potential to further develop the concept.
\\
\subsection{Future Work}

Due to the novel conceptual integration addressed, several aspects and topics need to be addressed by future research:
\begin{list}{-}{spacing}
	\item Extension of the structural semantics in OWL with possibilities to add Process Performance Indicators
	\item ASM definition of execution semantics for throwing events if process performance indicator bounderies are violated.
	\item ASM definition of execution semantics for handling violation events 
\end{list}
 

\section{Subject Oriented Project Management}

Subject orientation is focused on networks of indeppendent systems, which coordinate their cooperation by exchanging messages. The involved system may belong to different organisations.
In our global economy enterprises cooperate around the globe in order to create services or manufacture products for customers which are also distributed all over the world. The challenge of the cooperating partners as a federation of independent systems (virtual enterprise, VE) is to establish smooth cross-enterprise communication to reach the common objectives \cite{article:VirtualEnterprise}. Information and communication technologies (ICT) are essential to create a federation of independent software systems suitable to execute business processes across the involved companies. 
\\
Figure \ref{fig:DogFoodShop} shows an example of an order-to-cash scenario where federated applications support a cross-company business process. A dog food store sells its products via internet. It commissions a transportation service provider to deliver the ordered products to the customer, who confirms the reception of the goods. The store deducts the money from the customer's bank account. The process steps are facilitated by several independent software applications and message exchanges (order, order confirmation, delivery notification etc.) enabled by respective communication systems.


\begin{figure}[htbp]
	\centering
	\includegraphics[width=0.6\linewidth] {Figures/Chapter5/Project/DogFoodShop.jpg}
	\caption[Order-to-cash scenario in a federation of enterprises and applications (simplified)]{Order-to-cash scenario in a federation of enterprises and applications (simplified)}
	\label{fig:DogFoodShop}
\end{figure}

Developing such a mutually adjusted solution by a federation of independent enterprises requires a project management approach different from traditional software development projects taking a process perspective (cf. \cite{book:ProjectHistory}). Therefore our focus is on how to implement loosely coupled systems for exchanging information between independent partners, rather than tightly coupled solutions for sharing information or other resources.
The section is structured as follows. First software development methodology and its elements are reviewed with respect to developing federated systems. This leads to our proposal of a software development approach for federated systems based on subject orientation. 
\\
\subsection{Background}

\subsubsection{\textbf{Recommendations for creating federated systems}}
When independent enterprises develop a federated system a lot of managerial and technological aspects have to be considered, particularly with respect to managing collaborative business processes. This is reflected in the following recommendations (cf. \cite{book:PMThirdWave} , \cite{ChallengesDistPM} ):
\begin{enumerate}
	\item Start the foundation of a federation and identify members.
	\item Identify and describe the business services that organizations can provide or they need from partners in service level agreements.
	\item Harmonize the enactment of collaboration by coordinating the participating organizations according to defined business processes and identify the systems required for the federation.
	\item Integrate the identified and implemented services/systems into the intended application. 
	\item Maximize the autonomy of organizations when collaborating, thereby ensuring organizations to benefit most from their own business objectives.
	\item Represent the partnerships between collaborating organizations when collaborating, and update changes in partnership.
	\item Guarantee the business privacy of organizations in the course of collaboration.
	\item Allow partners and other third parties to monitor, measure, and oversee the execution of business processes.
\end{enumerate}


\subsubsection{\textbf{Federation of enterprise information systems}}
\cite{article:VirtualEnterprise} define virtual enterprises and federations of enterprise information systems as follows: "\textit{The Enterprise partners' Virtual Enterprise (EP VE) is the federation of partners in the community that come together to achieve the goal of a federated distributed system environment, sharing their resources, and collaborating to achieve a common goal: the Federated System VE (FS VE). The partners in the federation retain autonomy over their resources, deciding which resources (personnel, resource dollars, equipment, etc.) are sharable for achieving this goal. The results of this VE are then useable by the partners in furthering their individual systems. The FS VE is seen to be a virtual system of distributed processing components (hardware and software), which are physically implemented and managed by the partners. It is a federation of the partners' systems, where each system retains its autonomy over all processing system components and sharable data/information. Retaining autonomy means defining which data or information and software/hardware assets will participate in the federation and be accessible and usable by other systems in the federation.}"\\
The definition shows that the focus is on sharable resources. This means when setting up a federation the VE members need to clarify ownership of the shared resources as well as access rights and the rights to change those. Such an approach often implies tight coupling of the involved enterprises and the related resources. Entities leaving a federation then cause difficulties with respect to separating involved systems (changing access rights) and sorting out ownership of information.
Alternatively, information can be exchanged between the partners by messages, implying only a loose coupling of the involved systems. In this case the partners only need to agree upon structure and meaning of the data, e.g., using XML schemes, and upon the implementation of the message exchange, e.g., by web services. 
\\
\subsubsection{\textbf{Software development methodology}}
\textit{"A software development methodology is a collection of procedures, techniques, tools and documentation aids which help developers to implement software systems"}  \cite{book:ISDevelopment}. It may include modeling concepts, tools for model-driven architecture, integrated development environments (IDEs) etc. The so-called magic triangle (see figure \ref{fig:Triangle}) summarizes the various aspects of a software development methodology \cite{book:SoftEng}.

\begin{figure}[htbp]
	\centering
	\includegraphics[width=0.6\linewidth] {Figures/Chapter5/Project/Triangle.jpg}
	\caption[Magic triangle of software development methodologies]{Magic triangle of software development methodologies}
	\label{fig:Triangle}
\end{figure}

Concepts and Techniques are used to create models of the software to be implemented, and are thus significantly influencing which languages, procedures and tools are utilized. The applied concept implies the artifacts to be produced, of which the executable software system is the most important one. The Language is used to create the artifacts and tools. Procedures describe the sequence in which the activities for creating the various artifacts are executed. While languages and tools can be replaced without impacting concepts and procedures, the latter are decisively determining the shape of a software development environment.
\\
\subsubsection{\textbf{Modeling concepts}}
Developing a federated system like the dog food store requires modeling cross-company business processes and the entities performing activities in these processes.
\paragraph{Business process modeling} 
There are various approaches for specifying business process models. IT implementations of those models are called process-controlled applications [7] or workflows. The modeling approaches can be distinguished in three classes: (i) Control flow-based specifications put the focus on the activities. (ii) Object-based models mainly describe business objects and the sequence of operations to manipulate them. (iii) Communication-based models focus on the active entities in a process which exchange messages in order to coordinate their work.
By their nature the latter are promising candidates for modeling federations of systems. Business Process Model and Notation (BPMN), the currently most widely discussed modeling language, contains elements for the description of control flows and communication in business processes. In the following we discuss its communication-oriented features.
To model communication BPMN provides so-called pools, each representing a process that can exchange messages with processes in other pools. Conversation diagrams are the means to describe this mechanism: However, they do not allow specifying the sequence in which messages are exchanged. Although the sequence can be captured by collaboration diagrams, the semantics of sending and receiving messages is not precisely defined. For instance, it remains unclear whether messages are exchanged synchronously or asynchronously. Additionally a certain message from a pool can only be received in a single activity state, but not in other states. Choreography diagrams in BPMN also define the allowed message sequence between pools. [8] describe a choreography-based tool for specifying global processes. The problem is that choreography specifications cannot contain data. As a consequence a modeler can only describe message sequences being covered by regular expressions, which is the lowest level in the Chomsky hierarchy. This fact makes it impossible to model a behavior like the following: Pool S sends n messages of a type X to pool R. After that S sends a message Y to R. Subsequently S expects m messages of type A from pool R, which received the n messages of type X. The reason for that is that the messages cannot be counted, because data are not allowed in BPMN choreographies.
Given these properties of BPMN this notation has significant draw backs for modeling communication, hindering the precise development of federations of systems.

\paragraph{Multi-agent systems modeling}
The term agent has multiple meanings. We follow the definition given in [9]: An agent is an entity that performs a specific activity in an environment of which it is aware and that can respond to changes. A multi-agent system (MAS) is a system where several, perhaps all, of the connected entities are agents. The most important property of agents is their controlled autonomy: They independently execute their role-specific behavior, and in multi-agent systems they communicate with each other. These properties are alike those of federated systems which therefore can be considered as multi-agent systems. This means that software development methodologies for agent-oriented software (for an overview see [8]) can help developing federations of applications.
\\
\subsubsection{\textbf{Procedures}}
Software Life Cycles (SLC) build a framework for software development procedures. All software development projects follow a series of phases. While software life cycles can be defined in many different ways, each of them comprises the following generic activities:
\begin{list}{-}{spacing}
	\item Project conception or initiation
	\item Planning
	\item Execution with specification and implementation activities
	\item Termination
\end{list}

In the traditional waterfall approach these activities are performed in the sequence shown above. Other life cycle concepts propose overlapping the development steps, suggest alternatives like the V model or agile development procedures like Extreme Programming and Scrum. \cite{article:ObjectOrientedSWdev}, \cite{book:SoftEng} and \cite{book:ISDevelopment} give an overview of the various approaches.
\\
\subsubsection{\textbf{Work break down structure (WBS)}}
The work break-down structure describes the artifacts to be created in a project in a hierarchical way. A work break-down structure element may be a product, data, service, or any activity results contained in the software life cycle or any combination thereof. A WBS also provides the necessary framework for detailed cost estimating and control along with guidance for schedule development and control. The top level of the WBS should identify the major phases and milestones of the project in a summative fashion. Consequently, the phases used in the top level depend on the software development methodology applied in a project. The first level can either represent the phases used in the software life cycle or the major artifacts of the system to be developed. In case the top level is SLC-oriented it might be built by requirement specification, software architecture, programming, test etc. In the case of an evolutionary life cycle there will be topics like Release 1, Release 2 etc., followed by headlines like requirement specification on the second level.
Another alternative is to use top level headlines corresponding to artifacts created by modeling activities, such as 'create communication structure' or 'describe subject behavior'. 
\\
The WBS is created during the planning phase of a project life cycle. During this phase the project manager works with the project team to make sure that the client's needs are addressed and the project is planned completely and approved by the client prior to any sort of production beginning on the project.

\subsubsection{Organisational break down structure and software architecture}
An organizational breakdown structure (OBS) complements the WBS and resource breakdown structure of a project. Project organizations can be broken down in much the same way as the work or product. The OBS is created to reflect the strategy for managing the various aspects of the project and shows the hierarchical breakdown of the management structure. Hence, the work break down structure has a significant impact on the organizational structure of the project team. The same holds for the phases of the software life cycle and the system architecture influencing the work break down structure. Conway’s law states “organizations which design systems ... are constrained to produce designs which are copies of the communication structures of these organizations” \cite{article:ConwaysLaw}. A variation of Conway’s law can be found in [12]. "If the parts of an organization (e.g., teams, departments, or subdivisions) do not closely reflect the essential parts of the product, or if the relationship between organizations do not reflect the relationships between product parts, then the project will be in trouble... Therefore: Make sure the organization is compatible with the product architecture” \cite{book:OrgPatternsAgile}.
As we look at developing federations of systems with a federation of independent project teams, the system architecture needs to be aligned with the multiple project team structure.
\\

\subsection{Software Development Methodology For Federated Systems}
The software development methodology for federated systems proposed here is based on Subject-oriented Business Process Management (S-BPM)  


\subsubsection{Development as a multiple-team structure}
We now assume that the dog food order-to-cash scenario does not yet exist. The store wants to extend its services for the customers by offering online shopping and home delivery. In order to reach this business objective it takes the initiative to found a federation of enterprises which combine their services and develop a corresponding federation of systems.
Each federated enterprise establishes a project team, working on their parts of the solution independent from each other. This leads to a multiple-team project on the federation level \cite{book:OrgPatternsAgile}. As the teams belong to different, independent companies they all have their own development culture and methodology.
Since there is no single line management who can assign an overall project manager, the federation members need to agree on a project leader and the competencies related to this role. As the initiator of a federation has the most interest in the development of the federated solution it might be helpful that this company, in our case the store, recruits the leader.

His or her major task is to ensure smooth communication between the independent teams, respectively their managers. The project teams needs to coordinate how the systems they are developing communicate with each other. Their major communication paths are predefined by the communication structure of the system federation. This strategy leads to a high socio-technical-congruence. Figure \ref{Multiple-team} (CS: no missing) shows the team and communication structure of the dog food order-to-cash federation.

\begin{figure}[htbp]
	\centering
	\includegraphics[width=0.6\linewidth] {Figures/Chapter5/Project/MultipleTeam.jpg}
	\caption[Multiple-team project and its communication structure]{Multiple-team project and its communication structure}
	\label{fig:Multiple-team}
\end{figure}

Beside that top-level communication implied by the problem structure, each team can use services offered by other enterprises. Figure 6 reveals that the shipment company uses the service of carriers and forwarding agents, in order to implement the transportation service offered to the dog food shop. This communication relation is of no interest for other federation members and thus should not be visible to the top level teams. It belongs to the internal issues of the shipment project team.
\\
\subsubsection{Development process for federated systems}
 The artifacts to be created according to subject orientation need to be developed by a federation of teams related to the subject interaction structure.
 
\paragraph{Specification of the communication structure}\
The communication between the various members of the federation needs to be specified in more detail. This is done by assigning a subject to each member of the federation and defining the messages exchanged between the subjects. Together with the data transported by the messages a communication model of the system federation is defined. The advantage of the subject-oriented approach is that the system communication structure is directly in line with the communication structure of the corresponding developing teams. The result of that step is the subject interaction diagram (SID). 

\paragraph{Specification of the subject behaviour}\
After defining the communication structure the behavior of each subject is specified. The modelers describe the allowed sequence of messages exchanged on top level and the internal functions of the individual systems. These internal functions represent the services executed by the corresponding federation partner either directly or supported by other service providers. They also encapsulate the communication with those sub-contractors as it is of no interest on the top level of the federation.\\
The behavior of a subject is mainly defined by the corresponding project team, however, in close coordination with the teams responsible for the partner subjects. The teams only need to make sure a message sent to a partner has a receive state in the corresponding subject behavior and vice versa. This pairwise coupling means, e.g., that the behavior description of the shipment company has to contain a state for receiving the “Transfer order” message, transmitted by the related send state in the behavior diagram of the dog food store subject. In order to correctly model these interactions the responsible project teams need also to agree on the interaction sequence of the subjects. However, their internal task behavior (i.e. sequence of functions for task accomplishment) might not become visible to others, as is specified decentralized and might not be shared at all. 

\paragraph{Implementation of the input pool}\
The input pool is the abstract concept for defining the semantics of message exchange. Partners exchanging messages need to agree on how they implement the input pool semantics. Sending requires the sending subject to execute a function to deposit a message in the input pool of the receiver. For each subject doing so an implementation agreement is necessary. Since an input pool is owned by exactly one subject, the functionality for accessing it is local and does not need to be coordinated with the partners. In most cases input pools are implemented as web services.

\paragraph{Implementation of subject behaviour}\
Each team has to implement the behavior of its subject. This means they have to ensure that depositing and removing messages (including business objects) in or from the input pool are executed and internal functions are invoked in the specified sequence. Workflow engines are appropriate tools for implementing that functionality.\

\paragraph{Implementation of internal functions}\
The internal functions realize the kernel of the service contributed by a partner to a federated application. Messages are the means to cause the invocation of an internal function, and they transport its result to a partner subject. Internal functions can be based on existing systems, e.g., an SAP client.  They also can be implemented using another federated solution, or being developed from scratch. The way an internal function is realized is a local decision taken by the corresponding project team.

\paragraph{Operation of a federated system}\
Beside the development and deployment the non-functional aspects of a federated system need to be agreed upon by the contributing partners. For this purpose they negotiate service level agreements (SLA) defining response time, down time, reaction time in error cases etc. The SLA also includes business aspects like costs and regulations for exceptional situations like a member leaving the federation and bringing in another one.

\subsubsection{\textbf{Federated work break down structure}}
The various activities described so far can be organized in a federated work break-down structure as shown in figure \ref{fig:WBSDog}.

\begin{figure}[htbp]
	\centering
	\includegraphics[width=0.6\linewidth] {Figures/Chapter5/Project/WBSDog.jpg}
	\caption[Work break down structure for the development of a federated system]{Work break down structure for the development of a federated system}
	\label{fig:WBSDog}
\end{figure}

The tasks can be divided into three types:
\textbf{Joint work} concerns the top level of the federation and therefore is done collaboratively by all members of a federation. The major issue on this level is to agree on communication structure and behavior of the entire system, while the behavior of each subject can be described individually by the corresponding member of the federation.\\
\textbf{Some work can be done bilateral}. Communicating partners, e.g., agree on the coding of the business objects and the implementation of the input pool. They also define the service level agreements.\\
\textbf{Local work} comprises activities of the development teams which do need to be coordinated with teams of other federation members. A major example in this context is the set of internal functions of each subject, being a local matter, and developed following the particular culture and methodology of the respective team.
\\


\subsection{Conclusion}
We have presented an approach for developing federated systems. The concept considers the characteristics of virtual enterprises combining the services of the partners to satisfy customer needs while keeping legal, organizational, technological and cultural independence.
Our communication-oriented view follows the idea that the decentralized structure of federated systems needs to be reflected in the organizational structure of multiple project teams for developing such systems. Those teams belong to separate enterprises and are mutually independent with respect to methodology, technology etc. they use to develop their individual part of the federated system. 
The proposed approach establishes a layer above the enterprise-specific environments. It helps building coherence on the top level of the federated system solution, while the teams, system elements etc. on the individual level of each federation member keep the highest degree of independence.

\subsection{Future Work}
It has to be investigated whether the OWL definition and/or the execution semantics has to be adapted for a better project management support. Based on that results some guidelines for a subject oriented project management has to be developed and enhanced based on practical experiences.


\section{Subject-Oriented Fog Computing}

Many scenarios related to digitalization increasingly (i) require an easy-to-customize development environment, (ii) capture on-the-edge systems or devices under the control of users or responsible stakeholders. Typical examples are home support systems in healthcare, maker environments producing local goods, and intelligent transport control systems for smart regions. Developing such applications requires architectures that allow to network or compose systems in a modular, while effective and efficient way \cite{article:SurveyCompConcepts}. During the last years, with the advent of advanced equipment and technologies, such as production devices for the private consumer market, networked applications have become common. As a consequence of this trend, a significant issue also appears, namely the increases in the demand of both communication and execution capability. New applications, such as home care support systems, all deal with complex interaction operations, which should be understood by users, and thus require a high level of abstraction \cite{article:FogHealthcare},\cite{article:MobilecloudComp}. 
\\
Such demands pose significant challenges to existing development paradigms, particularly in terms of edge computing and stakeholder-oriented communication capacities (cf. \cite{article:SurveyCompConcepts},\cite{article:FogHealthcare}]). Using behavior abstractions aligning stakeholder needs with communication and processing capabilities in this context is an appealing idea. For instance, in-situ care support devices can be utilized to handle the tasks of preparing the pharmacy order or they can be employed to collaborate with each other to transmitting maintenance messages and sharing resources \cite{article:FogHealthcare}. Besides network technologies, mobile cloud computing is a typical enabler for this demand \cite{article:MobilecloudComp}.\\ 
However, according to Syed et al. \cite{article:FogPattern} purely cloud-based systems typically require low latency, support for heterogeneity, mobility, geographical distribution, location awareness, etc. Consequently, Fog Computing (FC) as a near-the-edge-computing paradigm has been defined as a collection of various small distributed clouds deployed closer to the systems or devices at the edge of a communication network (ibid.). Fog applications can be structured along several dimensions, either directly or indirectly referring to stakeholder interaction \cite{article:OoTAnalytics}:
\begin{list}{-}{spacing}
	\item Geo-distribution: wide (across region) and dense - high population of events, such as ramp accesses in traffic, sensor systems in production halls, clustering medical devices in home healthcare application development
	\item Low/predictable latency: tight within the scope of a certain location - intersection, production isle, treatment room
	\item Fog-cloud interplay: data at different time scales - sensors at intersection/traffic info at diverse collection points, supply chain monitoring/production control in process industry, monitoring body condition/treatment planning procedure in healthcare
	\item Multi-agencies orchestration: Agencies that run the system coordinate policy implementation at the same time, e.g., traffic authority runs light system while controlling law policies in real time; active elements for production control implement also governance regulations; home healthcare support is effective with respect to medical treatment and personal well-being.
	\item Consistency: adjusting demands and capabilities, such as getting the traffic landscape demands a degree of consistency between collection points, aligning engineering with production processes, or ensuring well-being while adapting medication to patient needs.
\end{list}

In this contribution, we present Subject-oriented Fog Computing (SFC), a choreographic approach and multi-layered infrastructure for Fog Computing. Separating modeling from organizational and technical implementation along a staged procedure it aims for supporting system architects, designers, and developers, who are interested in stakeholder interactions when building Fog Computing solutions. We propose a development and software architecture scheme without platform dependencies, open for various networked settings. It is based on behavior abstractions termed subjects that integrate a socio-technical design perspective, and allows composing applications from a stakeholder perspective (cf. [6-8]).
In the following section we review related research to developing fog applications according to stakeholder needs in various domains. Subsequently, we introduce SFC based on a System-of-Systems perspective, and provides an exemplary case from developing home healthcare support systems. Finally, we conclude summarizing SFC and indicating further standardization activities. 

\subsection{Fog Computing and Subjects}

We introduce Fog actors by starting with the encoded System-of-System perspective, sketching the federated nature of choreographic ecosystems (subsection above). We then provide the basic modeling notation and exemplify Fog actors as subjects for a home healthcare scenario. Finally, the corresponding Fog runtime system is sketched in terms of its application along the organizational and technical development phases. 

\subsubsection{Federated Systems}
When considering Fog Computing as an addition to cloud ecosystems we expand software architectures to include systems outside the software system which interact with the software system \cite{article:FogPattern}. Each component of the ecosystem can be represented as a system using behavior models. Thereby, cloud ecosystems can serve as service providers for the nodes of the network (of applications). The Fog network enriches the cloud ecosystem, e.g., for specific purpose like home healthcare with domain-specific models.\

Since these enrichments are compound systems, a System-of-Systems (SoS) perspective helps conceptualizing the construction and development of Fog applications \cite{article:SyS}. SoS have as essential properties 'autonomy, coherence, permanence, and organization' (ibid, p.1) and are constituted 'by many components interacting in a network structure', with most often physically and functionally heterogeneous components. For instance, home healthcare applications comprise support systems for dementia, blood pressure measurement, and pharmacy shopping, and need to be adaptable on-the-fly in case of changing operational conditions (cf. \cite{article:DesignHealth}).\

Since users tend to develop applications incrementally, their specifications are adapted to changes dynamically. Once these specifications in terms of SoS models become executable, users can interactively bootstrap their modifications. Behavior can be deployed, once being specified and validated. Utilizing subject-oriented modeling and execution capabilities (cf. \cite{Flei12}), systems or subjects are viewed as emerging from both the interaction between subjects and their specific behaviors encapsulated within the individual subjects. Like in reality, subjects as systems can operate in parallel and exchange messages asynchronously or synchronously.

\subsubsection{Subject-oriented Representation}
According to the SoS perspective, Fog applications operate as autonomous, concurrent behaviors of distributed Fog actors. A Fog actor or subject is a behavioral role assumed by some entity that is capable of performing actions. The entity can be a human, a piece of software, a machine (e.g., a robot), a device (e.g., a sensor), or a combination of these, such as intelligent sensor systems.
\\
When subject-oriented concepts and development techniques are applied, SoS subjects can execute local actions that do not involve interacting with other subjects (e.g., calculating a threshold value for medical intervention and storing a pharmacy address), and communicative actions that are concerned with exchanging messages between subjects, i.e. sending and receiving messages. Subjects are one of five core symbols used in specifying designs. Based on these symbols, two types of diagrams can be produced to conjointly represent a system: Subject Interaction Diagrams (SIDs) and Subject Behavior Diagrams (SBDs).
\\
SIDs provide an integrated view of a Fog SoS, comprising the subjects involved and the messages they exchange. The SID of a home healthcare support process is shown in Figure \real{fig:homeCare}. The aim of such systems is not only to support patients when needing healthcare at home, but also to profit from networked services, in particular, getting drugs in time from pharmacy, receiving in-situ service when required, and intelligent networking of local devices, while being scheduled for managing everyday life and being reminded of individual caretaking activities (cf. \cite{article:DesignHealth}).  
\\
Home healthcare comprises several subjects involved in near-edge communication: A Personal Scheduler coordinating all activities wherever a patient is located (traditionally available on a mobile device), a Medication Handler taking care of providing the correct medication at any time and location, Blood Pressure Measurement sensing the medical condition of the patient, and Shopping Collector as container for all items to be provided for home health care. In the figure the messages to be exchanged between the subjects are represented along the links between the subjects (rectangles).
\\
In-situ, and thus near-edge communication is required for delivering Blood Pressure Measurement data to the Personal Scheduler and the Medication Handler, as the patient handles the measurement device at home and needs to know, when to activate it and whether further measurements need to be taken. Another need for near-edge communication is given through the Shopping Collector: It receives requests from both, the Medication Handler when drugs are required from the pharmacy, physician, or hospital, and the Personal Scheduler, in case further shopping for the patient is required. As such, the Shopping Collector serves as an interface subject for shopping services to the homecare environment.

\begin{figure}[htbp]
	\centering
	\includegraphics[width=0.6\linewidth] {Figures/Chapter5/Fog/homeCare.jpg}
	\caption[Example of home care support (SID)]{Example of home care support (SID)}
	\label{fig:homeCare}
\end{figure}


As usual Subject Behavior Diagrams (SBDs) provide a local view of the process from the perspective of individual subjects. 
\\
Given these capabilities, SoS Fog designs are characterized by (i) simple communication protocols (using SIDs for a process overview) and thus, (ii) standardized behavior structures (enabled by send-receive pairs between SBDs), which (iii) scale in terms of complexity and scope. 
\\
Subject-oriented Fog Computing (SFC) allows meeting ad-hoc and domain-specific requirements. As validated behavior specifications can be executed without further model transformation, stakeholders can guide the implementation of specification, representing domain-specific task flows, and make ad-hoc changes by replacing individual subject behavior specifications during runtime. Due to the distributed nature and loose coupling of subject-oriented representations, the ultimate stage of scalability could be reached through dynamic and situation-sensitive formation of edge systems.
\\
SFC structures SoS, e.g., when federating a blood pressure measurement device with a personal health scheduling systems, according to their communicating with each other. When these devices need to communicate directly with the cloud, e.g., as required in case of maintenance, or calling a specialist for medication, this link is encoded in the diagrams and executed during runtime after technical implementation. On the modeling layer the activity is a request sent to another subject, waiting until an answer is received, and processing the received answer. 

\subsubsection{Execution}
Once a Subject Behavior Diagram, e.g., for the Blood Pressure Measurement subject is instantiated, it has to be decided (i) whether a human or a digital device (organizational implementation) and (ii) which actual device is assigned to the subject, acting as technical subject carrier (technical implementation) (cf. \cite{Flei12}). Typical subjects as edge devices are smart devices, which can have Internet connectivity, including smart phones, tablets, laptops, healthcare devices, etc. The subject-oriented runtime engine \cite{article:StakeHolderCentered} is then a Fog Computing infrastructure providing low-latency virtualized services and is linked with the Cloud Computing infrastructure by the same subject interaction mechanism. As there can be a variety of edge devices, such a Fog Computing platform also needs to manage and control these devices (see also foglets described below). 
\\
Size, storage capacity, processing capabilities, and latency increase as we move closer to cloud computing. The subject-oriented Fog acts as an intermediate layer between the edge devices and the cloud. Edge devices request computing, storage and communication services from the Fog according to the subject-oriented communication scheme. The Fog provides local, low latency response to these requests and forwards relevant data for computationally intensive processing, long-term analytics and persistent storage over to the cloud. Figure[Fog Computing Architecture]{Fog Computing Architecture} provides a schematic visualization of this constellation, as it can be used for implementing the sample home healthcare support system.

\begin{figure}[htbp]
	\centering
	\includegraphics[width=0.6\linewidth] {Figures/Chapter5/Fog/FogArch.jpg}
	\caption[Fog Computing Architecture]{Fog Computing Architecture}
	\label{fig:FogArch}
\end{figure}

With respect to the home-healthcare example, a typical infrastructure comprises local devices and their interconnected services, such as linking the Blood Pressure Measurement to the Personal Scheduler. These subjects can be either linked to an IoT SoS, e.g., coupling several sensor systems, or to Cloud services, as for accessing public databases when checking reference or availability data, depending on the state of affairs in the home healthcare setting.
\\
Fog nodes are subject carriers representing resources including hardware (computing, networking and storage) capabilities. They provide ‘local’ real-time data processing capabilities, and, despite multi-tenancy, can execute applications in isolation to prevent unwanted interference from other processes. Policies to control service orchestration, filtering, and for adding security can be implemented dedicating a specific control subject, since the primary scheme of control is choreography.
\\
The approach scales, due to the decentralized management mechanisms allowing to setup, and configure a large number of devices in the Fog. In this context, subjects correspond to foglets (cf. \cite{article:OoTAnalytics}), i.e. software agents for each fog node, monitoring the state of the node and services. A subject can use abstraction tier APIs to monitor the state associated with (physical) devices and services deployed on this device. It analyses the entire information (encoded in an SBD), and delivers it to receivers linked through messages for further processing. These subjects can also perform lifecycle activities. As demanded by Vaquero et al. \cite{article:FiningwayFog}, SFC comprising a fog abstraction layer provides uniform programmable interfaces for resource control and management. 
\\
According to the S-BPM concepts, normalization can be used to abstract essential behavior patterns. For instance, in case Blood Pressure Management requires a machine-dependent procedure, its action behavior (performing functions) as a subject can in principle contain many internal functions which are performed in sequence, in order to accomplish an assigned task. In these sequences of internal functions, no sending and receiving nodes are included. Accordingly, extensive and therefore confusing behavior diagrams can be avoided. Since these sequences of internal functions are not important for communication, model representations can be simplified, and normalized behavior can lead to larger functions by hiding functional details. Actually, for the sake of understanding the home healthcare setting, the subjects shown in Figure \ref{fig:homeCare} have been normalized.
\\
In case the communication patterns are generalized, the process-network feature of S-BPM facilitates representation. For instance, when the Shopping Collector needs to collect sensor data from various storage devices, such as a refrigerator or a food isle, its communication requests and the respective replies can be denoted in a summative way. In SFC this feature helps representing mutually dependent processes, i.e., when subjects of a near-edge process communicate with subjects of other (near-edge) processes. As shown in Figure 5 the Home care near-edge process interacts with the Goods delivery process through the Personal Scheduler. In this case, the interaction is not further detailed, rather indicated through directed links. The same holds for the interaction between the Shopping Collector and the Medication Handler, which helps ensuring the quality of drug support in the Medicare process.

\begin{figure}[htbp]
	\centering
	\includegraphics[width=0.6\linewidth] {Figures/Chapter5/Fog/InteractionHomeCare.jpg}
	\caption[Extended subject interaction diagram for the process ‘home care’]{Extended subject interaction diagram for the process 'home care'}
	\label{fig:InteractionHomeCare}
\end{figure}


For SFC implementation the open source engine UeberFlow \cite{DynamicPerspective} can be used. Hereby, SFC actions or tasks are ordered in the sequence as defined through SBDs and SIDs. The Workflow Specification of UeberFlow represents an entirely executable model of an application, given the subject actions and communication with others. It acts as container for so-called WorkflowUnits that are created for each subject, and captures all activities (WorkflowSteps). In addition, WorkflowUnit manages the data processed by the WorkflowSteps and its WorkflowFunctions. Consequently, Fog applications are executed through WorkflowSteps. 
\\
Thereby, the WorkflowFunctions are the most fine-grained units of execution in the UeberFlow Language meta-model, and define the actual execution logic of a WorkflowStep, its prerequisites and results. Once a step is triggered, a specific sequence of WorkflowFunctions is executed. The WorkflowFunctions can be one of 6 different types. For each of them an Actor has been implemented utilizing the Akka framework (http://akka.io/). Hence, an instance in UeberFlow is equivalent to all actor instances created in the context of this particular workflow instance. All of those actor instances are aggregated using the actor structuring and supervision mechanisms by defining a root actor representing the entire instance.

\subsection{Conclusion}
Fog Computing (FC) as a near-the-edge-computing paradigm has the potential to improve user support. When defined as a collection of various small distributed clouds deployed closer to the systems or devices at the edge of a communication network subject-oriented applications support 
\begin{list}{-}{spacing}
\item wide and dense geo-distribution due to their behavior abstraction, as e.g., required for home healthcare support systems, linking not only (medical) devices at home, but also medical infrastructure (physician, pharmacy, nursing services etc.) from the region
\item low or predictable latency due to the runtime concept of parallel processing 
\item cloud interplay of Fog nodes, due to separating specification from technical implementation which allows for processing data at different time scales, e.g., when monitoring body condition and supporting a patient treatment planning procedure 
\item multi-agencies choreography, loosening the need for orchestration, due to the inherent concept of choreography in subject-oriented architecting. Hence, Fog actors or subjects only need to be synchronized as tight as required, e.g., when a running monitor subject requires coordination with healthcare policy implementation at the same time
\item consistency, due to mapping all respective requirements to corresponding interaction patterns. Hence, demands and capabilities can be adjusted specifying message exchange patterns, in order to ensure overall consistent system states, either through subjects working in parallel, or through information distribution triggering further subject behavior.
\end{list}
Our future standardization effort will focus on including for networking information into the subject-oriented behavior abstractions, to enable modeling stakholder-specific settings according to their case-specific needs and available Fog actors. Once stakeholders are able to edit and validate the subject behavior models, they also can deal with organizational and technical implementation details, allowing them to adapt an entire application as System-of-System dynamically. Adaptation to new policies can be implemented in this way (cf. \cite{article:SecurityMgmt}), leading to more situation-sensitive Fog applications (cf. \cite{article:FogSimToolKit}). 



\section{Activity Based Costing}

CS: table refs are incorrect

\subsection{Basic Concepts}
\subsubsection{Process Controlling}
Process controlling has both a strategic and an operational dimension [cf. e.g. \cite{book:ProzesseSchmelzer}, p. 229 ff.]. We concentrate on methods and techniques for planning, designing and coordinating the supply of information necessary to allow continuous operational process controlling with key figures as indicated in the closed-loop approach to performance management (see lower part of figure \ref{fig:Approach-Performance} in \ref{sec:BAMinSubjectOrientation}). As operational process controlling aims for post-execution analysis of business process instances it can be complemented by Business Activity Monitoring (BAM) which, based on event processing concepts, observes instances during execution and sets alerts or triggers actions in real-time or near real-time according to the particular situations identified (cf. \cite{book:processmonitoring}, \cite{article:BlueprintEventDriven}).
\\
The major question is how to measure process performance. A typical parameter for the evaluation of process effectivity is customer satisfaction while process time, quality and cost and adherence to schedules are suitable to assess efficiency (cf. e.g. \cite{book:ProzesseSchmelzer} p. 229 ff.). As these parameters have high significance for the competitive position they are crucial for process controlling. While the assessment of customer satisfaction and maturity levels of processes usually are matters of periodic monitoring activities there might be other, more technical parameters needed to be watched permanently, like the response time of application systems. Those aspects are especially relevant if processes are extensively supported by IT.
\\
Figure \ref{fig:OpProcessCont} gives a conceptual overview of a key figure-based operational process controlling, split into continuous and periodic or occasional controlling activities, like it could be set up successively for the S-BPM approach.
The integrated collection and analysis of common managerial data allows for a cohesive evaluation and control in terms of process controlling [cf. \cite{book:ProzesseSchmelzer} p. 248 ff., 1 p. 385 ff., 7 p. 158 ff.]. It feeds back results to take decisions and actions, fostering a steadily growth of experience regarding the interdependencies between cost, quality and time (organizational learning).

\begin{figure}[htbp]
	\centering
	\includegraphics[width=0.6\linewidth] {Figures/Chapter5/ActivityBased/OpProcessCont.jpg}
	\caption[Operational Process Controlling]{Operational Process Controlling}
	\label{fig:OpProcessCont}
\end{figure}

In this section we focus on cost figures. Calculating them is more difficult than assessing those for time and quality. S-BPM allows determining cost figures for processes and process steps as well as for the occupation of cost centers and organizational units with little additional effort though. The reason is that S-BPM specifies subjects as actors in a process, their interaction and their assignment to elements of the organizational structure (organizational units, positions, roles). 
\\
The basic methodology to integrate such cost information into process controlling is Activity-Based Costing (ABC).


\subsubsection{Methodology of Activity-Based Costing}
The concept of Activity-Based Costing origins in the work of Miller and Vollmann \cite{article:HiddenFactory} and Cooper and Kaplan \cite{article:MeasureCost} and was established in the German-speaking community by Horvath und Mayer \cite{article:Prozesskosten}.
\\
ABC roots in a simple fact: producing and delivering a product or service involves many activities within cost centers and across boundaries of cost centers or functional areas, all causing costs. Major factors influencing these costs (cost drivers) usually are measures of the activity quantity, e.g. the number of purchasing orders being processed in procurement.
\\
\newline
\textbf{Step 1: Analysing activities}
\\
Starting point for ABC is an analysis of activities performed in the cost centers, using common methods like interviews, questionnaires, self-monitoring, third-party observation, document analysis or multi-moment recording. This analysis is essential for bordering cost center internal process steps and main processes running across cost center boundaries. Self-monitoring and multi-moment recording can bring up time standards for the execution of processes and their steps, but needs high effort. In order to ease the investigation of times, controllers instead often conduct interviews to find out what share of work force capacity the process steps occupy in a cost center.
The analysis results in a transparent, hierarchical process structure showing the assignment of activities to process steps, the assignment of process steps to cost centers and the aggregation of process steps to main processes (cf. figure \ref{fig:ProcessStruct} )

\begin{figure}[htbp]
	\centering
	\includegraphics[width=0.6\linewidth] {Figures/Chapter5/ActivityBased/ProcessStruct.jpg}
	\caption[Process Structure]{Process Structure}
	\label{fig:ProcessStruct}
\end{figure}


This first step of Activity-Based Costing can be based on the results of the activity bundles analysis and modeling in S-BPM [6]. ABC-relevant information on subjects, their activities and the business objects being worked on are contained in the S-BPM process model (see section 2.3).
\\
\newline
\textbf{Step 2: Determining cost drivers}
\\
Horvath und Mayer differentiate between activity quantity induced (aqi) and activity quantity neutral (aqi) processes [9]. The latter (e.g. leading a department) cause costs independent from the activity quantity (e.g. the salary of the department leader). In activity quantity induced processes (e.g. purchasing goods) resource consumption and related costs vary with the activity quantity. Hereby process costs incurring depend on the number of cost drivers and there is need to determine measures for those as a second step in ABC. Cost drivers serve as an allocation base for resource utilization and thus also for causing costs.
\\
Cost drivers need to meet some requirements in order to make cost dependence transparent:
\begin{list}{-}{spacing}
	\item Process costs should, at least in the long term, vary with the activity quantity
	\item Cost driver values should be easy to assess and to understand.
	\item Input of resources should be approximately the same for all process instances, otherwise processes and cost drivers need to be further differentiated.
\end{list}

Usually ideas of what the major cost driver is already come up during the activity analysis and the activity quantity can be determined simultaneously then.
\\
\newline
\textbf{Step 3: Determining process costs}
\\
In practice it is often difficult to only assign the major cost driver to the main processes, because a main process can consist of process steps with different cost drivers not being proportionally related.
\\
In a given organizational and cost accounting context resources and costs are planned and actual costs are recorded on cost center level. This means planning and assessing process costs initially is also related to cost centers.
\\
Although theory suggests to plan process costs analytically and by cost type like in direct costing, practitioners prefer more simple concepts. One alternative is to only plan labor costs analytically and to allocate all other cost types proportionally. Another option would be to assign to the analyzed processes the capacities they consume and the related costs. In any case the accountants usually assume the labor cost to be the major cost element. Process costs then can be computed by multiplying a qualified estimate of the number of employees involved in the process by their average wage. If need be activity quantity neutral costs also can be passed on proportionally. In case there are more activity quantity induced costs with a significant extent they need to be considered in addition to labor costs. Even then the described procedure is still easy to handle.
\\
\newline
\textbf{Step 4: Determining process cost rate}
\\
In a last step, for the purpose of job order costing or product calculation, a simple division results in a process cost rate similar to the computation of a machine hour rate.
As shown Activity-Based Costing can be implemented in various ways. The identified problem of different cost drivers that cannot be aggregated can be solved by using time-related allocation bases \cite{article:Rechenzwecke} p. 23.
\\
The concrete process times can be computed if a workflow engine writes time stamps for begin and end events of the process steps. In order to have the engine processing a workflow at runtime, process activities must be assigned to concrete actors. Both kinds of information are needed for establishing ABC as elaborated in section \ref{section:ExampleABC}.


\subsection{BPM as Data Supplier}
Subject-oriented Business Process Management (S-BPM) focuses on the acting elements (actors) and their interactions as they drive a process. Its modeling notation includes all building blocks of a complete sentence in natural language as there are subject, predicate and object. The clear formal semantic of the underlying process algebra makes it possible to automatically generate code and makes subject-oriented process descriptions executable at a finger tip \cite{Flei12}, \cite{article:HMD-S-BPM}.
\\
Major parts of the model are subject interaction diagrams, describing the subjects involved in the process and the messages they exchange, and subject behavior diagrams, specifying subject activities as there are sending and receiving messages and other functions (e.g. manipulating business objects). The ladder means that at a time subjects can either be in a send, receive or functional state. 
Transforming the model into a workflow and integrating IT solutions (e.g. ERP functionality) to support particular activities is subject to the embedding of the process into IT. Assigning subjects to elements of the existing organizational structure (organizational units, positions, roles) being responsible for carrying out the activities as defined in the model, is called embedding the process (model) into the organization. Existing directory services based on Lightweight Directory Access Protocol (e.g. Active Directory) can ease the assignment of subjects to roles, groups and people as implemented in Metasonic Suite.
A process engine like Metasonic Flow interprets the model at runtime, instantiates process instances and controls their execution. According to the defined behavior the engine involves users and IT services or applications as subject representatives. It also controls the handling of business objects included the in subject behavior (creation, modification, deletion, exchange through messages). During execution the engine can capture many single pieces of data relevant for process controlling, especially by setting time stamps for state transitions and by counting instances. Examples are 

\begin{list}{-}{spacing}
\item begin time and end time of every single instance,
\item begin time and end time of the single steps within an instance or
\item number of instances of a certain process per time unit.
\end{list}
Using such raw data suitable software can compute key figures like
\begin{list}{-}{spacing}
\item waiting time of an instance from the moment it appeared in the in-box of an actor until he or she takes it out for processing (per case, on average) and
\item processing time from taking the instance out of the in-box until putting the result into the out-box (per case, on average).
\end{list}

This means the workflow system generates a valuable data basis for a meaningful Activity-Based Costing. This data needs to be categorized though and the key figures need to be defined precisely and unique in order to derive useful management information (see example in section \ref{section:ExampleABC}).

\subsubsection{Example for Estimating Process Costs in S-BPM} \label{section:ExampleABC}
Effective process controlling, allowing to turn such decisions into the right actions additionally requires information about cost dimensions in processes and cost-related consequences of processes for cost centers. Cost information enables monetary valuation of the enterprise performance as well as identifying weak points in operations and valuating their economic impact.
\\
We exemplify the determination of process costs using an order process. Figure \ref{fig:CalcPro} depicts the behavior diagram for the subject 'purchaser' enriched with some time information. These are times recorded as time stamps for state transitions by the process engine and stored in its event log \cite{article:SubProcessMon}. For clarity reasons in the figure \ref{fig:CalcPro} we only added time stamps for one state and just added the duration for the others.

\begin{figure}[htbp]
	\centering
	\includegraphics[width=0.6\linewidth] {Figures/Chapter5/ActivityBased/CalcPro.jpg}
	\caption[Calculating Processing Time of the Subject 'Purchaser']{Calculating Processing Time of the Subject 'Purchaser'}
	\label{fig:CalcPro}
\end{figure}


In the example the processing time of the subject 'purchaser' in case of the sunshine path (goods are on stock) can be computed by adding up the differences between start and end time of the functional states 'create order request' and 'check delivery'. The sunshine path sequence is as follows: After creating the request the purchaser sends it to the approver and then waits for an answer. If the request is denied the instance ends (right column in figure \ref{fig:CalcPro}). Otherwise the purchaser forwards the request to the warehouse (left column), waits for them to announce the delivery date and checks it (second to left). Finally he waits for the delivery and checks it after reception, before the instance comes to an end (third to left). As a simplification we assume the subject representatives to work permanently when performing functions (being in a functional state). We did not insert times for send and receive states because message exchange is considered to be accomplished electronically with no latency for sending, transmission and receiving. Applying this procedure to all subjects could lead to the result in table \ref{tab:procTime}.

\begin{table}[htbp]
	\centering
\begin{tabular}{|p{5.0 cm } |c|}
\hline
	\textbf{Subject} & Processing Time \\
	\hline
	\hline
	Purchaser & 35 min\\
	\hline
	Approver & 10 min \\
	\hline
	Warehouse & 30 min \\
	\hline
	Invoice Verification & 5 min \\
	\hline
	Accounting/book keeping & 5 min \\
	\hline
	Total & 85 min \\
	\hline
\end{tabular}
\caption{Processing times}
\label{tab:procTime}
\end{table}

For estimating the costs of the process we need an hourly or minute-related rate of wage for the people being assigned to the subjects. It is also possible to provide those values aggregated on group or role level. In Figure 5 we visualized how the subjects in our example are mapped to persons, while table \ref{tab:Hourly-wages-rate} gives an overview of the rates for the employees involved as subject representatives.

\begin{figure}[htbp]
	\centering
	\includegraphics[width=0.6\linewidth] {Figures/Chapter5/ActivityBased/embedding.jpg}
	\caption[Embedding the Subjects into the Organization]{Embedding the Subjects into the Organization}
	\label{fig:Embedding}
\end{figure}

\begin{table}[htbp]
	\centering
	\begin{tabular}{|p{10.0 cm } |c|}
		\hline
		\textbf{Employee} & \textbf{Wage rate} \\
		\hline
		\hline
		Miller, Laurel, Schulz & 200 Euro/hour\\
		\hline
		Ployer, Schmidt, Doe, Keweling & 100 Euro/hour\\
		\hline
		Weber, Wagner, Kramer, Meier, Switch, Lampe, Smith, Stuart, Buchner, Richards, Thorwald, Regal, Mustermann, Jones & 50 Euro/hour\\
		\hline
	\end{tabular}
\label{tab:HourlyWages}
\caption{Hourly wages rate}
\end{table}



Having these time and wage values available it is a simple multiplication to determine the personnel-related process costs for every single instance (cf. table \ref{tab:ProcessCosts}). Considering a sufficient number of instances over a representative period of time allows computing a valid average cost value.

\begin{table}[htbp]
	\centering
	\begin{tabular}{|p{5.0 cm } |c|r|}
		\hline
		\textbf{Subject} & \textbf{Employee} & \textbf{Costs}\\
		\hline
		\hline
		Purchaser & Kramer & 35 Min. x 50 Euro/60 min.  =  29,17 Euro\\
		\hline
		Approver & Ployer & 10 Min. x 100 €/60 min. =  16,67 Euro\\
		\hline
		Warehouse & Lampe & 30 Min. x 50 €/60 min.  =  25,00 Euro\\
		\hline
		Invoice verification & Regal & 5 Min. x 50 €/60 min.    =    4,17 Euro\\
		\hline
		Accounting/Book keeping & Regal & 5 Min. x 50 €/60 min.    =    4,17 Euro\\
		\hline
		Total & & 79,18 Euro\\
		\hline
	\end{tabular}
\label{tab:ProcessCosts}
\caption{Personal Related Process Costs}
\end{table}

Assigning employees as subject representatives embeds the subjects into the organizational structure, because people belong to organizational units. As cost centers usually also are assigned to organizational units it is now possible to determine the costs of a process incurring in a certain cost center. From the cost center perspective it is also possible to see how its total costs are distributed over the processes and process steps it is involved in. Table \ref{tab:CostFig} shows examples for cost figures which can be computed.

\begin{table}[htbp]
	\centering
	\begin{tabular}{|p{3.0 cm } |p{10.0 cm }|}
		\hline
		\textbf{Key figure (Euro)} & \textbf{Computation (e.g. for 10 work days)}\\
		\hline
		\hline
		Process costs per process step/process & Multiply all processing times by the appropriate wage rate and aggregate the products over all instances occurring during the observation period\\
		\hline
		Process costs per cost center & Multiply all processing times incurring in the cost center by the appropriate wage rate and aggregate the products over all instances occurring during the observation period\\
		\hline
	\end{tabular}
\label{tab:CostFig}
\caption{Cost Figures}
\end{table}

As mentioned before key figures need to be defined carefully and precisely. A useful instrument helping to assure this are structured fact sheets being filled in with all necessary information \cite{book:KennzahlenIT}. Table \ref{tab:FactSgeet} depicts such a fact sheet created for our purposes \cite{book:MonitoringSubjekt}. A more formal structure can be found in \cite{article:ProcessPerfInd}.



\begin{table}[htbp]
	\centering
	\begin{tabular}{|p{3.0 cm } |p{10.0 cm }|}
		\hline
		\textbf{Attribute} & \textbf{Content}\\
		\hline
		\hline
		& \textbf{Characteristics}\\
		\hline
		Description & Average costs of a process activity for a certain period\\
		\hline
		To-be value/unit & tbd specifically (Euro)\\
		\hline
		Tolerance range/unit & tbd specifically (\%)\\
		\hline
		Escalation rule & In case of violation alert the process owner and start escalation process (tbd specifically)\\
		\hline
		Responsibility & Process Owner (tbd specifically)\\
		\hline
		\hline
		& \textbf{Measuring and Computing}\\
		\hline
		Measurement & Read time stamps written by Metasonic Flow, compute processing time as difference between time stamps for beginning and end, multiply processing time by hourly wage rate, divide product by number of completed instance\\
		\hline
		Algorithms & ∑ Processing time * hourly wage rate
					∑ completed instances\\
		\hline
		Data sources(general) & Tables in the database of Metasonic Suite:\newline
		RT\_PROCDESC, RT\_PROCINST, REC\_PARADESC, REC\_RECTRANS, UM\_USER\\
		\hline
		Data sources(specific) & Processing time:\newline
		\hspace*{4mm} \textbf{SELECT} TIMESTAMP1 \newline
		\hspace*{10mm} (\textbf{SELECT} STARTTIME \newline
		\hspace*{10mm} \textbf{FROM} RT\_PROCINST \newline
		\hspace*{10mm} \textbf{WHERE} RT\_PROCDESC = \textit{process} \newline
		\hspace*{10mm} \textbf{AND} ID = \textit{instance} \newline
		\hspace*{4mm}\textbf{FROM} REC\_RECTRANS \newline
		\hspace*{4mm}\textbf{WHERE} RT\_STDESC = \textit{state} \newline
		\hspace*{4mm}\textbf{AND} RT\_PROCINST = \textit{instance} \newline
		Hourly wage rate: UM\_USER (manually enriched by hourly wage rates) \newline
		Completed instances: see separate fact sheet\\
		\hline
		Frequency & weekly\\
		\hline
		\hline
		& \textbf{Presentation} \\
		\hline
		Addressees & Process Owner, Middle Management, Accountants (tbd specifically)\\
		\hline
		Presentation & As-is value and to-be value in combination with a sparkline showing the historical development, deviation from to-be value in \%\\
		\hline
		Archiving & Stored in additional database table, linked with RT\_PROCDESC\\
		\hline
	\end{tabular}
\label{tab:FactSgeet}
\caption{Fact Sheet for the Key Figure 'Average costs of a process activity'}
\end{table}

\subsection{Conclusion}
With the example in chapter 3 we could show that it is relatively easy to integrate cost information into S-BPM. Focusing on personnel costs as suggested avoids the problematic proportioning of costs and therefore is particularly suited for people-intensive areas with a high degree of indirect costs as it is characteristic for services.


\subsection{ Future Work}
A more detailed investigation of how the implementation of Activity-Based Costing can benefit from a preceding S-BPM implementation seems to be promising. Exploiting the conceptual particularities coming with and the data collected by S-BPM seems to bear considerable potential of savings when introducing ABC. Processes are defined and modeled and as-is process quantities per period are available as well as the distribution of the overall capacity of the cost centers over the process steps. These parameters allow determining a standard time for processes which lays the ground for planning process costs.
Next steps could be to extend the example by determining and specifying more key figures and testing them with representative numbers of instances of different processes. Learning from this could help to further elaborate the ABC concept for S-BPM.\\
The OWL specification has to be extended with features which support Activity based costing. This includes the data which should be collected and the structure in which these data are stored.

\appendix

\chapter[Classes and Properties of the PASS Ontology]{Classes and Properties of the\\PASS Ontology}


\section{All Classes (95)}

%appendix in smaller font size
\footnotesize

\begin{itemize}
	\item SRN = Subclass Reference Number; Is used for marking the coresponding relations in the following figures. The number identifies the subclass relation to the next level of super class.
	\item PASSProcessModelElement
	\begin{itemize}
		\item BehaviorDescribingComponent; SRN: 001 \\  \textit{Group of PASS-Model components that describe aspects of the behavior of subjects}
		\begin{itemize}
			\item Action; SRN: 002 \\ \textit{An Action is a grouping concept that groups a state with all its outgoing valid transitions}
			\item DataMappingFunction ; SRN: 003 \\ \textit{Standard Format for DataMappingFunctions must be define: XML? OWL? JSON? 
				Definitions of the ability/need to write or read data to and from a subject's personal data storage.
				DataMappingFunctions are behavior describing components since they define what the subject is supposed to do (mapping and translating data)
				Mapping may be done during reception of message, where data is taken from the message/Business Object (BO) and mapped/put into the local data field.
				It may be done during sending of a message where data is taken from the local vault and put into a BO.
				Or it may occur during executing a do function, where it is used to define read(get) and write (set) functions for the local data.}
			\begin{itemize}
				\item DataMappingIncomingToLocal ; SRN: 004 \\ \textit{A DataMapping that specifies how data is mapped from an an external source (message, function call etc.) to a subject's private defined data space.}
				\item DataMappingLocalToOutgoing ; SRN: 005 \\ \textit{A DataMapping that specifies how data is mapped from a subject's private data space to an an external destination (message, function call etc.)}
			\end{itemize}
			\item FunctionSpecification ; SRN: 006 \\ \textit{A function specification for state denotes \\ 
				Concept: Definitions of calls of (mostly technical) functions (e.g. Web-service, Scripts, Database access,) that are not part of the process model.\\
				Function Specifications are more than "Data Properties"? --> - If special function types (e.g. Defaults) are supposed to be reused, having them as explicit entities is a the better OWL-modeling choice.}
			\begin{itemize}
				\item CommunicationAct ; SRN: 007 \\ \textit{A super class for specialized FunctionSpecification of communication acts (send and receive)}
				\begin{itemize}
					\item ReceiveFunction ; SRN: 008 \\ \textit{Specifications/descriptions for Receive-Functions describe in detail what the subject carrier is supposed to do in a state.\\
						DefaultFunctionReceive1\_EnvoironmentChoice : present the surrounding execution environment with the given exit choices/conditions currently available depending on the current state of the subjects in-box. Waiting and not executing the receive action is an option.\\
						DefaultFunctionReceive2\_AutoReceiveEarliest: automatically execute the according activity with the highest priority as soon as possible. In contrast to DefaultFunctionReceive1, it is not an option to prolong the reception and wait e.g. for another message.}
					\item SendFunction ; SRN: 009 \\ \textit{Comments have to be added}
				\end{itemize}
				\item DoFunction ; SRN: 010 \\ \textit{Specifications or descriptions for Do-Functions describe in detail what the subject carrier is supposed to do in an according state.
					The default DoFunction\\ 1: present the surrounding execution environment with the given exit choices/conditions and receive choice of one exit option --> define its Condition to be fulfilled in order to go to the next according state.
					The default DoFunction \\2: execute automatic rule evaluation (see DoTransitionCondition - ToDo)
					More specialized Do-Function Specifications may contain Data mappings denoting what of a subjects internal local Data can and should be:\\
					a) read: in order to simply see it or in order to send it of to an external function (e.g. a web service)\\
					b) write: in order to write incoming Data from e.g. a web Service or user input, to the local data fault}
			\end{itemize}
			\item ReceiveType ; SRN: 011 \\ \textit{Comments have to be added}
			\item SendType ; SRN: 012 \\ \textit{Comments have to be added}
			\item State ; SRN: 013 \\ \textit{A state in the behavior descriptions of a model}
			\begin{itemize}
				\item ChoiceSegment ; SRN: 014 \\ \textit{ChoiceSegments are groups of defined ChoiceSegementPaths. The paths may contain any amount of states. However, those states may not reach out of the bounds of the ChoiceSegmentPath.}
				\item ChoiceSegmentPath ; SRN: 015 \\ \textit{ChoiceSegments are groups of defined ChoiceSegementPaths. The paths may contain any amount of states. However, those states may not reach out of the bounds of the ChoiceSegmentPath.The path may contain any amount of states but may those states may not reach out of the bounds of the choice segment path. Similar to an initial state of a behavior a choice segment path must have one determined initial state. A transition within a choice segment path must not have a target state that is not inside the same choice segment path.}
				\begin{itemize}
					\item MandatoryToEndChoiceSegmentPath ; SRN: 016 \\ \textit{Comments have to be added}
					\item MandatoryToStartChoiceSegmentPath ; SRN: 017 \\ \textit{Comments have to be added}
					\item OptionalToEndChoiceSegmentPath ; SRN: 018 \\ \textit{Comments have to be added}
					\item OptionalToStartChoiceSegmentPath ; SRN: 019 \\ \textit{ChoiceSegmentPath and (isOptionalToEndChoiceSegmentPath value false)}
				\end{itemize}
				\item EndState ; SRN: 020 \\ \textit{An end state a behavior. A subject behavior may have one or more end states. Only Do and Receive states may be end states. Send States cannot be end states.There are no individual end states that are not Do, Send, or Receive States at the same time.}
				\item GenericReturnToOriginReference ; SRN: 021 \\ \textit{Comments have to be added}
				\item InitialStateOfBehavior ; SRN: 022 \\ \textit{The initial state of a behavior}
				\item InitialStateOfChoiceSegmentPath ; SRN: 023 \\ \textit{Similar to an initial state of a behavior a choice segment path must have one determined initial state}
				\item MacroState ; SRN: 024 \\ \textit{A state that references a macro behavior that is executed upon entering this state. Only after executing the macro behavior this state is finished also.}
				\item StandardPASSState ; SRN: 025 \\ \textit{A super class to the standard PASS states: Do, Receive and Send}
				\begin{itemize}
					\item DoState ; SRN: 026 \\ \textit{The standard state in a PASS subject behavior diagram denoting an action or activity of the subject in itself.}
					\item ReceiveState ; SRN: 027 \\ \textit{The standard state in a PASS subject behavior diagram denoting an receive action or rather the waiting for a receive possibility.}
					\item SendState ; SRN: 028 \\ \textit{The standard state in a PASS subject behavior diagram denoting a send action}
				\end{itemize}
				\item StateReference ; SRN: 029 \\ \textit{A state reference is a model component that is a reference to a state in another behavior. For most modeling aspects it is a normal state.}
			\end{itemize}
			\item Transition ; SRN: 030 \\ \textit{An edge defines the transition between two states. A transition can be traversed if the outcome of the action of the state it originates from satisfies a certain exit condition specified by it's "Alternative}
			\begin{itemize}
				\item CommunicationTransition ; SRN: 031 \\ \textit{A super class for the CommunicationTransitions.}
				\begin{itemize}
					\item ReceiveTransition ; SRN: 032 \\ \textit{Comments have to be added}
					\item SendTransition ; SRN: 033 \\ \textit{Comments have to be added}
				\end{itemize}
				\item DoTransition ; SRN: 034 \\ \textit{Comments have to be added}
				\item SendingFailedTransition ; SRN: 035 \\ \textit{Comments have to be added}
				\item TimeTransition ; SRN: 036 \\ \textit{Generic super calls for all TimeTransitions, transitions with conditions based on time events. E.g.passing of a certain time duration or the (reoccurring) calendar event. }
				\begin{itemize}
					\item ReminderTransition ; SRN: 037 \\ \textit{Reminder transitions are transitions that can be traverses if a certain time based event or frequency has been reached. E.g. a number of months since the last traversal of this transition or the event of a certain preset calendar date etc.}
					\begin{itemize}
						\item CalendarBasedReminderTransition ; SRN: 038 \\ \textit{A reminder transition, for defining exit conditions measured in calendar years or months \\ Conditions are e.g.: reaching of (in model) preset calendar date (e.g. 1st of July) or the reoccurrence of a a long running frequency ("every Month", "2 times a year")"}
						\item TimeBasedReminderTransition ; SRN: 039 \\ \textit{Comments have to be added}
					\end{itemize}
					\item TimerTransition ; SRN: 040 \\ \textit{Generic super calls for all TimeTransitions, transitions with conditions based on time events. E.g.passing of a certain time duration or the (reoccurring) calendar event. }
					\begin{itemize}
						\item BusinessDayTimerTransition ; SRN: 041 \\ \textit{imer transitions, denote time outs for the state they originate from. The condition for a timer transition is that a certain amount of time has passed since the state it originates from has been entered.\\ The time unit for this timer transition is measured in business days. The definition of a business day depends on a subject's relevant or legal location}
						\item DayTimeTimerTransition ; SRN: 042 \\ \textit{Timer Transitions, denoting time outs for the state they originate from. The condition for a timer transition is that a certain amount of time has passed since the state it originates from has been entered.\\ Day or Time Timers are measured in normal 24 hour days. Following the XML standard for time and day duration. They are to be differed from the timers that are timeout in units of years or months.}
						\item YearMonthTimerTransition ; SRN: 044 \\ \textit{Timer transitions, denote time outs for the state they originate from. The condition for a timer transition is that a certain amount of time has passed since the state it originates from has been entered.\\ Year or Month timers measure time in calendar years or months. The exact definitions for years and months depends on relevant or legal geographical location of the subject.}
					\end{itemize}
					\item UserCancelTransition ; SRN: 045 \\ \textit{A user cancel transition denotes the possibility to exit a receive state without the reception of a specific message.\\ The user cancel allows for an arbitrary decision by a subject carrier/processor to abort a waiting process.}
				\end{itemize}
				\item TransitionCondition ; SRN: 046 \\ \textit{An exit condition belongs to alternatives which in turn is given for a state. An alternative (to leave the state) is only a real alternative if the exit condition is fulfilled (technically: if that according function returns "true") \\Note: Technically and during execution exit conditions belong to states. They define when it is allowed to leave that state. However, in PASS models exit conditions for states are defined and connected to the according transition edges. Therefore transition conditions are individual entities and not DataProperties.\\ The according matching must be done by the model execution environment.\\ By its existence, an edge/transition defines one possible follow up "state" for its state of origin. It is coupled with an "Exit Condition" that must be fulfilled in the originating state in order to leave the state.}
				\begin{itemize}
					\item DoTransitionCondition ; SRN: 047 \\ \textit{A TransitionCondition for the according DoTransitions and DoStates. }
					\item MessageExchangeCondition ; SRN: 048 \\ \textit{MessageExchangeConditon is the super class for Send End Receive Transition Conditions the both require either the sending or receiving (exchange) of a message to be fulfilled.}
					\begin{itemize}
						\item ReceiveTransitionCondition ; SRN: 049 \\ \textit{ReceiveTransitionConditions are conditions that state that a certain message must have been taken out of a subjects in-box to be fulfilled.\\ These are the typical conditions defined by Receive Transitions.}
						\item SendTransitionCondition ; SRN: 050 \\ \textit{SendTransitionConditions are conditions that state that a certain message must have been successfully passed to another subjects in-box to be fulfilled.\\ These are the typical conditions defined by Send transitions.}
						\end {itemize}
						\item SendingFailedCondition ; SRN: 051 \\ \textit{Comments have to be added}
						\item TimeTransitionCondition ; SRN: 052 \\ \textit{A condition that is deemed 'true' and thus the according edge is gone, if: a surrounding execution system has deemed the time since entering the state and starting with the execution of the according action as too long (predefined by the outgoing edge) \\ A condition that is true if a certain time defined has passed since the state this condition belongs to has been entered. (This is the standard TimeOut Exit condition)}
						\begin{itemize} 
							\item ReminderEventTransitionCondition ; SRN: 053 \\ \textit{Comments have to be added}
							%					\begin{itemize}
							%						\item CalendarBasedReminderTransitionCondition
							%						\item TimeBasedReminderTimeOutTransitionCondition
							%					\end{itemize}
							\item TimerTransitionCondition ; SRN: 054 \\ \textit{Comments have to be added}
							%					\begin{itemize}
							%						\item BusinessDayTimerTransitionCondition
							%						\item DayTimeTimerCondition
							%						\item YearMonthTimerTransitionCondition
							%					\end{itemize}
						\end{itemize}
					\end{itemize}
				\end{itemize}
			\end{itemize}		
			
			\item DataDescribingComponent ; SRN: 055 \\ \textit{Subject-Oriented PASS Process Models are in general about describing the activities and interaction of active entities. Yet these interactions are rarely done without data that is being generated by activities and transported via messages. While not considered by Börger's PASS interpreter, the community agreed on adding the ability to integrate the means to describe data objects or data structures to the model and enabling their connection to the process model. It may be defined that messages or subject have their individual $DataObjectDefinition$ in form of a $SubjectDataDefinition$ in the case of $FullySpecifiedSubject$s and \\ $PayloadDataObjectDesfinition$ in the case of \\ $MessageSpecifications$ In general, it expected that these \\ $DataObjectDefinition$ list on or more data fields for the message or subject with an internal data type that is described via a $DataTypeDefinition$. There is a rudimentary concept for a simple build-in data type definition closely oriented at the concept of ActNConnect. Otherwise, the principle idea of the OWL standard is to allow and employ existing or custom technologies for the serialized definition of data structures \\ ($CustomOrExternalDataTypeDefinition$) such as XML-Schemata (XSD), according elements with JSON or directly the powerful expressiveness of OWL itself.}
			\begin{itemize}
				\item DataObjectDefinition ; SRN: 056 \\ \textit{Data Object Definitions are model elements used to describe that certain other model elements may posses or carrier Data Objects.\\ E.G. a message may carrier/include a Business Objects. Or the private Data Space of a Subject may contain several Data Objects. \\A Data Objects should refer to a DataTypeDefinition denoting its DataType and structure.\\ DataObject: states that a data item does exist (similar to a variable in programming)DataType: the definition of an Data Object's structure.}
				\begin{itemize}
					\item DataObjectListDefintion ; SRN: 057 \\ \textit{Data definition concept for PASS model build in capabilities of data modeling. Defines a simple list structure.}
					\item PayloadDataObjectDefinition ; SRN: 058 \\ \textit{Messages may have a description regarding their payload (what is transported with them).\\This can either be a description of a physical (real) object or a description of a (digital) data object}
					\item SubjectDataDefinition ; SRN: 059 \\ \textit{Comments have to be added}
				\end{itemize}
				\item DataTypeDefinition ; SRN: 060 \\ \textit{Data Type Definitions are complex descriptions of the supposed structure of Data Objects. \\ DataObject: states that a data item does exist (similar to a variable in programming). \\DataType: the definition of an Data Object's structure.}
				\begin{itemize}
					\item CustomOrExternalDataTypeDefinition ; SRN: 061 \\ \textit{Using this class, tool vendors can include their own custom data definitions in the model.}
					\begin{itemize}
						\item JSONDataTypeDefinition ; SRN: 062 \\ \textit{Comments have to be added}
						\item OWLDataTypeDefinition ; SRN: 63 \\ \textit{Comments have to be added}
						\item XSD-DataTypeDefinition ; SRN: 064 \\ \textit{XML Schemata Description (XSD) is an established technology for describing structure of Data Objects (XML documents) with many tools available that can verify a document against the standard definition}
					\end{itemize}
					\item ModelBuiltInDataTypes ; SRN: 065 \\ \textit{Comments have to be added}
				\end{itemize}
				\item PayloadDescription ; SRN: 066 \\ \textit{Comments have to be added}
				\begin{itemize}
					\item PayloadDataObjectDefinition ; SRN: 067 \\ \textit{Messages may have a description regarding their payload (what is transported with them).\\This can either be a description of a physical (real) object or a description of a (digital) data object}
					\item PayloadPhysicalObjectDescription ; SRN: 068 \\ \textit{Messages may have a description regarding their payload (what is transported with them).\\This can either be a description of a physical (real) object or a description of a (digital) data object}
				\end{itemize}
			\end{itemize}
			\item InteractionDescribingComponent ; SRN: 069 \\ \textit{This class is the super class of all model elements used to define or specify the interaction means within a process model}
			\begin{itemize}
				\item InputPoolConstraint ; SRN: 070 \\ \textit{Subjects do implicitly posses input pools.\\
					During automatic execution of a PASS model in a work-flow engine this message box is filled with messages.\\
					Without any constraints models this message in-box is assumed to be able to store an infinite amount of messages.\\
					For some modeling concepts though it may be of importance to restrict the size of the input pool for certain messages or senders.\\This is done using several different Type of InputPoolConstraints that are attached to a fully specified subject.\\Should a constraint be applicable, an "InputPoolConstraintHandlingStrategy" will be executed by a work-flow engine to determine what to do with the message that does not fit in the pool.\\
					Limiting the input pool for certain reasons to size 0 together with the InputPoolConstraintStrategy-Blocking is effectively modeling that a communication must happen synchronously instead of the standard asynchronous mode. The sender can send his message only if the receiver is in an according receive state, so the message can be handled directly without being stored in the in-box.}
				\begin{itemize}
					\item MessageSenderTypeConstraint ; SRN: 071 \\ \textit{An InputPool constraint that limits the number of message of a certain type and from a certain sender in the input pool.\\
						E.g. "Only one order from the same customer" (during happy hour at the bar)}
					\item MessageTypeConstraint ; SRN: 072 \\ \textit{An InputPool constraint that limits the number of message of a certain type in the input pool.\\
						E.g. You can accept only "three request at once}
					\item SenderTypeConstraint ; SRN: 073 \\ \textit{An InputPool constraint that limits the number of message from a certain Sender subject in the input pool.\\
						E.g. as long as a customer has non non-fulfilled request of any type he may not place messages}
				\end{itemize}
				\item InputPoolContstraintHandlingStrategy ; SRN: 074 \\ \textit{Should an InputPoolConstraint be applicable, an "InputPoolConstraintHandlingStrategy" will be executed by a work-flow engine to determine what to do with the message that does not fit in the pool.\\
					There are types of HandlingStrategies. \\
					InputPoolConstraintStrategy-Blocking - No new message will be adding will need to be repeated until successful \\
					InputPoolConstraintStrategy-DeleteLatest - The new message will be added, but the last message to arrive before that applicable to the same constraint will be overwritten with the new one. (LIFO deleting concept)\\
					InputPoolConstraintStrategy-DeleteOldest - The message will be added, but the earliest message in the input pool applicable to the same constraint will be deleted (FIFO deleting concept)\\
					InputPoolConstraintStrategy-Drop - Sending of the message succeeds. However the new message will not be added to the in-box. Rather it will be deleted directly.}
				\item MessageExchange ; SRN: 075 \\ \textit{A message exchange is an element in the interaction description section that specifies exactly one possibility of exchanging messages in the given process context of the model.\\
					A message exchange is a triple of, a sender, a receiver, and the specification of the message that may be exchanged.\\
					While message exchanges are singular occurrences, they may be grouped in MessageExchangeLists}
				\item MessageExchangeList ; SRN: 076 \\ \textit{While MessageExchanges are singular occurrences, they may be grouped in MessageExchangeLists.\\
					In graphical PASS modeling that is usually the case when one arrow between two subjects contains more than one message and thereby specifies more than one possible message exchange channel between the two subjects.}
				\item MessageSpecification ; SRN: 077 \\ \textit{MessageSpecification are model elements that specify the existence of a message. At minimum its name and id.\\It may contain additional specification for its payload (contained Data, exact form etc.)}
				\item Subject ; SRN: 078 \\ \textit{The subject is the core model element of a subject-oriented PASS process model.}
				\begin{itemize}
					\item FullySpecifiedSubject ; SRN: 079 \\ \textit{Fully specified Subjects in a PASS graph are entities that, in contrast to interface subjects, linked to one ore more Behaviors (they posses a behavior).}
					\item InterfaceSubject ; SRN: 080 \\ \textit{Interface Subjects are Subjects that are not linked to a behavior. In contrast, they may refer to FullySpecifiedSubjects that are described in other process models.}
					\item MultiSubject ; SRN: 081 \\ \textit{The Multi-Subject is term for a subject that "has a maximum subject instantiation restriction" within a process context larger than 1.}
					\item SingleSubject ; SRN: 082 \\ \textit{Single Subject are subject with a maximumInstanceRestriction of 1}
					\item StartSubject ; SRN: 083 \\ \textit{Subjects that start their behavior with a Do or Send state are active in a process context from the beginning instead of requiring a message from another subject.\\
						Usually there should be only one Start subject in a process context.}
				\end{itemize}
			\end{itemize}
			
			\item PASSProcessModel ; SRN: 084 \\ \textit{The main class that contains all relevant process elements}
			\item SubjectBehavior ; SRN: 085 \\ \textit{Additional to the subject interaction a PASS Model consist of multiple descriptions of subject's behaviors. These are graphs described with the means of $BehaviorDescribingComponents$ \\
				A subject in a model may be linked to more than one behavior.}
			\begin{itemize}
				\item GuardBehavior ; SRN: 086 \\ \textit{A guard behavior is a special usually additional behavior that guards the Base Behavior of a subject.ßß
					It starts with a (guard) receive state denoting a special interrupting message. Upon reception of that message the subject will execute the according receive transition and the follow up states until it is either redirected to a state on the base behavior or terminates in an end-state within the guard behavior}
				\item MacroBehavior ; SRN: 087 \\ \textit{A macro behavior is a specialized behavior that may be entered and exited from a function state in another behavior.}
				\item SubjectBaseBehavior ; SRN: 088 \\ \textit{The standard behavior model type}
			\end{itemize}
		\end{itemize}
		
		\item SimplePASSElement ; SRN: 089 \\ \textit{Comments have to be added}
		\begin{itemize}
			\item CommunicationTransition ; SRN: 090 \\ \textit{A super class for the CommunicationTransitions.}
			\begin{itemize}
				\item ReceiveTransition ; SRN: 091 \\ \textit{Comments have to be added}
				\item SendTransition ; SRN: 092 \\ \textit{Comments have to be added}
			\end{itemize}
			\item DataMappingFunction ; SRN: 093 \\ \textit{Definitions of the ability/need to write or read data to and from a subject's personal data storage.\\
				DataMappingFunctions are behavior describing components since they define what the subject is supposed to do (mapping and translating data)\\
				Mapping may be done during reception of message, where data is taken from the message/Business Object (BO) and mapped/put into the local data field.\\
				It may be done during sending of a message where data is taken from the local vault and put into a BO.\\
				Or it may occur during executing a do function, where it is used to define read(get) and write (set) functions for the local data.}
			\begin{itemize}
				\item DataMappingIncomingToLocal ; SRN: 094 \\ \textit{A DataMapping that specifies how data is mapped from an an external source (message, function call etc.) to a subject's private defined data space.}
				\item DataMappingLocalToOutgoing ; SRN: 095 \\ \textit{A DataMapping that specifies how data is mapped from a subject's private data space to an an external destination (message, function call etc.)" }
			\end{itemize}
			\item DoTransition ; SRN: 096 \\ \textit{Comments have to be added}
			\item DoTransitionCondition ; SRN: 097 \\ \textit{A TransitionCondition for the according DoTransitions and DoStates.}
			\item EndState ; SRN: 098 \\ \textit{An end state a behavior. A subject behavior may have one or more end states. Only Do and Receive states may be end states. Send States cannot be end states.\\
				There are no individual end states that are not Do, Send, or Receive States at the same time.}
			\item FunctionSpecification ; SRN: 099 \\ \textit{A function specification for state denotes\\
				Concept: Definitions of calls of (mostly technical) functions (e.g. Web-service, Scripts, Database access,) that are not part of the process model.\\
				Function Specifications are more than "Data Properties"? --> - If special function types (e.g. Defaults) are supposed to be reused, having them as explicit entities is a the better OWL-modeling choice.}
			\begin{itemize}
				\item CommunicationAct ; SRN: 100 \\ \textit{A super class for specialized FunctionSpecification of communication acts (send and receive)}
				\begin{itemize}
					\item ReceiveFunction ; SRN: 101 \\ \textit{Specifications/descriptions for Receive-Functions describe in detail what the subject carrier is supposed to do in a state.\\
						DefaultFunctionReceive1\_EnvoironmentChoice : present the surrounding execution environment with the given exit choices/conditions currently available depending on the current state of the subjects in-box. Waiting and not executing the receive action is an option.\\
						DefaultFunctionReceive2\_AutoReceiveEarliest: automatically execute the according activity with the highest priority as soon as possible. In contrast to DefaultFunctionReceive1, it is not an option to prolong the reception and wait e.g. for another message.}
					\item SendFunction ; SRN: 102 \\ \textit{Comments have to be added}
				\end{itemize}
				\item DoFunction ; SRN: 103 \\ \textit{Specifications or descriptions for Do-Functions describe in detail what the subject carrier is supposed to do in an according state.\\
					The default DoFunction 1: present the surrounding execution environment with the given exit choices/conditions and receive choice of one exit option --> define its Condition to be fulfilled in order to go to the next according state.\\
					The default DoFunction 2: execute automatic rule evaluation (see DoTransitionCondition).\\
					More specialized Do-Function Specifications may contain Data mappings denoting what of a subjects internal local Data can and should be:\\
					a) read: in order to simply see it or in order to send it of to an external function (e.g. a web service)\\
					b) write: in order to write incoming Data from e.g. a web Service or user input, to the local data fault}
			\end{itemize}
			\item InitialStateOfBehavior ; SRN: 104 \\ \textit{The initial state of a behavior}
			\item MessageExchange ; SRN: 105 \\ \textit{A message exchange is an element in the interaction description section that specifies exactly one possibility of exchanging messages in the given process context of the model.\\
				A message exchange is a triple of, a sender, a receiver, and the specification of the message that may be exchanged.\\
				While message exchanges are singular occurrences, they may be grouped in MessageExchangeLists}
			\item MessageExchangeCondition ; SRN: 106 \\ \textit{MessageExchangeConditon is the super class for Send End Receive Transition Conditions the both require either the sending or receiving (exchange) of a message to be fulfilled.}
			\begin{itemize}
				\item ReceiveTransitionCondition ; SRN: 107 \\ \textit{ReceiveTransitionConditions are conditions that state that a certain message must have been taken out of a subjects in-box to be fulfilled.\\
					These are the typical conditions defined by Receive Transitions.}
				\item SendTransitionCondition ; SRN: 108 \\ \textit{SendTransitionConditions are conditions that state that a certain message must have been successfully passed to another subjects in-box to be fulfilled.\\
					These are the typical conditions defined by Send transitions.}
			\end{itemize} 
			\item MessageExchangeList ; SRN: 109 \\ \textit{While MessageExchanges are singular occurrences, they may be grouped in MessageExchangeLists.\\
				In graphical PASS modeling that is usually the case when one arrow between two subjects contains more than one message and thereby specifies more than one possible message exchange channel between the two subjects.}
			\item MessageSpecification ; SRN: 110 \\ \textit{MessageSpecification are model elements that specify the existence of a message. At minimum its name and id.\\
				It may contain additional specification for its payload (contained Data, exact form etc.)}
			\item ModelBuiltInDataTypes ; SRN: 111 \\ \textit{Comments have to be added}
			\item PayloadDataObjectDefinition ; SRN: 112 \\ \textit{Messages may have a description regarding their payload (what is transported with them).\\
				This can either be a description of a physical (real) object or a description of a (digital) data object}
			\item StandardPASSState ; SRN: 113 \\ \textit{A super class to the standard PASS states: Do, Receive and Send}
			\begin{itemize}
				\item DoState ; SRN: 114 \\ \textit{The standard state in a PASS subject behavior diagram denoting an action or activity of the subject in itself.}
				\item ReceiveState ; SRN: 115 \\ \textit{The standard state in a PASS subject behavior diagram denoting an receive action or rather the waiting for a receive possibility.}
				\item SendState ; SRN: 116 \\ \textit{The standard state in a PASS subject behavior diagram denoting a send action}
			\end{itemize}
			\item Subject ; SRN: 117 \\ \textit{The subject is the core model element of a subject-oriented PASS process model.}
			\begin{itemize}
				\item FullySpecifiedSubject ; SRN: 118 \\ \textit{Fully specified Subjects in a PASS graph are entities that, in contrast to interface subjects, linked to one ore more Behaviors (they posses a behavior).}
				\item InterfaceSubject ; SRN: 119 \\ \textit{Interface Subjects are Subjects that are not linked to a behavior. In contrast, they may refer to FullySpecifiedSubjects that are described in other process models.}
				\item MultiSubject ; SRN: 120 \\ \textit{The Multi-Subject is term for a subject that "has a maximum subject instantiation restriction" within a process context larger than 1.}
				\item SingleSubject ; SRN: 121 \\ \textit{Single Subject are subject with a maximumInstanceRestriction of 1}
				\item StartSubject ; SRN: 122 \\ \textit{Subjects that start their behavior with a Do or Send state are active in a process context from the beginning instead of requiring a message from another subject.\\
					Usually there should be only one Start subject in a process context.}
			\end{itemize}
			\item SubjectBaseBehavior ; SRN: 123 \\ \textit{The standard behavior model type}
		\end{itemize}
	\end{itemize}	
	
	\normalsize
	
	\section{Object Properties (42)}
	
	\footnotesize
	\begin{landscape}
		\begin {longtable} {| p{0.3\textwidth} | p{0.11\textwidth} | p{0.36\textwidth}|p{0.4\textwidth}| p{0.12\textwidth}|}
		%\begin {longtable} {| p{0.3\textheight} | p{0.11\textheight} | p{0.30\textheight}|p{0.4\textheight}| p{0.12\textheight}|}
		\hline
		Property name &  & Domain-Range & Comments &Reference\\
		\toprule
		\endhead
		\hline
		belongsTo & Domain: & PASSProcessModelElement &Generic ObjectProperty that links two process elements, where one is contained in the other (inverse of contains). & \ \ 200 \\
		& Range: & PASSProcessModelElement & &\\
		\hline
		contains & Domain: &PASSProcessModelElement&Generic ObjectProperty that links two model elements where one contains another (possible multiple) & \ \ 201\\
		& Range: & PASSProcessModelElement & & \\
		\hline
		containsBaseBehavior & Domain: &Subject & &\ \ 202\\ 
		& Range: &SubjectBehavior & &\\
		\hline
		containsBehavior & Domain: &Subject & &\ \ 203\\ 
		& Range: & SubjectBehavior & &\\
		\hline
		containsPayload-Description & Domain: & MessageSpecification & & \ \ 204\\
		& Range: &PayloadDescription & &\\
		\hline
		guardedBy & Domain: &State, Action & & \ \ 205\\
		& Range: &GuardBehavior & &\\
		\hline
		guardsBehavior &Domain: &GuardBehavior & Links a GuardBehavior to another SubjectBehavior. Automatically all individual states in the guarded behavior are guarded by the guard behavior. There is an SWRL Rule in the ontology for that purpose.& \ \ 206 \\
		& Range: &SubjectBehavior &  &\\
		\hline
		guardsState & Domain: &State, Action & &\ \ 207\\
		& Range: &guardedBy & & \\
		\hline
		hasAdditionalAttribute & Domain: &PASSProcessModelElement& &\ \ 208\\
		& Range: &AdditionalAttribute&  &\\
		\hline
		hasCorrespondent & Domain: & &Generic super class for the ObjectProperties that link a Subject with a MessageExchange either in the role of Sender or Receiver. & \ \ 209\\
		& Range: &Subject & &\\
		\hline
		hasDataDefinition &Domain: &  & & \ \ 210\\
		& Range: &DataObjectDefinition & &\\ 
		\hline
		hasDataMapping-Function &Domain: &state, SendTransition, ReceiveTransition & &\ \ 211\\
		& Range: &DataMappingFunction & & \\
		\hline 
		hasDataType & Domain: &PayloadDescription or DataObjectDefinition & &\ \ 212\\
		& Range: &DataTypeDefinition &  &\\
		\hline
		hasEndState & Domain: &SubjectBehavior or ChoiceSegmentPath & &\ \ 213\\
		& Range: &State, not SendState &  &\\
		\hline
		hasFunction-Specification & Domain: &State& &\ \ 214\\
		& Range: &FunctionSpecification&  &\\
		\hline
		hasHandlingStrategy &Domain: &InputPoolConstraint & &\ \ 215\\
		& Range: &InputPoolContstraint-HandlingStrategy &  &\\
		\hline
		hasIncomingMessage-Exchange & Domain: &Subject& &\ \ 216\\
		& Range: &MessageExchange &  &\\
		\hline
		hasIncomingTransition &Domain: &State & &\ \ 217\\
		& Range: &Transition &  &\\
		\hline
		hasInitialState & Domain: &SubjectBehavior or ChoiceSegmentPath & &\ \ 218\\
		& Range: &State &  &\\
		\hline
		hasInputPoolConstraint &Domain: &Subject & &\ \ 219\\
		& Range: &InputPoolConstraint &  &\\
		\hline
		hasKeyValuePair &Domain: & & &\ \ 220\\
		& Range: & &  &\\
		\hline
		hasMessageExchange & Domain: &Subject & Generic super class for the ObjectProperties linking a subject with either incoming or outgoing MessageExchanges.&\ \ 221\\
		& Range: & &  &\\
		\hline
		hasMessageType & Domain: &MessageTypeConstraint or  MessageSenderTypeConstraint or  MessageExchange & &\ \ 222\\
		& Range: &MessageSpecification &  &\\
		\hline
		hasOutgoingMessage-Exchange & Domain: &Subject& &\ \ 223\\
		& Range: &MessageExchange&  &\\
		\hline
		hasOutgoingTransition &Domain: &State & &\ \ 224\\
		& Range: &Transition&  &\\
		\hline
		hasReceiver &Domain: &MessageExchange & &\ \ 225\\
		& Range: &Subject & &\\
		\hline
		hasRelationToModel-Component & Domain: &PASSProcessModelElement&Generic super class of all object properties in the standard-pass-ont that are used to link model elements with one-another. &\ \ 226\\
		& Range: &PASSProcessModelElement & & \\
		\hline
		hasSender &Domain: &MessageExchange && \ \ 227\\
		& Range: &Subject & &\\
		\hline
		hasSourceState & Domain: &Transition& &\ \ 228\\
		& Range: &State&  &\\
		\hline
		hasStartSubject & Domain: &PASSProcessModel& &\ \ 229\\
		& Range: &StartSubject& & \\
		\hline
		hasTargetState & Domain &Transition& &\ \ 230\\
		& Range &State& & \\
		\hline
		hasTransitionCondition &Domain &Transition & &\ \ 231\\
		& Range &TransitionCondition & & \\
		\hline
		isBaseBehaviorOf &Domain: &SubjectBaseBehavior & A specialized version of the "belongsTo" ObjectProperty to denote that a -SubjectBehavior belongs to a Subject as its BaseBehavior&\ \ 232\\
		& Range: &&  &\\
		\hline
		\pagebreak
		isEndStateOf & Domain: &State and not SendState & &\ \ 233\\
		& Range: &SubjectBehavior or ChoiceSegmentPath &  &\\
		\hline
		isInitialStateOf & Domain: &State& &\ \ 234\\
		& Range: &SubjectBehavior or ChoiceSegmentPath &  &\\
		\hline
		isReferencedBy & Domain: & & &\ \ 235\\
		& Range: &&  &\\
		\hline
		references & Domain: & & &\ \ 236\\
		& Range: & &  &\\
		\hline
		referencesMacroBehavior &Domain: &MacroState & &\ \ 237\\
		& Range: &MacroBehavior & & \\
		\hline
		refersTo & Domain: &CommunicationTransition&Communication transitions (send and receive) should refer to a message exchange that is defined on the interaction layer of a model. & \ \ 238\\
		& Range: &MessageExchange& & \\
		\hline
		requiresActiveReception-OfMessage &Domain: &ReceiveTransitionCondition & &\ \ 239\\
		& Range: &MessageSpecification &  &\\
		\hline
		requiresPerformed-MessageExchange & Domain: &MessageExchangeCondition& &\ \ 240\\
		& Range: &MessageExchange &  &\\
		\hline
		SimplePASSObject-Propertie & Domain: & &Every element/sub-class of SimplePASSObjectProperties is also a Child of PASSModelObjectPropertiy. This is simply a surrogate class to group all simple elements together &\ \ 241\\
		& Range: & &  &\\
		\hline
	\end{longtable}
	\end {landscape}
	
	
	\normalsize
	
	\section{Data Properties (27)}
	
	\footnotesize
	
	\begin{landscape}
		\begin {longtable} {| p{0.5
				\textwidth} | p{0.11\textwidth} | p{0.3\textwidth}|p{0.3\textwidth}|p{0.12\textwidth}|}
		\hline
		Property name &  & Domain-Range & Comments &Reference\\
		\toprule
		\endhead
		\hline
		hasBusinessDayDurationTimeOutTime & Domain: &  & &\\
		& Range: &  & &\\
		\hline
		hasCalendarBasedFrequencyOrDate & Domain: &  & &\\
		& Range: &  & &\\
		\hline
		hasDataMappingString & Domain: &  & &\\
		& Range: &  & &\\
		\hline
		hasDayTimeDurationTimeOutTime & Domain: &  & &\\
		& Range: &  & &\\
		\hline
		hasDurationTimeOutTime & Domain: &  & &\\
		& Range: &  & &\\
		\hline
		hasFeelExpressionAsDataMapping & Domain: &  &See \url{https://www.omg.org/spec/DMN} for specification of Feel-Statement-Strings
		
		The idea of these expression is to map data fields from and to the internal Data storage of a subject &\\
		& Range: &  & &\\
		\hline
		hasGraphicalRepresentation & Domain: &  & The process models are in principle abstract graph structures. Yet the visualization of process models is very important since many process models are initially created in a graphical form using a graph editor that was created to foster human comprehensibility. If available any process element may have a graphical representation attached to it&\\
		& Range: &  & &\\
		\hline
		hasKey & Domain: &  & &\\
		& Range: &  & &\\
		\hline
		hasLimit & Domain: &  & &\\
		& Range: &  & &\\
		\hline
		hasMaximumSubjectInstanceRestriction & Domain: &  & &\\
		& Range: &  & &\\
		\hline
		hasMetaData & Domain: &  & &\\
		& Range: &  & &\\
		\hline
		hasModelComponentComment & Domain: &  &equivalent to rdfs:comment &\\
		& Range: &  & &\\
		\hline
		hasModelComponentID & Domain: &  &The unique ID of a PASSProcessModelComponent &\\
		& Range: &  & &\\
		\hline
		hasModelComponentLabel & Domain: &  &The human legible label or description of a model element. &\\
		& Range: &  & &\\
		\hline
		hasPriorityNumber & Domain: &  &Transitions or Behaviors have numbers that denote their execution priority in situations where two or more options could be executed.
		
		This is important for automated execution.
		
		E.g. when two messages are in the in-box and could be followed, the message denoted on the transition with the higher priority (lower priority number) is taken out and processed.
		
		Similarly, SubjectBehaviors with higher priority (lower priority number) are to be executed before Behaviors with lower priority. &\\
		& Range: &  & &\\
		\hline
		hasReoccuranceFrequenyOrDate & Domain: &  &A data field meant for the two classes ReoccuranceTimeOutTransition and ReoccuranceTimeOutExitCondition.
		
		ToDo: Define the according data format for describing the iteration frequencies or reoccurring dates. Opinion: rather complex: expressive capabilities should cover expressions like: "every 2nd Monday of Month at 7:30 in Morning." Every 29th of July" or "Every Hour", "ever 25 Minuets" , "once each day", "twice each week" etc &\\
		& Range: &  & &\\
		\hline
		hasSVGRepresentation & Domain: &  & The Scalable Vector Graphic (SVG) XML format is a text based standard to describe vector graphics.
		
		Adding according image information as XML literals is therefor a suitable, yet not necessarily easily changeable option to include the graphical representation of model elements in the an OWL file.&\\
		& Range: &  & &\\
		\hline
		hasTimeBasedReoccuranceFrequencyOrDate & Domain: &  & &\\
		& Range: &  & &\\
		\hline
		hasTimeValue & Domain: &  & Generic super class for all data properties of time based transitions.&\\
		& Range: &  & &\\
		\hline
		hasToolSpecificDefinition & Domain: &  & This is a placeholder DataProperty meant as a tie in point for tool vendors to include tool specific data values/properties into models.
		
		By denoting their own data properties as sub-classes to this one the according data fields can easily be recognized as such. However, this is only an option and a place holder to remind that something like this is possible.&\\
		& Range: &  & &\\
		\hline
		hasValue & Domain: &  & &\\
		& Range: &  & &\\
		\hline
		hasYearMonthDurationTimeOutTime & Domain: &  & &\\
		& Range: &  & &\\
		\hline
		isOptionalToEndChoiceSegmentPath & Domain: &  & &\\
		& Range: &  & &\\
		\hline
		isOptionalToStartChoiceSegmentPath & Domain: &  & &\\
		& Range: &  & &\\
		\hline
		owl:topDataProperty & Domain: &  & &\\
		& Range: &  & &\\
		\hline
		PASSModelDataProperty & Domain: &  &Generic super class of all DataProperties that PASS process model elements may have. &\\
		& Range: &  & &\\
		\hline
		SimplePASSDataProperties
		& Domain: &  & Every element/sub-class of SimplePASSDataProperties is also a Child of PASSModelDataPropertiy. This is simply a surrogate class to group all simple elements together&\\
		& Range: &  & &\\
		\hline
	\end{longtable}
	\end {landscape}
	

\chapter{An ASM Interpreter Model for PASS}
\label{ASM-Interpreter}

The following tables show the relationships between the PASS ontology and the PASS execution semantics described as ASMs.
Because of historical reasons in the ASMs names for entities and relations are different from the names used in the ontology.
The tables below show the mapping of the entity and relation names in the ontology to the names used in the ASMs.

\section{Mapping of ASM Places to OWL Entities}
Places are formally also functions or rules, but are used in principle as passive/static storage places.

%appendix in smaller font size
\footnotesize

\begin{landscape}
	\begin {longtable} {| p{0.5\textwidth} | p{0.3\textwidth} | p{0.6\textwidth}|}
	\hline
	OWL Model element &   ASM interpreter & Description\\
	\toprule
	\endhead
	\hline
	
	X - Execution concept – the state the subject is currently in as defined by a \textbf{State} in the model 
	& \textit{SID\_state} 
	&  Execution concept – no model representation, Not to be confused by a model “state” in an SBD Diagram. State in the SBD diagram define possible SID\_States.
	\\
	\hline
	
	\textbf{SubjectBehavior }	– under the assumption that it is complete and sound.
	& \textit{D} 
	&  A Diagram that is a completely connected SBD
	\\
	\hline
	
	\textbf{State}
	& \textit{node} 
	&  A specific element of diagram D -	Every node 1:1 to state
	\\
	\hline
	
	\textbf{State}
	& \textit{state} 
	& The current active state of a diagram determined by the nodes of Diagram D
	\\
	\hline
	
	\textbf{InitialStateOfBehavior, \newline EndState }
	& \textit{initial state, \newline end state} 
	& The interpreter expects and SBD Graph D to contain exactly one initial (start) state and at least one end state.
	\\
	\hline
	
	\textbf{Transition}
	& \textit{edge / outEdge} 
	& “Passive Element” of an edge in an SBD-graph
	\\
	\hline
	
	\textbf{TransitionConditionn}
	& \textit{ExitCondition} 
	& Static Concept that represents a Data condition
	\\
	\hline
	
	Execution Concept – ID of a Subject Carrier responsible possible multiple Instances of according to specific  \textbf{SubjectBehavior}
	& \textit{subj} 
	& Identifier for a specific Subject Carrier that may be responsible for multiple Subjects
	\\
	\hline
	
	Represented in the model with \textbf{InterfaceSubject}
	& \textit{ExternalSubject} 
	& A representation of a service execution entity outside of the boundaries of the interpreter
	(The PASS-OWL Standardization community decided on the new Term of Interface Subject to replace the often-misleading older term of External Subject)
	
	\\
	\hline
	\textbf{SubjectBehavior} or rather \textbf{SubjectBaseBehavior} as MacroBehaviors and GuardBehaviors are not covered by Börger
	& \textit{subject-SBD / \newline SBDsubject\textsubscript{subject}} 
	& Names for completely connected graphs / diagrams representing SBDs
	\\
	\hline
	
	Object Property: \textbf{\textit{hasFunctionSpecification}} 
	(linking \textbf{State}, and \textbf{FunctionSpecification} -->(\textbf{State} \textbf{\textit{hasFunctionSpecification }} \textbf{FunctionSpecification})
	& \textit{service(state) / \newline service(node)} 
	& Rule/Function that reads/returns the service of function of a given state/node
	\\
	\hline
	
	\textbf{DoState} \newline \textbf{SendState} \newline \textbf{ReceiveState}
	& \textit{function state, \newline send state, \newline receive state} 
	& The ASM spec does not itself contain these terms. The description text, however, uses them to describe states with an according service (e.g. a state in which a (ComAct = Send) service is executed is referred to as a send state)
	Seen from the other side: a SendState is a state with service(state) = Send)
	\newline
	Both send and receive services are a ComAct service.
	The ComAct service is used to define common rules of these communication services.
	
	\\
	\hline
	\textbf{CommunicationActs} with sub-classes (\textbf{ReceiveFunction SendFunction}) \newline
	\textbf{\textit{DefaultFunctionReceive1\_EnvironmentChoice \newline
			DefaultFunctionReceive2\_AutoReceiveEarliest \newline
			DefaultFunctionSend }}
	& \textit{ComAct} 
	& Specialized version of Perform-ASM Rule for communication, either send or receive. These rules distinguish internally between send and receive.
	\\
	\hline
	
	
\end{longtable}
\end {landscape}

\section{Main Execution/Interpreting Rules}
The interpreter ASM Spec has main-function or rules that are being executed while interpreted.

\begin{itemize}
	\item BEHAVIOR(subj,state)
	\item PROCEED(subj,service(state),state)
	\item PERFORM(subj,service(state),state)
	\item START (subj,X, node)
\end{itemize}

These make up the main interpreter algorithm for PASS SBDs and therefore have no corresponding model elements but rather are or contain the instructions of how to interpret a model.





\begin{landscape}
	\begin {longtable} {| p{0.7\textwidth} | p{0.3\textwidth} | p{0.4\textwidth}|}
	\hline
	OWL Model element &   ASM interpreter & Description\\
	\toprule
	\endhead
	\hline
	
	Execution concept
	& \textit{BEHAVIOR(subj;state)} 
	&  Main interpreter ASM-rule/Method
	\\
	\hline
	
		Execution concept
	& \textit{BEHAVIOR(subj;node)}
	& ASM-Rule to interpret a specific node of Diagram D for a specific subject
	\\
	\hline
	
		Execution concept
	& \textbf{Behaviorsubj (D)}
	&  Set of all ASM rules to interprete all nodes/states in a SBD(iagram) D for a given subj (set of all \textit{BEHAVIOR(subj;node)}
	\\
	\hline
	
		\textcolor{orange}{\textbf{State}} \textcolor{blue}{\textbf{hasFunctionSpecification}} \textcolor{orange}{\textbf{FunctionSpecification}}
		\newline
		Specialized in:
			\newline
		\textcolor{orange}{\textbf{DoFunction}} and
			\newline
		\textcolor{orange}{\textbf{CommunicationActs}} with
			\newline
		\textcolor{orange}{\textbf{ReceiveFunction}}
			\newline
		\textcolor{orange}{\textbf{SendFunction}}
			\newline
		There exist a few default activities:
			\newline
		\textcolor{purple}{\textbf{DefaultFunctionDo1\_EnvoironmentChoice}}
			\newline
		\textcolor{purple}{\textbf{DefaultFunctionDo2\_AutomaticEvaluation}}
	& \textit{PERFORM(subj ; service(state); state)}
	&  Main interpreter ASM-rule/Method
	\\
	\hline
	
	\textcolor{orange}{\textbf{CommunicationActs}} with
	\newline
	\textcolor{orange}{\textbf{ReceiveFunction}}
	\newline
	\textcolor{orange}{\textbf{SendFunction}}
	\newline
	\textcolor{purple}{\textbf{DefaultFunctionReceive1\_EnvironmentChoice}}
	\newline
	\textcolor{purple}{\textbf{DefaultFunctionReceive2\_AutoReceiveEarliest}}
	\newline
	\textcolor{purple}{\textbf{DefaultFunctionSend}}
	& \textit{PERFORM(subj ;ComAct; state)}
	&  ASM-Rule specifying the execution of a Comunication act in an according state)
	\\
	\hline
	
\caption{Main Execution/Interpreting Rules}
\label{tab:places-Entities}
\end{longtable}
\end {landscape}


\section{Functions}

Functions return some element. They are activities that can be performed to determine something.
Dynamic functions can be considered as “variables” known from programming languages, they can be read and written.
Static functions are initialized before the execution, they can only be read.
Derived functions "evaluate” other functions, they can only be read. “They may be thought of as a global method with read-only variables” 

\begin{landscape}
	\begin {longtable} {| p{0.7\textwidth} | p{0.3\textwidth} | p{0.4\textwidth}|}
	\hline
	OWL Model element &   ASM interpreter & Description\\
	\toprule
	\endhead
	\hline
	
	Function that the return state should correspond to/be derived from one of the multiple \textcolor{orange}{\textbf{State }}in an SBD model
	& \textit{SID\_state(subj)} 
	&  Dynamic ASM-Function that stores the current state of a subj
	\\
	\hline
	
	\textcolor{orange}{\textbf{State }} \textcolor{blue}{\textbf{hasOutgoingTransition }} \textcolor{orange}{\textbf{Transition }}  \newline
	(input / worked on link  / output (Set of Transition)
	(linking State with )
	& \textit{OutEdge(state) \newline
		OutEdge(state;i)}
	& Function that returns the set of outgoing edges of a state or a single specific edge i
	\\
	\hline
	
	Object Property: \textcolor{blue}{\textbf{hasTargetState}}
	(linking \textcolor{orange}{\textbf{Transition}} and \textcolor{orange}{\textbf{State}} --->\newline
	 \textcolor{orange}{\textbf{Transition  }}  \textcolor{blue}{\textbf{hasTargetState }} \textcolor{orange}{\textbf{State}}
	& \textit{target(edge)/ \newline
	target(outEdge)} /
	&  Function that returns the follow up state of an outgoing transition (\textit{outEdge} is a special denomination for an \textit{edge}returned by the \textit{outEdge}-Function)
	\\
	\hline
	
	Object Property: \textcolor{blue}{\textbf{hasSourceState}}
	(linking \textcolor{orange}{\textbf{Transition}} and \textcolor{orange}{\textbf{State}}-->
	\textcolor{orange}{\textbf{Transition}} \textcolor{blue}{\textbf{hasSourceState}} \textcolor{orange}{\textbf{State}}
	(input / worked on link  / output)
	& \textit{source(edge)}
	&  Function that returns the source state of an edge
	\\
	\hline
	\multicolumn{3}{c|}{\textbf{Determine Follow up state Mechanic}}
%	\\
%	\hline
%	eee& ZZZZ & nnnn
	\\
	\hline
	Exit conditions in PASS are defined on their corresponding
	\newline 
	\textcolor{orange}{\textbf{Transitions}} and therefore are called
	\newline
	\textcolor{orange}{\textbf{TransitionCondition}}
	\newline 
	\textcolor{orange}{\textbf{Transitions}}  have  (\textcolor{blue}{\textbf{hasTransitionCondition}})
	(\textcolor{orange}{\textbf{State}} $\rightarrow$
	\textcolor{blue}{\textbf{hasOutgoingTransition}} $\rightarrow$ \textcolor{orange}{\textbf{Transition }}  $\rightarrow$ \textcolor{blue}{\textbf{hasTransitionCondition}} $\rightarrow$ \textcolor{orange}{\textbf{TransitionCondition}})
	& \textit{ExitCond(e)}
	\newline
	\textit{ExitCond(outEdge)}
	\newline
	\textit{ExitCond\_i(e)}
	\newline
	\textit{ExitCond(e)(subj,state)}
	&  Derived Function that evaluates the ExitCondition of a given edge/outgoing edge 
	\\
	\hline
	Execution Concept 
	& select\textsubscript{Edge}
	&  ASM Function that determines an edged (transition) to follow.
	\\
	\hline
	Execution Concept (connected to: \textcolor{orange}{\textbf{State}}, and \textcolor{orange}{\textbf{FunctionSpecification}})
	& \textit{completed(subj ; \newline service(state); state)}
	&  Function that returns true if the Service of a certain state is complete IF the subject is in that state
	\\
	\hline
	Execution Concept
	& 
	&  Rule/Function that gives that returns the service of function of a given state
	\\
	\hline

	\caption{Derived Functions}
	\label{tab:Derived-Functions}
\end{longtable}
\end {landscape}

\section{Extended Concepts  – Refinements for the Semantics of Core Actions}

\begin{landscape}
	\begin {longtable} {| p{0.7\textwidth} | p{0.3\textwidth} | p{0.4\textwidth}|}
	\hline
	OWL Model element &   ASM interpreter & Description\\
	\toprule
	\endhead
	\hline
	
	Function that the return state should correspond to/be derived from one of the multiple \textcolor{orange}{\textbf{State }}in an SBD model
	& \textit{SID\_state(subj)} 
	&  Dynamic ASM-Function that stores the current state of a subj
	\\
	\hline
	
	\textcolor{orange}{\textbf{State }} \textcolor{blue}{\textbf{hasOutgoingTransition }} \textcolor{orange}{\textbf{Transition }}  \newline
	(input / worked on link  / output (Set of Transition)
	(linking State with )
	& \textit{OutEdge(state) \newline
		OutEdge(state;i)}
	& Function that returns the set of outgoing edges of a state or a single specific edge i
	\\
	\hline
	\caption{Refinrments places}
\label{tab:Refinrments places}
\end{longtable}
\end {landscape}


\section{Input Pool Handling}


\begin{landscape}
	\begin {longtable} {| p{0.7\textwidth} | p{0.3\textwidth} | p{0.4\textwidth}|}
	\hline
	OWL Model element &   ASM interpreter & Description\\
	\toprule
	\endhead
	\hline
	
	Refers to a set of \textcolor{orange}{\textbf{InputPoolConstraints }} of \textcolor{orange}{\textbf{Subject  }}that has \textcolor{blue}{\textbf{hasInputPoolConstraints}} – for its Input Pool  
	& \textit{constraintTable(inputPool )} 
	&  Function that Returns the set of all input Pool constrains
	\\
	\hline
	
	Execution Concept with evalution relevance for: \textcolor{orange}{\textbf{MessageSenderTypeConstraint}} and
	\textcolor{orange}{\textbf{SenderTypeConstraint }}
	& \textit{sender/receiver}
	& Identifiers for possible subject instances trying to access an input pool
	\\
	\hline
	
		Refers to a set of \textcolor{orange}{\textbf{InputPoolConstraints }} of \textcolor{orange}{\textbf{Subject  }}that has \textcolor{blue}{\textbf{hasInputPoolConstraints}} – for its Input Pool  
	& \textit{msgType )} 
	&  Function that Returns the set of all input Pool constrains
	\\
	\hline
	
	Execution Concept
	& \textit{select \textsubscript{MsgKind(subj ;state;alt;i)}} 
	&  ASM Function that determines the message kind (“message type”) to be received in a given receive state.
	\\
	\hline
	
	\textcolor{orange}{\textbf{InputPoolContstraintHandlingStrategy  }}
	\newline
	And their individual defaultinstances:
	\newline
	\textcolor{purple}{\textbf{InputPoolConstraintStrategy-Blocking }}
	\newline
	\textcolor{purple}{\textbf{InputPoolConstraintStrategy-DeleteLatest  }}
	\newline
	\textcolor{purple}{\textbf{InputPoolConstraintStrategy-DeleteOldest }}
	\newline
	\textcolor{purple}{\textbf{InputPoolConstraintStrategy-Drop}}
	& \textit{/{Blocking; DropYoungest; DropOldest; DropIncoming/}} 
	&  Default Input Pool handling strategies for 
	\\
	\hline
	
	Execution Concept – can be restricted by  \textcolor{orange}{\textbf{InputPoolConstraint}} – for its Input Pool  
	& \textit{P / inputPool} 
	&  The actual Input Pool
	\\
	\hline
	
	
	synchronous communication
	& \multicolumn{2}{p{9 cm}|}{Definition for an input pool \textbf{constraint set to 0} requiring sender and receiver interpreter to be in the corresponding send and receive states at the same time in order to actually communicate (as messages cannot be passed to an input pool)}
	\\
	\hline
	
	\caption{Input Pool Handling}
	\label{tab:Input-Pool-Handling}
\end{longtable}
\end{landscape}

\section{Other Functions}
%
\begin{landscape}
	\begin {longtable} {| p{0.6\textwidth} | p{0.4\textwidth} | p{0.4\textwidth}|}
	\hline
	OWL Model element &   ASM interpreter & Description\\
	\toprule
	\endhead
	\hline
	
	Exit conditions in PASS are defined on their corresponding \textcolor{orange}{\textbf{Transitions }} and therefore are called \textcolor{orange}{\textbf{TransitionCondition}}. Execution Concept: can be set on.
	Execution Concept: used to determine the correct exit
	& \textit{NormalExitCond} 
	&  is used internally to “remember” that neither a timeout nor a user cancel have happened, so that the correct exit transition can be taken.
	\\
	\hline
	In the model to be interpreted the according aspects are captured by \textcolor{orange}{\textbf{TimerTransitions}} that have (\textcolor{blue}{\textbf{hasTransitionCondition}}) a  
	\textcolor{orange}{\textbf{TimerTransitionCondition}} containing the date. The timeout(state) function should read the information. 
	&\multicolumn{2}{|p{11 cm}|}{\textbf{Timer/Timeout Mechanic:} The evaluation and handling of timeouts is defined (and refined) with several rules and functions.\textit{OutEdge(timeout(state), Timeout(subj , state, timeout(state)), SetTimeoutClock(subj ; state)}  are used to evaluate the timeout condition, \textit{OutEdge(Interrupt\_service(state)(subj , state)}  is used to define how the corresponding service should be canceled. \textit{OutEdge(TimeoutExitCond)}  is used internally to “remember” that a timeout happened, so that the correct exit transition can be taken.}
	

	\\
	\hline
	In PASS models the possibility to arbitrarily cancel the execution of a (receive) function and the possible course of action afterwards may be discerned via a \textcolor{orange}{\textbf{UserCancelTransitions }}
	&\multicolumn{2}{|p{11 cm}|}{\textbf{User Cancel/Abrupt Mechanic:} The evaluation and handling of user cancels is defined (and refined) with several rules and functions. \textit{UserAbruption(subj, state)} is used to evaluate the user decision, \textit{Abrupt\_service(state)(subj , state)} is used to define how the corresponding service should be abrupted. \textit{AbruptionExitCond} is used internally to “remember” that a user cancel happened, so that the correct exit transition can be taken.}
	\\
	\hline
	With the definition of the data properties \textcolor{green}{\textbf{hasMaximumSubjectInstanceRestriction}}
	The \textcolor{orange}{\textbf{}MultiSubject } are actually the standard and \textcolor{orange}{\textbf{SingleSubject}} the special case
	& \textit{MultiRound / mult(alt) / InitializeMultiRoun /
		ContinueMultiRoundSuccess 
		(among others}
	& Definition of Functions and ASM rules for interaction between multiple Subjects at once
		\\
	\hline
	
	Handling of \textcolor{orange}{\textbf{ChoiceSegment }} \& \textcolor{orange}{\textbf{	ChoiceSegmentPath}}

	 \textcolor{blue}{\textbf{hasOutgoingTransition }} \textcolor{orange}{\textbf{Transition }}  \newline
	(input / worked on link  / output (Set of Transition)
	(linking State with )
	& \textit{AltAction /
		altEntry(D) / altExit(D)
		AltBehDgm(altSplit)
		altJoin(altSplit)}
	& Rules for the semantics/handling of ChoiceSegements 
		\\
	\hline
	\textcolor{orange}{\textbf{State }} \textcolor{blue}{\textbf{hasOutgoingTransition }} \textcolor{orange}{\textbf{Transition }}  \newline
	(input / worked on link  / output (Set of Transition)
	(linking State with )
	& \textit{Compulsory(altEntry(D))} and textit{Compulsory(altExit(D))}
	& 
	\\
	\hline
	\caption{Other Functions}
	\label{tab:Other-Functions}
\end{longtable}
\end{landscape}
%
\section{Elements Not Covered not by Börger (directly)}
\begin{landscape}
	\begin {longtable} {| p{0.5\textwidth} | p{0.9\textwidth} | }
	\hline
	OWL Model element &    Description\\
	\toprule
	\endhead
	\hline
\textcolor{orange}{\textbf{ReminderTransition / ReminderEventTransitionCondition  }}
&This type time-logic-based transitions did not exist when the original ASM interpreter was conceived. They were added to PASS for the OWL Standard. They can be handled by assuming the existence of an implicit calendar subject that sends an interrupt message (reminder) upon a time condition (e.g. reaching of a calendarial date) has been achieved.
(includes the specialized (\textcolor{orange}{\textbf{CalendarBasedReminderTransition, TimeBasedReminderTransition}}

\\

	\hline
\textcolor{orange}{\textbf{DataDescribingComponent / DataMappingFunction }}
&The PASS OWL standard envisions the integration and usage of classic data element (Data Objects) as part of a process model. The Börger Interpreter does not assume the existence of such data elements as part of the model. However, the refinement concept of ASMs could easily been used to integrate according interpretation aspects.
(Includes Elements such as \textcolor{orange}{\textbf{PayloadDescription }} for Messages or \textcolor{orange}{\textbf{DataMappingFunction }}

\\
	\hline
\caption{Other Functions}
\label{tab:In ASM not covered}
\end{longtable}
\end{landscape}
%\textcolor{orange}{\textbf{Objekte}}\\
%\textcolor{blue}{Object properties}
%\textcolor{purple}{\textbf{constraint}}
%\textcolor{green}{data properties}


\chapter{Mapping Ontology to Abstract State Machine}


%appendix in smaller font size
\footnotesize

\section{Conceptual Differences ASM Semantic / OWL Model}

% Some content might move to Appendix B / replace it.

This implementation provides more language elements than covered by the OWL description and also gives some concrete implementations.

Additionally to OWL:
\begin{itemize}
    \item Internal Behavior: usage of Subject Data: DataModificationFunction (VarMan: Selection, Concatenation, \ldots), DataMappingFunction (StoreMessage, UseMessageContent, UseCorrelationID).
    Subject Data can be scoped to Macro Instance.
    \item Interaction: Subject Restart, Inputpool Functions (CloseIP, OpenIP, \ldots), CorrelationID, Mobility of Channel.
\end{itemize}

There are also some conceptional differences and limitations:
\begin{itemize}
    \item only blocking asynchronous send, i.e. IP strategy blocking - no delete / drop. no sync send.
    \item modal split / join instead of ChoiceSegement. only mandatory to start and end / no optional start or optional end.
    \item timeout transitions: designed for interactive validation -> duration in seconds, no business days. also no reminders.
    \item Observer: different approach: no native support; possible via Modal Split + Receive State with State Priority + following Cancel Function
\end{itemize}



\newpage
\section{Actual Appendix}

\begin{listing}[H]
\begin{minted}[fontsize=\small]{lexer.py:CoreASMLexer -x}
function channelFor : Agents -> LIST

derived processIDFor(a)       = processIDOf(channelFor(a))
derived processInstanceFor(a) = processInstanceOf(channelFor(a))
derived subjectIDFor(a)       = subjectIDOf(channelFor(a))
derived agentFor(a)           = agentOf(channelFor(a))

derived processIDOf(ch)       = nth(ch, 1)
derived processInstanceOf(ch) = nth(ch, 2)
derived subjectIDOf(ch)       = nth(ch, 3)
derived agentOf(ch)           = nth(ch, 4)
\end{minted}
\caption{channelFor}
\label{lst:asm:channelFor}
\end{listing}


\begin{listing}[H]
\begin{minted}[fontsize=\small]{lexer.py:CoreASMLexer -x}
// Channel -> List[List[MI, StateNumber]]
function killStates : LIST -> LIST

// Channel * macroInstanceNumber -> List[StateNumber]
function activeStates : LIST * NUMBER -> LIST
\end{minted}
\caption{activeStates}
\label{lst:asm:activeStates}
\end{listing}



\begin{listing}[H]
\begin{minted}[fontsize=\small]{lexer.py:CoreASMLexer -x}
// Channel * MacroInstanceNumber * StateNumber -> BOOLEAN
function initializedState : LIST * NUMBER * NUMBER -> BOOLEAN

// Channel * MacroInstanceNumber * StateNumber -> BOOLEAN
function completed : LIST * NUMBER * NUMBER -> BOOLEAN

// Channel * MacroInstanceNumber * StateNumber
function timeoutActive  : LIST * NUMBER * NUMBER -> BOOLEAN
function cancelDecision : LIST * NUMBER * NUMBER -> BOOLEAN

// Channel * MacroInstanceNumber * StateNumber -> BOOLEAN
function abortionCompleted : LIST * NUMBER * NUMBER -> BOOLEAN

// Channel * MacroInstanceNumber * StateNumber -> ..
function selectedTransition : LIST * NUMBER * NUMBER -> NUMBER // -> TransitionNumber
function initializedSelectedTransition : LIST * NUMBER * NUMBER -> BOOLEAN
function startTime  : LIST * NUMBER * NUMBER -> NUMBER

// can exit
// Channel * MacroInstanceNumber * TransitionNumber -> BOOLEAN
function exitCondition : LIST * NUMBER * NUMBER -> BOOLEAN
// TODO: may rename, possibly transitionEnabled ???

// Channel * MacroInstanceNumber * TransitionNumber -> BOOLEAN
function transitionCompleted : LIST * NUMBER * NUMBER -> BOOLEAN
\end{minted}
\caption{initializedState}
\label{lst:asm:initializedState}
\end{listing}



\begin{listing}[H]
\begin{minted}[fontsize=\small]{lexer.py:CoreASMLexer -x}
derived shouldTimeout(ch, MI, stateNumber) = return boolres in {
  let processID = processIDOf(ch) in {
    if (hasTimeoutTransition(processID, stateNumber) = true and startTime(ch, MI, stateNumber) != undef) then {
      let transitionNumber = first_outgoingTimeoutTransition(processID, stateNumber) in
      let timeout = transitionTimeout(processID, transitionNumber) * 1000 * 1000 * 1000 in {
        let runningTime = (nanoTime() - startTime(ch, MI, stateNumber)) in
          boolres := (runningTime > timeout)
      }
    }
    else {
      boolres := false
    }
  }
}
\end{minted}
\caption{shouldTimeout}
\label{lst:asm:shouldTimeout}
\end{listing}


\begin{listing}[H]
\begin{minted}[fontsize=\small]{lexer.py:CoreASMLexer -x}
// Channel * macroInstanceNumber * varname -> [vartype, content]
function variable : LIST * NUMBER * STRING -> LIST

// Channel -> Set[(macroInstanceNumber, varname)]
function variableDefined : LIST -> SET
\end{minted}
\caption{variable}
\label{lst:asm:variable}
\end{listing}


\begin{listing}[H]
\begin{minted}[fontsize=\small]{lexer.py:CoreASMLexer -x}
// receiverChannel * senderSubjID * messageType * correlationID -> [msg1, msg2, ...]
function inputPool : LIST * STRING * STRING * NUMBER -> LIST

/* store all locations where an inputPool was defined for to allow IPEmpty and receiving from "?" */
// receiverChannel -> {[senderSubjID, messageType, correlationID], ..}
function inputPoolDefined : LIST -> SET
\end{minted}
\caption{inputPool}
\label{lst:asm:inputPool}
\end{listing}



\begin{listing}[H]
\begin{minted}[fontsize=\small]{lexer.py:CoreASMLexer -x}
// Channel * MacroInstanceNumber * StateNumber -> Set[Messages]
function receivedMessages : LIST * NUMBER * NUMBER -> SET

// Channel * MacroInstanceNumber * StateNumber -> Set[Channel]
function receivers : LIST * NUMBER * NUMBER -> SET

// Channel * MacroInstanceNumber * StateNumber -> STRING
function messageContent : LIST * NUMBER * NUMBER -> LIST

// Channel * MacroInstanceNumber * StateNumber -> Set[Channel]
function reservationsDone : LIST * NUMBER * NUMBER -> SET
\end{minted}
\caption{receivedMessages}
\label{lst:asm:receivedMessages}
\end{listing}




\begin{listing}[H]
\begin{minted}[fontsize=\small]{lexer.py:CoreASMLexer -x}
// Channel * macroInstanceNumber -> result
function macroTerminationResult : LIST * NUMBER -> ELEMENT

// Channel * macroInstanceNumber -> MacroNumber
function macroNumberOfMI : LIST * NUMBER -> NUMBER

// Channel * macroInstanceNumber * StateNumber -> MacroInstance
function callMacroChildInstance : LIST * NUMBER * NUMBER -> NUMBER
\end{minted}
\caption{macroTerminationResult}
\label{lst:asm:macroTerminationResult}
\end{listing}




\begin{listing}[H]
\begin{minted}[fontsize=\small]{lexer.py:CoreASMLexer -x}
// called form PerformEnd (iff other states are active) and AbortCallMacro
rule AbortMacroInstance(MIAbort, ignoreState) = {
  foreach currentState in activeStates(channelFor(self), MIAbort) do {
    add [MIAbort, currentState] to killStates(channelFor(self))
  }

  ClearAllVarInMIForChannel(channelFor(self), MIAbort)
}
\end{minted}
\caption{AbortMacroInstance}
\label{lst:asm:AbortMacroInstance}
\end{listing}




\begin{listing}[H]
\begin{minted}[fontsize=\small]{lexer.py:CoreASMLexer -x}
rule StartASMAgent(ch) = {
  extend Agents with a do seqblock
    channelFor(a) := ch

    add a to asmAgents

    program(a) := @StartMainMacro
  endseqblock
}
\end{minted}
\caption{StartASMAgent}
\label{lst:asm:StartASMAgent}
\end{listing}


\begin{listing}[H]
\begin{minted}[fontsize=\small,escapeinside=~~]{lexer.py:CoreASMLexer -x}
rule StartMainMacro = {
  let ch = channelFor(self) in {
    killStates(ch) := []

    variableDefined(ch) := {[0, "$self"], [0, "$empty"]}
    variable(ch, 0, "$self") := ["ChannelInformation", {ch}]
    variable(ch, 0, "$empty") := ["Text", ""]
  }

  let mID  = subjectMainMacro(processIDFor(self), subjectIDFor(self)) in
  let startState = macroStartState(processIDFor(self), mID) in
  let MI   = 1 in // 0 reserved for top-level variable manipulation; 1 mainmacro
  {
    macroNumberOfMI(channelFor(self), MI) := mID

    nextMacroInstanceNumber(channelFor(self)) := MI  + 1

    activeStates(channelFor(self), MI) := [startState]
  }

  program(self) := @SubjectBehaviour
}
\end{minted}
\caption{StartMainMacro}
\label{lst:asm:StartMainMacro}
\end{listing}




\begin{listing}[H]
\begin{minted}[fontsize=\small]{lexer.py:CoreASMLexer -x}
rule StartMacro(MI, currentStateNumber, mIDNew, MINew) = {
  let processID = processIDFor(self) in
  let startState = macroStartState(processID, mIDNew) in {
    activeStates(channelFor(self), MINew) := []
    AddState(MI, currentStateNumber, MINew, startState)
  }
}
\end{minted}
\caption{StartMacro}
\label{lst:asm:StartMacro}
\end{listing}




\begin{listing}[H]
\begin{minted}[fontsize=\small]{lexer.py:CoreASMLexer -x}
/*
0 - REPEAT
  -* Behaviour should be executed again for this state
  -* results of previous execution will be merged with following execution in one global ASM step
  -* no other states are allowed to be executed
1 - DONE
  -* no other active states are allowed to be executed
  -* new states are started
  -* global ASM / LTS step should be done
2 - NEXT
  -* nothing changed / waiting for input
  -* other active states with the same priority can be executed
  -* active states with lower priority can not be executed
3 - LOWER
  -* nothing changed / waiting for input
  -* other active states, even with lower priority, can be executed
*/

// ch * MacroInstanceNumber * stateNumber => Int
function executionState : LIST * NUMBER * NUMBER -> NUMBER


/*
1 - DONE
2 - NEXT
3 - LOWER
*/
// ch * MacroInstanceNumber => Int
function macroExecutionState : LIST * NUMBER -> NUMBER

// ch * MacroInstanceNumber * stateNumber => List[(MI, s)]
function addStates : LIST * NUMBER * NUMBER -> LIST

// ch * MacroInstanceNumber * stateNumber => List[(MI, s)]
function removeStates : LIST * NUMBER * NUMBER -> LIST
\end{minted}
\caption{SetExecutionState}
\label{lst:asm:SetExecutionState}
\end{listing}




\begin{listing}[H]
\begin{minted}[fontsize=\small]{lexer.py:CoreASMLexer -x}
rule AddState(MI, currentStateNumber, MINew, sNew) = {
  add [MINew, sNew] to addStates(channelFor(self), MI, currentStateNumber)
}

rule RemoveState(MI, currentStateNumber, MIOld, sOld) = {
  add [MIOld, sOld] to removeStates(channelFor(self), MI, currentStateNumber)
}
\end{minted}
\caption{AddState}
\label{lst:asm:AddState}
\end{listing}







\begin{listing}[H]
\begin{minted}[fontsize=\small]{lexer.py:CoreASMLexer -x}
rule SubjectBehaviour = {
  choose x in killStates(channelFor(self)) do {
    KillBehaviour(nth(x, 1), nth(x, 2))
  }
  ifnone seqblock
    MacroBehaviour(1)

    // reset
    macroExecutionState(channelFor(self), 1) := undef
  endseqblock
}
\end{minted}
\caption{SubjectBehaviour}
\label{lst:asm:SubjectBehaviour}
\end{listing}




\begin{listing}[H]
\begin{minted}[fontsize=\small]{lexer.py:CoreASMLexer -x}
rule KillBehaviour(MI, currentStateNumber) = {
  if (initializedState(channelFor(self), MI, currentStateNumber) != true) then {
    remove [MI, currentStateNumber] from killStates(channelFor(self))
    remove n from activeStates(channelFor(self), MI)
  }
  else seqblock
    executionState(channelFor(self), MI, currentStateNumber) := undef
    addStates(channelFor(self), MI, currentStateNumber)       := []
    removeStates(channelFor(self), MI, currentStateNumber)    := []

    if (abortionCompleted(channelFor(self), MI, currentStateNumber) != true) then {
      Abort(MI, currentStateNumber)
    }
    else {
      RemoveState(MI, currentStateNumber, MI, currentStateNumber)
      SetExecutionState(MI, currentStateNumber, 1)
    }

    if (executionState(channelFor(self), MI, currentStateNumber) != 1) then {
      Crash()
    }
    else if (|addStates(channelFor(self), MI, currentStateNumber)| > 0) then {
      Crash()
    }
    else if (|removeStates(channelFor(self), MI, currentStateNumber)| > 0) then {
      if (removeStates(channelFor(self), MI, currentStateNumber) != [[MI, currentStateNumber]]) then {
        Crash()
      }
      else {
        foreach x in removeStates(channelFor(self), MI, currentStateNumber) do {
          let xMI = nth(x, 1),
            xN  = nth(x, 2) in {
            remove [xMI, xN] from killStates(channelFor(self))
            ResetState(xMI, xN)
            remove xN from activeStates(channelFor(self), xMI)
          }
        }
      }
    }
  endseqblock
}
\end{minted}
\caption{KillBehaviour}
\label{lst:asm:KillBehaviour}
\end{listing}




\begin{listing}[H]
\begin{minted}[fontsize=\small]{lexer.py:CoreASMLexer -x}
rule MacroBehaviour(MI) = {
  let processID = processIDFor(self) in
  local listres := activeStates(channelFor(self), MI) in seqblock // remaining states
    macroExecutionState(channelFor(self), MI) := undef

    // can not be done with foreach as listres is modified depending on the executionState and the priorities of the other states
    while (|listres| > 0) do
    let stateNumber = getAnyStateWithHighestPrio(processID, listres) in {
      seqblock
        executionState(channelFor(self), MI, stateNumber) := undef
        addStates(channelFor(self), MI, stateNumber)      := []
        removeStates(channelFor(self), MI, stateNumber)   := []

        Behaviour(MI, stateNumber)

        // WARNING: mutates listres!
        let state = executionState(channelFor(self), MI, stateNumber) in
          UpdateRemainingStates(MI, stateNumber, state, listres)

        UpdateActiveStates(MI, stateNumber)
      endseqblock
    }
  endseqblock
}
\end{minted}
\caption{MacroBehaviour}
\label{lst:asm:MacroBehaviour}
\end{listing}




\begin{listing}[H]
\begin{minted}[fontsize=\small]{lexer.py:CoreASMLexer -x}
// WARNING: mutates listres!
rule UpdateRemainingStates(MI, stateNumber, exState, remainingStates) = {
  if (exState = 0) then { // REPEAT
    remainingStates := [stateNumber]

    macroExecutionState(channelFor(self), MI) := 1 // DONE - something changed and nothing else should happen

    // quasi-reset
    // 2019-02-13: why? will happen in next loop iteration anyway
    executionState(channelFor(self), MI, stateNumber) := undef
  }
  else if (exState = 1) then { // DONE
    seq
      // end loop ...
      remainingStates := []
    next
      // ... but add new states of this MI to initialize them,
      // so that all states have the same start time
      // TODO: theoretically it should be possible to initialize states from a different MI as well. Why not?
      foreach x in addStates(channelFor(self), MI, stateNumber)
        with nth(x, 1) = MI do {
        add nth(x, 2) to remainingStates
      }

    macroExecutionState(channelFor(self), MI) := 1 // DONE - something changed and nothing else should happen

    // quasi-reset
    // 2019-02-13: why? I guess for UI?
    executionState(channelFor(self), MI, stateNumber) := 2
  }
  else if (exState = 2) then { // NEXT
    seqblock
      remove stateNumber from remainingStates // remove self

      remainingStates := filterStatesWithSamePrio(processIDFor(self), remainingStates, statePriority(processIDFor(self), stateNumber)) // reduce to states with same priority

      if (macroExecutionState(channelFor(self), MI) != 1) then {
        macroExecutionState(channelFor(self), MI) := 2 // NEXT - nothing changed but block lower; unless already DONE
      }
    endseqblock
  }
  else if (exState = 3) then { // LOWER
    remove stateNumber from remainingStates // remove self

    if (macroExecutionState(channelFor(self), MI) != 1 and macroExecutionState(channelFor(self), MI) != 2) then {
      macroExecutionState(channelFor(self), MI) := 3 // LOWER - nothing changed and lower not blocked; unless already DONE or NEXT
    }
  }
}
\end{minted}
\caption{UpdateRemainingStates}
\label{lst:asm:UpdateRemainingStates}
\end{listing}




\begin{listing}[H]
\begin{minted}[fontsize=\small]{lexer.py:CoreASMLexer -x}
rule UpdateActiveStates(MI, stateNumber) = seqblock
  // NOTE: everything needs to be sequential, as activeStates is a list and not a set

  foreach x in addStates(channelFor(self), MI, stateNumber) do {
    let xMI = nth(x, 1),
      xN  = nth(x, 2) in {
      add xN to activeStates(channelFor(self), xMI)
    }
  }

  // NOTE: reset only needed for verification as optimization -> no need to carry over to next step / store globally
  addStates(channelFor(self), MI, stateNumber) := undef

  foreach x in removeStates(channelFor(self), MI, stateNumber) do {
    let xMI = nth(x, 1),
      xN  = nth(x, 2) in {
      // remove one instance of xN, if any
      remove xN from activeStates(channelFor(self), xMI)

      // NOTE: reset only needed for verification as optimization
      // FIXME 2019-02-15: what happens if a state is multiple times in activeStates - this would reset the other ones as well?!
      ResetState(xMI, xN)
    }
  }

  // NOTE: reset only needed for verification as optimization -> no need to carry over to next step / store globally
  removeStates(channelFor(self), MI, stateNumber) := undef
endseqblock
\end{minted}
\caption{UpdateActiveStates}
\label{lst:asm:UpdateActiveStates}
\end{listing}




\begin{listing}[H]
\begin{minted}[fontsize=\small]{lexer.py:CoreASMLexer -x}
// whether to abort the state or not
derived abortState(MI, stateNumber) = return boolres in {
  boolres := ((timeoutActive(channelFor(self), MI, stateNumber) = true) or (cancelDecision(channelFor(self), MI, stateNumber) = true))
}
\end{minted}
\caption{abortState}
\label{lst:asm:abortState}
\end{listing}




\begin{listing}[H]
\begin{minted}[fontsize=\small]{lexer.py:CoreASMLexer -x}
rule Behaviour(MI, currentStateNumber) = {
  let processID = processIDFor(self) in {
    if (initializedState(channelFor(self), MI, currentStateNumber) != true) then {
      StartState(MI, currentStateNumber)
    }
    else if (abortState(MI, currentStateNumber) = true) then { // -> derived from timeoutActive / cancelDecision
      AbortState(MI, currentStateNumber)
    }
    else if (completed(channelFor(self), MI, currentStateNumber) != true) then {
      Perform(MI, currentStateNumber)
    }
    else if (initializedSelectedTransition(channelFor(self), MI, currentStateNumber) != true) then {
      StartSelectedTransition(MI, currentStateNumber)
    }
    else {
      let transitionNumber = selectedTransition(channelFor(self), MI, currentStateNumber) in
      let t = targetStateNumber(processID, transitionNumber) in {
        if (transitionCompleted(channelFor(self), MI, transitionNumber) != true) then {
          PerformTransition(MI, currentStateNumber, transitionNumber)
        }
        else {
          Proceed(MI, currentStateNumber, t)

          SetExecutionState(MI, currentStateNumber, 1) // DONE, make global update
        }
      }
    }
  }
}
\end{minted}
\caption{Behaviour}
\label{lst:asm:Behaviour}
\end{listing}




\begin{listing}[H]
\begin{minted}[fontsize=\small]{lexer.py:CoreASMLexer -x}
rule AbortState(MI, currentStateNumber) = {
  if (abortionCompleted(channelFor(self), MI, currentStateNumber) != true) then {
    Abort(MI, currentStateNumber)
  }
  else {
    if (cancelDecision(channelFor(self), MI, currentStateNumber) = true) then {
      let transitionNumber = first_outgoingCancelTransition(processIDFor(self), currentStateNumber) in
      let t = targetStateNumber(processIDFor(self), transitionNumber) in {
        Proceed(MI, currentStateNumber, t)
      }
    }
    else if (timeoutActive(channelFor(self), MI, currentStateNumber) = true) then {
      let transitionNumber = first_outgoingTimeoutTransition(processIDFor(self), currentStateNumber) in
      let t = targetStateNumber(processIDFor(self), transitionNumber) in {
        Proceed(MI, currentStateNumber, t)
      }
    }

    SetExecutionState(MI, currentStateNumber, 1) // DONE, make global update
  }
}
\end{minted}
\caption{AbortState}
\label{lst:asm:AbortState}
\end{listing}




\begin{listing}[H]
\begin{minted}[fontsize=\small]{lexer.py:CoreASMLexer -x}
rule StartState(MI, currentStateNumber) = {
  let processID = processIDFor(self) in
  let sType     = stateType(processID, currentStateNumber) in
  seqblock
    InitializeCompletion(MI, currentStateNumber)
    abortionCompleted(channelFor(self), MI, currentStateNumber) := false

    ResetTimeout(MI, currentStateNumber)
    cancelDecision(channelFor(self), MI, currentStateNumber) := false

    DisableAllTransitions(MI, currentStateNumber)
    initializedSelectedTransition(channelFor(self), MI, currentStateNumber) := false

    wantInput(channelFor(self), MI, currentStateNumber) := {}

    case sType of
      "function"       : StartFunction(MI, currentStateNumber)
      "internalAction" : StartInternalAction(MI, currentStateNumber)
      "send"           : StartSend(MI, currentStateNumber)
      "receive"        : SetExecutionState(MI, currentStateNumber, 3) // NEXT, nothing to do. Handle all other startStates, LTS step implied afterwards
      "end"            : StartEnd(MI, currentStateNumber)
    endcase

    initializedState(channelFor(self), MI, currentStateNumber) := true
  endseqblock
}
\end{minted}
\caption{StartState}
\label{lst:asm:StartState}
\end{listing}




\begin{listing}[H]
\begin{minted}[fontsize=\small]{lexer.py:CoreASMLexer -x}
rule ResetState(MI, stateNumber) = {
  executionState(channelFor(self), MI, stateNumber) := undef

  // StartState

  initializedState(channelFor(self), MI, stateNumber) := undef

  completed(channelFor(self), MI, stateNumber) := undef
  abortionCompleted(channelFor(self), MI, stateNumber) := undef

  startTime(channelFor(self), MI, stateNumber) := undef
  timeoutActive(channelFor(self), MI, stateNumber) := undef

  cancelDecision(channelFor(self), MI, stateNumber) := undef

  selectedTransition(channelFor(self), MI, stateNumber) := undef

  forall transitionNumber in outgoingNormalTransitions(processIDFor(self), stateNumber) do {
    exitCondition(channelFor(self), MI, transitionNumber) := undef
    transitionCompleted(channelFor(self), MI, transitionNumber) := undef
  }

  initializedSelectedTransition(channelFor(self), MI, stateNumber) := undef

  wantInput(channelFor(self), MI, stateNumber) := undef

  // StartSend
  receivers(channelFor(self), MI, stateNumber) := undef
  reservationsDone(channelFor(self), MI, stateNumber) := undef
  messageContent(channelFor(self), MI, stateNumber) := undef
}
\end{minted}
\caption{ResetState}
\label{lst:asm:ResetState}
\end{listing}




\begin{listing}[H]
\begin{minted}[fontsize=\small]{lexer.py:CoreASMLexer -x}
rule Perform(MI, currentStateNumber) = {
  let processID = processIDFor(self) in
  let sType     = stateType(processID, currentStateNumber) in {
    case sType of
      "function"       : PerformFunction(MI, currentStateNumber)
      "internalAction" : PerformInternalAction(MI, currentStateNumber)
      "send"           : PerformSend(MI, currentStateNumber)
      "receive"        : PerformReceive(MI, currentStateNumber)
      "end"            : PerformEnd(MI, currentStateNumber)
    endcase
  }
}
\end{minted}
\caption{Perform}
\label{lst:asm:Perform}
\end{listing}




\begin{listing}[H]
\begin{minted}[fontsize=\small]{lexer.py:CoreASMLexer -x}
rule StartSelectedTransition(MI, currentStateNumber) = {
  let transitionNumber = selectedTransition(channelFor(self), MI, currentStateNumber) in {
    InitializeCompletionTransition(MI, transitionNumber)
    initializedSelectedTransition(channelFor(self), MI, currentStateNumber) := true
  }

  SetExecutionState(MI, currentStateNumber, 0)
}
\end{minted}
\caption{StartSelectedTransition}
\label{lst:asm:StartSelectedTransition}
\end{listing}






\begin{listing}[H]
\begin{minted}[fontsize=\small]{lexer.py:CoreASMLexer -x}
rule Proceed(MI, s_from, s_to) = {
  AddState(MI, s_from, MI, s_to)
  RemoveState(MI, s_from, MI, s_from)
}
\end{minted}
\caption{Proceed}
\label{lst:asm:Proceed}
\end{listing}




\begin{listing}[H]
\begin{minted}[fontsize=\small]{lexer.py:CoreASMLexer -x}
rule StartTimeout(MI, currentStateNumber) = {
  startTime(channelFor(self), MI, currentStateNumber) := nanoTime()
  timeoutActive(channelFor(self), MI, currentStateNumber) := false
}

rule ResetTimeout(MI, currentStateNumber) = {
  startTime(channelFor(self), MI, currentStateNumber) := undef
  timeoutActive(channelFor(self), MI, currentStateNumber) := undef
}

rule ActivateTimeout(MI, currentStateNumber) = {
  timeoutActive(channelFor(self), MI, currentStateNumber) := true
}

rule InitializeCompletion(MI, currentStateNumber) = {
  completed(channelFor(self), MI, currentStateNumber) := false
}

rule SetCompleted(MI, currentStateNumber) = {
  SetExecutionState(MI, currentStateNumber, 0)
  completed(channelFor(self), MI, currentStateNumber) := true
}

rule SetAbortionCompleted(MI, currentStateNumber) = {
  SetExecutionState(MI, currentStateNumber, 1) // DONE, make global update
  abortionCompleted(channelFor(self), MI, currentStateNumber) := true
}


rule EnableTransition(MI, transitionNumber) = {
  exitCondition(channelFor(self), MI, transitionNumber) := true
}

rule EnableAllTransitions(MI, currentStateNumber) = {
  forall transitionNumber in outgoingNormalTransitions(processIDFor(self), currentStateNumber) do {
    EnableTransition(MI, transitionNumber)
  }
}


rule DisableTransition(MI, currentStateNumber, transitionNumber) = {
  exitCondition(channelFor(self), MI, transitionNumber) := false
}

rule DisableAllTransitions(MI, currentStateNumber) = {
  selectedTransition(channelFor(self), MI, currentStateNumber) := undef

  forall transitionNumber in outgoingNormalTransitions(processIDFor(self), currentStateNumber) do {
    DisableTransition(MI, currentStateNumber, transitionNumber)
  }
}

rule InitializeCompletionTransition(MI, transitionNumber) = {
  transitionCompleted(channelFor(self), MI, transitionNumber) := false
}


rule SetCompletedTransition(MI, currentStateNumber, transitionNumber) = {
  SetExecutionState(MI, currentStateNumber, 0)

  transitionCompleted(channelFor(self), MI, transitionNumber) := true
}
\end{minted}
\caption{StartTimeout}
\label{lst:asm:StartTimeout}
\end{listing}





\begin{listing}[H]
\begin{minted}[fontsize=\small]{lexer.py:CoreASMLexer -x}
rule StartInternalAction(MI, currentStateNumber) = {
  let processID = processIDFor(self) in {
    StartTimeout(MI, currentStateNumber)

    EnableAllTransitions(MI, currentStateNumber)

    SetExecutionState(MI, currentStateNumber, 3) // handle all other startStates, LTS step afterwards
  }
}
\end{minted}
\caption{StartInternalAction}
\label{lst:asm:StartInternalAction}
\end{listing}




\begin{listing}[H]
\begin{minted}[fontsize=\small]{lexer.py:CoreASMLexer -x}
rule PerformInternalAction(MI, currentStateNumber) = {
  if (shouldTimeout(channelFor(self), MI, currentStateNumber) = true) then {
    SetCompleted(MI, currentStateNumber) // sets executionState to REPEAT
    ActivateTimeout(MI, currentStateNumber)
  }
  else {
    if (selectedTransition(channelFor(self), MI, currentStateNumber) != undef) then {
      SetCompleted(MI, currentStateNumber) // sets executionState to REPEAT
    }
    else {
      SelectTransition(MI, currentStateNumber)
    }
  }
}
\end{minted}
\caption{PerformInternalAction}
\label{lst:asm:PerformInternalAction}
\end{listing}






\begin{listing}[H]
\begin{minted}[fontsize=\small]{lexer.py:CoreASMLexer -x}
rule StartFunction(MI, currentStateNumber) = {
  StartTimeout(MI, currentStateNumber)

  let processID = processIDFor(self) in
  let actionName = stateFunction(processID, currentStateNumber) in {
    if (startFunction(actionName) = undef) then {
      skip
    }
    else {
      call startFunction(actionName) (MI, currentStateNumber)
    }
  }

  SetExecutionState(MI, currentStateNumber, 3) // handle all other startStates, LTS step afterwards
}
\end{minted}
\caption{StartFunction}
\label{lst:asm:StartFunction}
\end{listing}




\begin{listing}[H]
\begin{minted}[fontsize=\small]{lexer.py:CoreASMLexer -x}
rule PerformFunction(MI, currentStateNumber) = {
  if (shouldTimeout(channelFor(self), MI, currentStateNumber) = true) then {
    SetCompleted(MI, currentStateNumber) // sets executionState to REPEAT
    ActivateTimeout(MI, currentStateNumber)
  }
  else {
    let processID = processIDFor(self) in
    let actionName = stateFunction(processID, currentStateNumber),
      args       = stateFunctionArguments(processID, currentStateNumber) in
        call performFunction(actionName) (MI, currentStateNumber, args)
  }
}
\end{minted}
\caption{PerformFunction}
\label{lst:asm:PerformFunction}
\end{listing}




\begin{listing}[H]
\begin{minted}[fontsize=\small]{lexer.py:CoreASMLexer -x}
rule AbortFunction(MI, currentStateNumber) = {
  let processID = processIDFor(self) in
  let functionName = stateFunction(processID, currentStateNumber) in {
    if (abortFunction(functionName) = undef) then {
      SetAbortionCompleted(MI, currentStateNumber)  // sets executionState to DONE
    }
    else {
      call abortFunction(functionName) (MI, currentStateNumber) // must set abortionCompleted eventually
    }
  }
}
\end{minted}
\caption{AbortFunction}
\label{lst:asm:AbortFunction}
\end{listing}




\begin{listing}[H]
\begin{minted}[fontsize=\small]{lexer.py:CoreASMLexer -x}
rule PerformTransitionFunction(MI, currentStateNumber, transitionNumber) = {
  let processID = processIDFor(self) in
  let functionName = stateFunction(processID, currentStateNumber) in {
    if (performTransitionFunction(functionName) = undef) then {
      SetCompletedTransition(MI, currentStateNumber, transitionNumber) // sets executionState to REPEAT
    }
    else {
      call performTransitionFunction(functionName) (MI, currentStateNumber, transitionNumber) // must set executionState
    }
  }
}
\end{minted}
\caption{PerformTransitionFunction}
\label{lst:asm:PerformTransitionFunction}
\end{listing}




\begin{listing}[H]
\begin{minted}[fontsize=\small]{lexer.py:CoreASMLexer -x}
rule SetCompletedAction(MI, currentStateNumber, res) = {
  let processID = processIDFor(self) in {
    if (res = undef) then {
      choose transitionNumber in outgoingNormalTransitions(processID, currentStateNumber) do {
        selectedTransition(channelFor(self), MI, currentStateNumber) := transitionNumber
      }
    }
    else {
      let transitionNumber = getTransitionByLabel(processID, currentStateNumber, res) in
        selectedTransition(channelFor(self), MI, currentStateNumber) := transitionNumber
    }
  }

  SetCompleted(MI, currentStateNumber) // sets executionState to REPEAT
}
\end{minted}
\caption{SetCompletedAction}
\label{lst:asm:SetCompletedAction}
\end{listing}





\begin{listing}[H]
\begin{minted}[fontsize=\small]{lexer.py:CoreASMLexer -x}
rule StartSend(MI, currentStateNumber) = {
  let processID = processIDFor(self) in
  let transitionNumber = first_outgoingNormalTransition(processID, currentStateNumber) in {
    receivers(channelFor(self), MI, currentStateNumber) := undef
    reservationsDone(channelFor(self), MI, currentStateNumber) := {}
    messageContent(channelFor(self), MI, currentStateNumber) := loadVar(MI, messageContentVar(processID, first_outgoingNormalTransition(processID, currentStateNumber)))

    let cIDVarname = messageNewCorrelationVar(processID, transitionNumber) in
      if (cIDVarname != undef and cIDVarname != "") then {
        SetVar(MI, cIDVarname, "CorrelationID", nextCorrelationID)
        // FIXME: it's a bad idea to save the cID early and read this variable later: it might get changed in the meantime
        nextCorrelationID := nextCorrelationID + 1
        nextCorrelationIDUsedBy(nextCorrelationID) := self // ensure no other agent increments nextCorrelationID
      }
  }

  SetExecutionState(MI, currentStateNumber, 3) // handle all other startStates, LTS step afterwards
}
\end{minted}
\caption{StartSend}
\label{lst:asm:StartSend}
\end{listing}




\begin{listing}[H]
\begin{minted}[fontsize=\small]{lexer.py:CoreASMLexer -x}
rule PerformSend(MI, currentStateNumber) = {
  if (receivers(channelFor(self), MI, currentStateNumber) = undef) then {
    SelectReceivers(MI, currentStateNumber) // sets executionState to DONE / REPEAT / NEXT
  }
  else if (messageContent(channelFor(self), MI, currentStateNumber) = undef) then {
    SetMessageContent(MI, currentStateNumber) // sets executionState to DONE / NEXT
  }
  else if (startTime(channelFor(self), MI, currentStateNumber) = undef) then {
    StartTimeout(MI, currentStateNumber)

    SetExecutionState(MI, currentStateNumber, 0)
  }
  else if ( | receivers(channelFor(self), MI, currentStateNumber) | = | reservationsDone(channelFor(self), MI, currentStateNumber) |) then {
    TryCompletePerformSend(MI, currentStateNumber) // sets executionState to NEXT / REPEAT
  }
  else if (shouldTimeout(channelFor(self), MI, currentStateNumber) = true) then {
    SetCompleted(MI, currentStateNumber) // sets executionState to REPEAT
    ActivateTimeout(MI, currentStateNumber)
  }
  else {
    DoReservations(MI, currentStateNumber) // sets executionState to DONE or NEXT
  }
}
\end{minted}
\caption{PerformSend}
\label{lst:asm:PerformSend}
\end{listing}




\begin{listing}[H]
\begin{minted}[fontsize=\small]{lexer.py:CoreASMLexer -x}
rule TryCompletePerformSend(MI, currentStateNumber) = {
  if (anyNonProperTerminated(receivers(channelFor(self), MI, currentStateNumber)) = true) then {
    debuginfo TryCompletePerformSend self + ": a receiver where a reservation was placed has terminated non-proper in the meantime, refusing to continue"

    if (shouldTimeout(channelFor(self), MI, currentStateNumber) = true) then {
      debuginfo TryCompletePerformSend self + ": shouldTimeout"
      SetCompleted(MI, currentStateNumber) // sets executionState to REPEAT
      ActivateTimeout(MI, currentStateNumber)
    }
    else {
      debuginfo TryCompletePerformSend self + ": NEXT"
      SetExecutionState(MI, currentStateNumber, 2)
    }
  }
  else {
    // there must be only one transition
    let transitionNumber = first_outgoingNormalTransition(processIDFor(self), currentStateNumber) in {
      selectedTransition(channelFor(self), MI, currentStateNumber) := transitionNumber
    }

    SetCompleted(MI, currentStateNumber) // sets executionState to REPEAT
  }
}
\end{minted}
\caption{TryCompletePerformSend}
\label{lst:asm:TryCompletePerformSend}
\end{listing}




\begin{listing}[H]
\begin{minted}[fontsize=\small]{lexer.py:CoreASMLexer -x}
rule SelectReceivers(MI, currentStateNumber) = {
  let processID = processIDFor(self) in
  let transitionNumber = first_outgoingNormalTransition(processID, currentStateNumber) in
  let countMin = messageSubjectCountMin(processID, transitionNumber),
    countMax = messageSubjectCountMax(processID, transitionNumber) in {
    if (messageSubjectVar(processID, transitionNumber) != undef and messageSubjectVar(processID, transitionNumber) != "") then {
      let rChs = loadChannelsFromVariable(MI, messageSubjectVar(processID, transitionNumber), messageSubjectId(processID, transitionNumber)) in {

        // countMin = 0 => ALL
        if (| rChs | = 0 or (countMin != 0 and | rChs | < countMin)) then {
          debuginfo SelectReceivers self + ": WARN: not enough receivers given in '" + messageSubjectVar(processID, transitionNumber) + "'"

          SetExecutionState(MI, currentStateNumber, 2) // NEXT
        }
        // countMax = 0 => at least countMin, but unlimited
        else if (countMax != 0 and | rChs | > countMax)  then {
          debuginfo SelectReceivers self + ": too many receivers given -> Selection"

          SelectReceivers_Selection(MI, currentStateNumber, rChs, countMin, countMax) // sets either Next or Repeat
        }
        else {
          debuginfo SelectReceivers self + ": receivers fit min/max -> use them"

          receivers(channelFor(self), MI, currentStateNumber) := rChs

          debuginfo SelectReceivers self + ": REPEAT"
          SetExecutionState(MI, currentStateNumber, 0)
        }
      }
    }
    else {
      if (selectAgentsResult(channelFor(self), MI, currentStateNumber) != undef) then {
        let y = selectAgentsResult(channelFor(self), MI, currentStateNumber) in
        {
          if (|y| < countMin or |y| > countMax) then {
            debuginfo SelectReceivers self + ": invalid receivers selected, must be between "+countMin+" and "+countMax+"!"
            Crash()
          }

          receivers(channelFor(self), MI, currentStateNumber) := y

          selectAgentsResult(channelFor(self), MI, currentStateNumber) := undef

          SetExecutionState(MI, currentStateNumber, 0)
        }
      }
      else {
        let sIDLocal = messageSubjectId(processID, transitionNumber) in {
          SelectAgents(MI, currentStateNumber, sIDLocal, countMin, countMax) // sets executionState to DONE / REPEAT / NEXT
        }
      }
    }
  }
}
\end{minted}
\caption{SelectReceivers}
\label{lst:asm:SelectReceivers}
\end{listing}




\begin{listing}[H]
\begin{minted}[fontsize=\small]{lexer.py:CoreASMLexer -x}
rule SelectReceivers_Selection(MI, currentStateNumber, rChs, minimum, maximum) = {
  let res = selectionResult(channelFor(self), MI, currentStateNumber) in
  if (res = undef) then {
    let src = ["ChannelInformation", rChs] in {
      Selection(MI, currentStateNumber, src, minimum, maximum)
    }
  }
  else {
    selectionResult(channelFor(self), MI, currentStateNumber) := undef

    receivers(channelFor(self), MI, currentStateNumber) := res

    SetExecutionState(MI, currentStateNumber, 0)
  }
}
\end{minted}
\caption{SelectReceivers_Selection}
\label{lst:asm:SelectReceivers_Selection}
\end{listing}




\begin{listing}[H]
\begin{minted}[fontsize=\small]{lexer.py:CoreASMLexer -x}
rule AbortSend(MI, currentStateNumber) = {
  foreach r in reservationsDone(channelFor(self), MI, currentStateNumber) do {
    CancelReservation(MI, currentStateNumber, r)
  }

  SetAbortionCompleted(MI, currentStateNumber)  // sets executionState to DONE
}
\end{minted}
\caption{AbortSend}
\label{lst:asm:AbortSend}
\end{listing}




\begin{listing}[H]
\begin{minted}[fontsize=\small]{lexer.py:CoreASMLexer -x}
rule PerformTransitionSend(MI, currentStateNumber, transitionNumber) = {
  let processID = processIDFor(self) in {
    let storeReceiverVarname = messageStoreReceiverVar(processID, transitionNumber) in
      if (storeReceiverVarname != undef and storeReceiverVarname != "") then {
        SetVar(MI, storeReceiverVarname, "ChannelInformation", reservationsDone(channelFor(self), MI, currentStateNumber))
      }

    foreach r in reservationsDone(channelFor(self), MI, currentStateNumber) do {
      ReplaceReservation(MI, currentStateNumber, r)

      EnsureRunning(r)
    }

    SetCompletedTransition(MI, currentStateNumber, transitionNumber) // sets executionState to REPEAT
  }
}
\end{minted}
\caption{PerformTransitionSend}
\label{lst:asm:PerformTransitionSend}
\end{listing}




\begin{listing}[H]
\begin{minted}[fontsize=\small]{lexer.py:CoreASMLexer -x}
rule SetMessageContent(MI, currentStateNumber) = {
  if not(contains(wantInput(channelFor(self), MI, currentStateNumber), "MessageContentDecision")) then {
    add "MessageContentDecision" to wantInput(channelFor(self), MI, currentStateNumber)

    SetExecutionState(MI, currentStateNumber, 1)
  }
  else {
    debuginfo SetMessageContent self + ": waiting for messageContent"
    debuginfo SetMessageContent self + ": allow next states to be executed, but not with lower priority"
    SetExecutionState(MI, currentStateNumber, 2)
  }
}
\end{minted}
\caption{SetMessageContent}
\label{lst:asm:SetMessageContent}
\end{listing}




\begin{listing}[H]
\begin{minted}[fontsize=\small]{lexer.py:CoreASMLexer -x}
rule DoReservations(MI, currentStateNumber) = {
  local boolres := false in { // hasPlacedReservation
    seq
      let receiversTodo = (receivers(channelFor(self), MI, currentStateNumber) diff reservationsDone(channelFor(self), MI, currentStateNumber)) in
      foreach receiver in receiversTodo do {
        local boolres1 := false in {
          seq
            boolres1 <- DoReservation(MI, currentStateNumber, receiver) // returns true iff a reservation was made
          next
            if (boolres1 = true) then {
              boolres := true
            }
        }
      }
    next
      if (boolres = true) then {
        debuginfo DoReservations self + ": DONE: reservation(s) placed, make update"
        SetExecutionState(MI, currentStateNumber, 1)
      }
      else {
        debuginfo DoReservations self + ": NEXT: no reservations made, allow other states"
        SetExecutionState(MI, currentStateNumber, 2)
      }
  }
}
\end{minted}
\caption{DoReservations}
\label{lst:asm:DoReservations}
\end{listing}





\begin{listing}[H]
\begin{minted}[fontsize=\small]{lexer.py:CoreASMLexer -x}
// result = hasPlacedReservation
rule DoReservation(MI, currentStateNumber, receiverChannel) = {
  if (properTerminated(receiverChannel) = true) then {
      let processID          = processIDFor(self) in
      let transitionNumber   = first_outgoingNormalTransition(processID, currentStateNumber),
        senderChannel      = channelFor(self),
        receiverProcessID  = processIDOf(receiverChannel) in
      let senderSubjectID    = searchSenderSubjectID(processID, subjectIDFor(self), receiverProcessID),
        msgCorrelationID   = loadCorrelationID(MI, messageNewCorrelationVar(processID, transitionNumber)),
        ipCorrelationID    = loadCorrelationID(MI, messageWithCorrelationVar(processID, transitionNumber)) in
      let reservationMessage = [senderChannel, messageType(processID, transitionNumber), {}, msgCorrelationID, true] in {
        seq
          if (inputPool(receiverChannel, senderSubjectID, messageType(processID, transitionNumber), ipCorrelationID) = undef) then {
              add [senderSubjectID, messageType(processID, transitionNumber), ipCorrelationID] to inputPoolDefined(receiverChannel)
              inputPool(receiverChannel, senderSubjectID, messageType(processID, transitionNumber), ipCorrelationID) := []
          }
        next
          if (inputPoolIsClosed(receiverChannel, senderSubjectID, messageType(processID, transitionNumber), ipCorrelationID) != true) then {
              if (inputPoolGetFreeSpace(receiverChannel, senderSubjectID, messageType(processID, transitionNumber)) > 0) then {
                enqueue reservationMessage into inputPool(receiverChannel, senderSubjectID, messageType(processID, transitionNumber), ipCorrelationID)
                add receiverChannel to reservationsDone(channelFor(self), MI, currentStateNumber)
                debuginfo DoReservation self + ": added reservation to inputPool: " + reservationMessage
                result := true
              }
              else {
                debuginfo DoReservation self + ": no free space!"
                result := false
              }
          }
          else {
              debuginfo DoReservation self + ": inputPoolIsClosed"
              result := false
          }
      }
  }
  else {
      debuginfo DoReservation self + ": non-properTerminated, skipping receiver"
      result := false
  }
}
\end{minted}
\caption{DoReservation}
\label{lst:asm:DoReservation}
\end{listing}




\begin{listing}[H]
\begin{minted}[fontsize=\small]{lexer.py:CoreASMLexer -x}
rule CancelReservation(MI, currentStateNumber, receiverChannel) = {
  let processID          = processIDFor(self) in
  let transitionNumber   = first_outgoingNormalTransition(processID, currentStateNumber),
      senderChannel      = channelFor(self),
      receiverProcessID  = processIDOf(receiverChannel) in
  let senderSubjectID    = searchSenderSubjectID(processID, subjectIDFor(self), receiverProcessID),
      msgCorrelationID   = loadCorrelationID(MI, messageNewCorrelationVar(processID, transitionNumber)),
      ipCorrelationID    = loadCorrelationID(MI, messageWithCorrelationVar(processID, transitionNumber)) in
  let reservationMessage = [senderChannel, messageType(processID, transitionNumber), {}, msgCorrelationID, true],
      IP = inputPool(receiverChannel, senderSubjectID, messageType(processID, transitionNumber), ipCorrelationID) in {
      inputPool(receiverChannel, senderSubjectID, messageType(processID, transitionNumber), ipCorrelationID) := dropnth(IP, head(indexes(IP, reservationMessage)))
  }
}
\end{minted}
\caption{CancelReservation}
\label{lst:asm:CancelReservation}
\end{listing}




\begin{listing}[H]
\begin{minted}[fontsize=\small]{lexer.py:CoreASMLexer -x}
rule ReplaceReservation(MI, currentStateNumber, receiverChannel) = {
  let processID          = processIDFor(self) in
  let transitionNumber   = first_outgoingNormalTransition(processID, currentStateNumber),
      senderChannel      = channelFor(self),
      receiverProcessID  = processIDOf(receiverChannel) in
  let senderSubjectID    = searchSenderSubjectID(processID, subjectIDFor(self), receiverProcessID),
      msgCorrelationID   = loadCorrelationID(MI, messageNewCorrelationVar(processID, transitionNumber)),
      ipCorrelationID    = loadCorrelationID(MI, messageWithCorrelationVar(processID, transitionNumber)) in
  let reservationMessage = [senderChannel, messageType(processID, transitionNumber), {}, msgCorrelationID, true],
      message            = [senderChannel, messageType(processID, transitionNumber), messageContent(channelFor(self), MI, currentStateNumber), msgCorrelationID, false],
      IP = inputPool(receiverChannel, senderSubjectID, messageType(processID, transitionNumber), ipCorrelationID) in {
      // TODO: discuss: setnth or dropnth & enqueue?
      inputPool(receiverChannel, senderSubjectID, messageType(processID, transitionNumber), ipCorrelationID) := setnth(IP, head(indexes(IP, reservationMessage)), message)
  }
}
\end{minted}
\caption{ReplaceReservation}
\label{lst:asm:ReplaceReservation}
\end{listing}





\begin{listing}[H]
\begin{minted}[fontsize=\small]{lexer.py:CoreASMLexer -x}
rule PerformReceive(MI, currentStateNumber) = {
  // startTime must be the time of the first attempt to receive in order to support receiving with timeout=0
  if (startTime(channelFor(self), MI, currentStateNumber) = undef) then {
    StartTimeout(MI, currentStateNumber)

    SetExecutionState(MI, currentStateNumber, 0)
  }
  else {
    if (shouldTimeout(channelFor(self), MI, currentStateNumber) = true) then {
      SetCompleted(MI, currentStateNumber) // sets executionState to REPEAT
      ActivateTimeout(MI, currentStateNumber)
    }
    else {
      let processID = processIDFor(self) in
      seq
        forall transitionNumber in outgoingNormalTransitions(processID, currentStateNumber) do {
          CheckIP(MI, currentStateNumber, transitionNumber)
        }
      next
        let enabledOutgoingTransitions = outgoingEnabledTransitions(channelFor(self), MI, currentStateNumber) in {
          if (|enabledOutgoingTransitions| > 0) then {
            seqblock
            debuginfo PerformReceive self + ": at least one IP with messages :)"

            if (selectedTransition(channelFor(self), MI, currentStateNumber) != undef) then {
              debuginfo PerformReceive self + ": skipping automatic decision, there is already an transition selected: " + selectedTransition(channelFor(self), MI, currentStateNumber)
            }
            else if (|enabledOutgoingTransitions| = 1) then {
              let transitionNumber = firstFromSet(enabledOutgoingTransitions) in {
                if (transitionIsAuto(processID, transitionNumber) = true) then {
                  debuginfo PerformReceive self + ": making automatic decision for transition " + transitionPretty(MI, transitionNumber)
                  selectedTransition(channelFor(self), MI, currentStateNumber) := transitionNumber
                }
                else {
                  debuginfo PerformReceive self + ": can not make automatic decision, not an auto transition: " + transitionPretty(MI, transitionNumber)
                }
              }
            }
            else {
              debuginfo PerformReceive self + ": can not make automatic decision, too much transitions: " + enabledOutgoingTransitions
            }

            if (selectedTransition(channelFor(self), MI, currentStateNumber) != undef) then {
              debuginfo PerformReceive self + ": the decision has been made.."
              SetCompleted(MI, currentStateNumber) // sets executionState to REPEAT
            }
            else {
              debuginfo PerformReceive self + ": no decision made, waiting for selectedTransition"

              SelectTransition(MI, currentStateNumber)
            }
            endseqblock
          }
          else {
            debuginfo PerformReceive self + ": no IP with messages, trying later.."
            debuginfo PerformReceive self + ": allow all other states to be executed, especially states with lower priority"
            SetExecutionState(MI, currentStateNumber, 3)
          }
        }
    }
  }
}
\end{minted}
\caption{PerformReceive}
\label{lst:asm:PerformReceive}
\end{listing}




\begin{listing}[H]
\begin{minted}[fontsize=\small]{lexer.py:CoreASMLexer -x}
rule PerformTransitionReceive(MI, currentStateNumber, transitionNumber) = {
  ReceiveMessage(MI, currentStateNumber, transitionNumber)
}
\end{minted}
\caption{PerformTransitionReceive}
\label{lst:asm:PerformTransitionReceive}
\end{listing}




\begin{listing}[H]
\begin{minted}[fontsize=\small]{lexer.py:CoreASMLexer -x}
rule ReceiveMessage(MI, currentStateNumber, transitionNumber) = {
  let processID   = processIDFor(self) in
  let s           = messageSubjectId         (processID, transitionNumber),
      sChsVarname = messageSubjectVar        (processID, transitionNumber),
      mt          = messageType              (processID, transitionNumber),
      cIDVarname  = messageWithCorrelationVar(processID, transitionNumber) in
  // TODO 2019-02-22: local receivedMessages ? Or is that function used elsewhere?
  // Alternative: directly return as listres via result in InputPool_Pop?
  local stringres1, // subjectID
      setres1,    // subjectChannels
      stringres2, // messageType
      numres1 in  // correlationID
  seqblock

      debuginfo ReceiveMessage self + ": ReceiveMessage in state " + statePretty(MI, currentStateNumber) + " with transition " + transitionPretty(MI, transitionNumber)

      // TODO/NOTE: same structure as CheckIP. May refactor to reduce duplicated code?

      if (s = "*") then {
        debuginfo ReceiveMessage self + ": wildcard for subject is ? and not *"
        Crash()
      }
      else if (s = "?") then {
        stringres1 := undef
      }
      else {
        stringres1 := s
      }


      if (sChsVarname = undef or sChsVarname = "") then {
        setres1 := undef
      }
      else {
        seq
          setres1 := loadChannelsFromVariable(MI, sChsVarname, s) // (stringres1 would be fine too)
        next
          debuginfo CheckIP self + ": considering only messages from the following channels: " + setres1
      }


      if (mt = "*") then {
        debuginfo ReceiveMessage self + ": wildcard for message type is ? and not *"
        Crash()
      }
      else if (mt = "?") then {
        stringres2 := undef
      }
      else {
        stringres2 := mt
      }


      if (cIDVarname = "*") then {
        debuginfo ReceiveMessage self + ": wildcard for cID type is ? and not *"
        Crash()
      }
      else if (cIDVarname = "?") then {
        numres1 := undef
      }
      else {
        numres1 := loadCorrelationID(MI, cIDVarname)
      }



      let sL    = stringres1,
        sChsL = setres1,
        mtL   = stringres2,
        cL    = numres1,
        countMinL = messageSubjectCountMin(processID, transitionNumber),
        countMaxL = messageSubjectCountMax(processID, transitionNumber) in {
        // InputPool_Pop stores the popped messages in receivedMessages
        // undef is wildcard
        InputPool_Pop(MI, currentStateNumber, sL, sChsL, mtL, cL, countMinL, countMaxL)
      }

      let msgs = receivedMessages(channelFor(self), MI, currentStateNumber) in {
        debuginfo ReceiveMessage self + ": receivedMessages: " + msgs

        if (messageStoreMessagesVar(processID, transitionNumber) != undef and messageStoreMessagesVar(processID, transitionNumber) != "") then {
          let varname = messageStoreMessagesVar(processID, transitionNumber) in {
              debuginfo ReceiveMessage self + ": storing messages in: '" + varname + "'"
              SetMessageSet(MI, varname, msgs)
          }
        }
      }

      SetCompletedTransition(MI, currentStateNumber, transitionNumber) // sets executionState to REPEAT
  endseqblock
}
\end{minted}
\caption{ReceiveMessage}
\label{lst:asm:ReceiveMessage}
\end{listing}




\begin{listing}[H]
\begin{minted}[fontsize=\small]{lexer.py:CoreASMLexer -x}
rule CheckIP(MI, currentStateNumber, transitionNumber) = {
  let processID   = processIDFor(self) in
  let sID         = messageSubjectId         (processID, transitionNumber),
      sChsVarname = messageSubjectVar        (processID, transitionNumber),
      mT          = messageType              (processID, transitionNumber),
      cIDVarname  = messageWithCorrelationVar(processID, transitionNumber) in
  local stringres1, // subjectID
      setres1,    // subjectChannels
      stringres2, // messageType
      numres1 in  // correlationID
  seqblock

      // TODO/NOTE: same structure as ReceiveMessage. May refactor to reduce duplicated code?

      if (sID = "?") then {
        stringres1 := undef
      }
      else if (sID = "*") then {
        debuginfo CheckIP self + ": subject wildcard is '?'"
        Crash()
      }
      else {
        stringres1 := sID
      }


      if (sChsVarname = undef or sChsVarname = "") then {
        setres1 := undef
      }
      else {
        seq
          setres1 := loadChannelsFromVariable(MI, sChsVarname, sID) // (stringres1 would be fine too)
        next
          debuginfo CheckIP self + ": considering only messages from the following channels: " + setres1
      }


      if (mT = "?") then {
        stringres2 := undef
      }
      else if (mT = "*") then {
        debuginfo CheckIP self + ": message type wildcard is '?'"
        Crash()
      }
      else {
        stringres2 := mT
      }


      if (cIDVarname = "?") then {
        numres1 := undef
      }
      else if (cIDVarname = "*") then {
        debuginfo CheckIP self + ": correlationID wildcard is '?'"
        Crash()
      }
      else {
        numres1 := loadCorrelationID(MI, messageWithCorrelationVar(processID, transitionNumber))
      }


      // no reservations, supress multiple channels
      let usedSpace = inputPoolUsedSpace(channelFor(self), stringres1, setres1, stringres2, numres1, true, true) in {
        if (messageSubjectCountMin(processID, transitionNumber) = 0) then {
          if (usedSpace > |setres1|) then {
              debuginfo CheckIP self + ": WARN: internal error: more messages than expected senders! " + usedSpace + " > " + |setres1|
              Crash()
          }
          else if (usedSpace = |setres1|) then {
              debuginfo CheckIP self + ": enough messages :) " + usedSpace + " = " + |setres1|
              EnableTransition(MI, transitionNumber)
          }
          else {
              debuginfo CheckIP self + ": Not enough messages: " + usedSpace + " < " + |setres1|
              DisableTransition(MI, currentStateNumber, transitionNumber)
          }
        }
        else if (usedSpace >= messageSubjectCountMin(processID, transitionNumber)) then {
          debuginfo CheckIP self + ": enough messages :) " + usedSpace + " >= " + messageSubjectCountMin(processID, transitionNumber)
          EnableTransition(MI, transitionNumber)
        }
        else {
          debuginfo CheckIP self + ": Not enough messages: " + usedSpace + " < " + messageSubjectCountMin(processID, transitionNumber)
          DisableTransition(MI, currentStateNumber, transitionNumber)
        }
      }
  endseqblock
}
\end{minted}
\caption{CheckIP}
\label{lst:asm:CheckIP}
\end{listing}





\begin{listing}[H]
\begin{minted}[fontsize=\small]{lexer.py:CoreASMLexer -x}
rule StartEnd(MI, currentStateNumber) = {
  SetExecutionState(MI, currentStateNumber, 0) // REPEAT, continue directly to PerformEnd, avoiding LTS step
}
\end{minted}
\caption{StartEnd}
\label{lst:asm:StartEnd}
\end{listing}




\begin{listing}[H]
\begin{minted}[fontsize=\small]{lexer.py:CoreASMLexer -x}
rule PerformEnd(MI, currentStateNumber) = {
  if (|activeStates(channelFor(self), MI)| > 1) then {
      AbortMacroInstance(MI, currentStateNumber) // do not remove self. calls ClearAllVarInMIForChannel bat that's not needed

      SetExecutionState(MI, currentStateNumber, 1) // DONE, make global update
  }
  else {
      if (MI = 1) then {
        let res = head(stateFunctionArguments(processIDFor(self), currentStateNumber)) in {
          // just for debugging purposes, a termination of the Main Macro should not have a result, but this is used in TestTransitions2
          if (res = undef) then { // no parameters for End state
              debuginfo PerformEnd self + ": within mainMacro, terminate subject without result value"
          }
          else {
              debuginfo PerformEnd self + ": within mainMacro. WARN: terminate subject with result value: " + res
          }
        }

        ClearAllVarInMIForChannel(channelFor(self), 0)
        ClearAllVarInMIForChannel(channelFor(self), 1)

        FinalizeInteraction()

        program(self) := undef
        remove self from asmAgents
      }
      else {
        ClearAllVarInMIForChannel(channelFor(self), MI)

        let res = head(stateFunctionArguments(processIDFor(self), currentStateNumber)) in {
          if (res = undef) then { // no parameters for End state
              debuginfo PerformEnd self + ": terminated without result value"
              macroTerminationResult(channelFor(self), MI) := true
          }
          else {
              debuginfo PerformEnd self + ": terminated with result value: " + res
              macroTerminationResult(channelFor(self), MI) := res
          }
        }
      }

      // remove self
      RemoveState(MI, currentStateNumber, MI, currentStateNumber)
      SetExecutionState(MI, currentStateNumber, 1) // DONE - make global update
  }
}
\end{minted}
\caption{PerformEnd}
\label{lst:asm:PerformEnd}
\end{listing}






\begin{listing}[H]
\begin{minted}[fontsize=\small]{lexer.py:CoreASMLexer -x}
rule StartTau(MI, currentStateNumber) = {
  EnableAllTransitions(MI, currentStateNumber)
}

rule Tau(MI, currentStateNumber, args) = {
  let processID = processIDFor(self) in {
      choose transitionNumber in outgoingEnabledTransitions(channelFor(self), MI, currentStateNumber) with (transitionIsAuto(processID, transitionNumber) = true) do {
        debuginfo Tau self + ": transition chosen, it is normal and auto: " + transitionPretty(MI, transitionNumber)

        selectedTransition(channelFor(self), MI, currentStateNumber) := transitionNumber

        SetCompleted(MI, currentStateNumber) // sets executionState to REPEAT
      }
      ifnone {
        debuginfo Tau self + ": unable to choose auto transition!"

        if (selectedTransition(channelFor(self), MI, currentStateNumber) != undef) then {
          debuginfo Tau self + ": selectedTransition had been set"
          SetCompleted(MI, currentStateNumber) // sets executionState to REPEAT
        }
        else {
          debuginfo Tau self + ": SelectTransition"
          SelectTransition(MI, currentStateNumber)
        }
      }
  }
}
\end{minted}
\caption{Tau}
\label{lst:asm:Tau}
\end{listing}




\begin{listing}[H]
\begin{minted}[fontsize=\small]{lexer.py:CoreASMLexer -x}
rule AbortVarMan(MI, currentStateNumber) = {
  ResetSelection(MI, currentStateNumber)
  SetAbortionCompleted(MI, currentStateNumber)
}

rule VarMan(MI, currentStateNumber, args) = {
  let method = head(args) in {
      debuginfo VarMan self + ": method: " + method
      debuginfo VarMan self + ": args: " + tail(args)

      case method of
        "assign"                : VarMan_Assign               (MI, currentStateNumber, nth(args, 2), nth(args, 3))
        "storeData"             : VarMan_StoreData            (MI, currentStateNumber, nth(args, 2), nth(args, 3))
        "clear"                 : VarMan_Clear                (MI, currentStateNumber, nth(args, 2))

        "concatenation"         : VarMan_Concatenation        (MI, currentStateNumber, nth(args, 2), nth(args, 3), nth(args, 4))
        "intersection"          : VarMan_Intersection         (MI, currentStateNumber, nth(args, 2), nth(args, 3), nth(args, 4))
        "difference"            : VarMan_Difference           (MI, currentStateNumber, nth(args, 2), nth(args, 3), nth(args, 4))

        "extractContent"        : VarMan_ExtractContent       (MI, currentStateNumber, nth(args, 2), nth(args, 3))
        "extractChannel"        : VarMan_ExtractChannel       (MI, currentStateNumber, nth(args, 2), nth(args, 3))
        "extractCorrelationID"  : VarMan_ExtractCorrelationID (MI, currentStateNumber, nth(args, 2), nth(args, 3))

        "selection"             : VarMan_Selection            (MI, currentStateNumber, nth(args, 2), nth(args, 3), nth(args, 4), nth(args, 5))
      endcase
  }
}


rule VarMan_Assign(MI, currentStateNumber, A, X) = {
  let a = loadVar(MI, A) in {
      SetVar(MI, X, head(a), last(a))

      SetCompletedAction(MI, currentStateNumber, undef) // sets executionState to REPEAT
  }
}

rule VarMan_StoreData(MI, currentStateNumber, X, A) = {
  SetVar(MI, X, "Data", A)

  SetCompletedAction(MI, currentStateNumber, undef) // sets executionState to REPEAT
}

rule VarMan_Clear(MI, currentStateNumber, X) = {
  ClearVar(MI, X)

  SetCompletedAction(MI, currentStateNumber, undef) // sets executionState to REPEAT
}

rule VarMan_Concatenation(MI, currentStateNumber, A, B, X) = {
  let a = loadVar(MI, A),
      b = loadVar(MI, B) in {
      if (a = undef and b = undef) then {
        ClearVar(MI, X)

        SetCompletedAction(MI, currentStateNumber, undef) // sets executionState to REPEAT
      }
      else if (a = undef) then {
        SetVar(MI, X, head(b), last(b))

        SetCompletedAction(MI, currentStateNumber, undef) // sets executionState to REPEAT
      }
      else if (b = undef) then {
        SetVar(MI, X, head(a), last(a))

        SetCompletedAction(MI, currentStateNumber, undef) // sets executionState to REPEAT
      }
      else if (head(a) != head(b) or not(contains({"MessageSet", "ChannelInformation"}, head(a)))) then {
        debuginfo VarMan_Concatenation self + ": invalid parameters"

        Crash()
      }
      else {
        let x = (last(a) union last(b)) in {
          SetVar(MI, X, head(a), x)

          SetCompletedAction(MI, currentStateNumber, undef) // sets executionState to REPEAT
        }
      }
  }
}

rule VarMan_Intersection(MI, currentStateNumber, A, B, X) = {
  let a = loadVar(MI, A) in
  let b = loadVar(MI, B) in {
      if (a = undef or b = undef) then {
        ClearVar(MI, X)

        SetCompletedAction(MI, currentStateNumber, undef) // sets executionState to REPEAT
      }
      else if (head(a) != head(b) or not(contains({"MessageSet", "ChannelInformation"}, head(a)))) then {
        debuginfo VarMan_Intersection self + ": invalid parameters"
        debuginfo VarMan_Intersection self + ": A = " + undefStr(A)
        debuginfo VarMan_Intersection self + ": a = " + undefStr(a)
        debuginfo VarMan_Intersection self + ": B = " + undefStr(B)
        debuginfo VarMan_Intersection self + ": b = " + undefStr(b)

        Crash()
      }
      else {
        let x = (last(a) intersect last(b)) in {
          SetVar(MI, X, head(a), x)

          SetCompletedAction(MI, currentStateNumber, undef) // sets executionState to REPEAT
        }
      }
  }
}

rule VarMan_Difference(MI, currentStateNumber, A, B, X) = {
  let a = loadVar(MI, A) in
  let b = loadVar(MI, B) in {
      if (a = undef) then {
        ClearVar(MI, X)

        SetCompletedAction(MI, currentStateNumber, undef) // sets executionState to REPEAT
      }
      else if (b = undef) then {
        SetVar(MI, X, head(a), last(a))

        SetCompletedAction(MI, currentStateNumber, undef) // sets executionState to REPEAT
      }
      else if (head(a) != head(b) or not(contains({"MessageSet", "ChannelInformation"}, head(a)))) then {
        debuginfo VarMan_Difference self + ": invalid parameters"
        debuginfo VarMan_Difference self + ": A = " + undefStr(A)
        debuginfo VarMan_Difference self + ": a = " + undefStr(a)
        debuginfo VarMan_Difference self + ": B = " + undefStr(B)
        debuginfo VarMan_Difference self + ": b = " + undefStr(b)

        Crash()
      }
      else {
        let x = (last(a) diff last(b)) in {
          SetVar(MI, X, head(a), x)

          SetCompletedAction(MI, currentStateNumber, undef) // sets executionState to REPEAT
        }
      }
  }
}


rule VarMan_ExtractContent(MI, currentStateNumber, A, X) = {
  let a = loadVar(MI, A) in {
    // TODO: How to deal with empty / undef variables?
    if (head(a) != "MessageSet") then {
      debuginfo VarMan_ExtractContent self + ": invalid parameter"
      debuginfo VarMan_ExtractContent self + ": A = " + undefStr(A)
      debuginfo VarMan_ExtractContent self + ": a = " + undefStr(a)

      Crash()
    }
    else {
      let messages = last(a) in
      let messagesContent = map(messages, @msgContent) in {
        if (| messagesContent | = 0) then {
          debuginfo VarMan_ExtractContent self + ": no content to extract"
          Crash()
        }
        else if (| messagesContent | = 1) then {
          let content = firstFromSet(messagesContent) in {
            debuginfo VarMan_ExtractContent self + ": exact one content to extract: " + content
            SetVar(MI, X, head(content), last(content))
          }
        }
        else {
          // check if all contents have the same type, and that the type is union-able
          let x = firstFromSet(messagesContent) in {
            if (head(x) = "MessageSet" or head(x) = "ChannelInformation") then {
              // check if all are the same type
              if (forall y in messagesContent holds (head(y) = head(x))) then {
                debuginfo VarMan_ExtractContent self + ": flattening content: " + content
                SetVar(MI, X, head(x), flattenSet(map(messagesContent, @last)))
              }
              else {
                debuginfo VarMan_ExtractContent self + ": all messages must have the same underlying datatype!"
                debuginfo VarMan_ExtractContent self + ": messages: " + messages
                debuginfo VarMan_ExtractContent self + ": messagesContent: " + messagesContent

                Crash()
              }
            }
            else {
              debuginfo VarMan_ExtractContent self + ": invalid message content type, can not be merged: '" + head(x) + "'"

              Crash()
            }
          }
        }

        SetCompletedAction(MI, currentStateNumber, undef) // sets executionState to REPEAT
      }
    }
  }
}

rule VarMan_ExtractChannel(MI, currentStateNumber, A, X) = {
  let a = loadVar(MI, A) in {
    // TODO: How to deal with empty / undef variables?
    if (head(a) != "MessageSet") then {
      debuginfo VarMan_ExtractChannel self + ": invalid parameter"
      debuginfo VarMan_ExtractChannel self + ": A = " + undefStr(A)
      debuginfo VarMan_ExtractChannel self + ": a = " + undefStr(a)

      Crash()
    }
    else {
      let messages = last(a) in
      let messagesChannel = map(messages, @msgChannel) in {
        if (| messagesChannel | = 0) then {
          debuginfo VarMan_ExtractChannel self + ": no channels to extract"
          Crash()
        }
        else {
          debuginfo VarMan_ExtractChannel self + ": messagesChannel: " + messagesChannel
          SetVar(MI, X, "ChannelInformation", messagesChannel)
        }

        SetCompletedAction(MI, currentStateNumber, undef) // sets executionState to REPEAT
      }
    }
  }
}

rule VarMan_ExtractCorrelationID(MI, currentStateNumber, A, X) = {
  let a = loadVar(MI, A) in {
    // TODO: How to deal with empty / undef variables?
    if (head(a) != "MessageSet") then {
      debuginfo VarMan_ExtractCorrelationID self + ": invalid parameter"
      debuginfo VarMan_ExtractCorrelationID self + ": A = " + undefStr(A)
      debuginfo VarMan_ExtractCorrelationID self + ": a = " + undefStr(a)

      Crash()
    }
    else {
      let messages = last(a) in
      let messagesCorrelationID = map(messages, @msgCorrelation) in {
        if (| messagesCorrelationID | != 1) then {
          debuginfo VarMan_ExtractCorrelationID self + ": a CorrelationID can only be extracted when there is exactly one, got: " + messagesCorrelationID
          Crash()
        }
        else {
          debuginfo VarMan_ExtractCorrelationID self + ": messagesCorrelationID: " + messagesCorrelationID
          SetVar(MI, X, "CorrelationID", firstFromSet(messagesCorrelationID))
        }

        SetCompletedAction(MI, currentStateNumber, undef) // sets executionState to REPEAT
      }
    }
  }
}
\end{minted}
\caption{VarMan}
\label{lst:asm:VarMan}
\end{listing}




\begin{listing}[H]
\begin{minted}[fontsize=\small]{lexer.py:CoreASMLexer -x}
// CH * MI * n
function selectionVartype  : LIST * NUMBER * NUMBER -> STRING
function selectionData     : LIST * NUMBER * NUMBER -> LIST
function selectionOptions  : LIST * NUMBER * NUMBER -> LIST
function selectionMin      : LIST * NUMBER * NUMBER -> NUMBER
function selectionMax      : LIST * NUMBER * NUMBER -> NUMBER
function selectionDecision : LIST * NUMBER * NUMBER -> SET

function selectionResult   : LIST * NUMBER * NUMBER -> SET

rule VarMan_Selection(MI, currentStateNumber, srcVarname, dstVarname, minimum, maximum) = {
  let src = loadVar(MI, srcVarname),
    res = selectionResult(channelFor(self), MI, currentStateNumber) in
  if (res = undef) then {
    // TODO: cancel / timeout transition?
    Selection(MI, currentStateNumber, src, minimum, maximum)
  }
  else {
    selectionResult(channelFor(self), MI, currentStateNumber) := undef

    SetVar(MI, dstVarname, head(src), res)

    SetCompletedAction(MI, currentStateNumber, undef) // sets executionState to REPEAT
  }
}

rule ResetSelection(MI, currentStateNumber) = {
  selectionVartype (channelFor(self), MI, currentStateNumber) := undef
  selectionData    (channelFor(self), MI, currentStateNumber) := undef
  selectionOptions (channelFor(self), MI, currentStateNumber) := undef
  selectionMin     (channelFor(self), MI, currentStateNumber) := undef
  selectionMax     (channelFor(self), MI, currentStateNumber) := undef
  selectionDecision(channelFor(self), MI, currentStateNumber) := undef
}

rule Selection(MI, currentStateNumber, src, minimum, maximum) = {
  if (selectionData(channelFor(self), MI, currentStateNumber) = undef) then {
      if (head(src) = "MessageSet") then {
        let l = toList(last(src)) in {
          selectionData   (channelFor(self), MI, currentStateNumber) := l
          selectionOptions(channelFor(self), MI, currentStateNumber) := map(l, @msgToString)
        }
      }
      else if (head(src) = "ChannelInformation") then {
        let l = toList(last(src)) in {
          selectionData   (channelFor(self), MI, currentStateNumber) := l
          selectionOptions(channelFor(self), MI, currentStateNumber) := map(l, @chToString)
        }
      }
      else {
        debuginfo Selection self + ": can not perform selection on datatype '" + head(x) + "'"

        Crash()
      }

      selectionVartype (channelFor(self), MI, currentStateNumber) := head(src)
      selectionMin     (channelFor(self), MI, currentStateNumber) := minimum
      selectionMax     (channelFor(self), MI, currentStateNumber) := maximum
      selectionDecision(channelFor(self), MI, currentStateNumber) := undef // just to be sure

      debuginfo Selection self + ": REPEAT => Behaviour should be executed again for this state"
      SetExecutionState(MI, currentStateNumber, 0)
  }
  else if (selectionDecision(channelFor(self), MI, currentStateNumber) = undef) then {
      if not(contains(wantInput(channelFor(self), MI, currentStateNumber), "SelectionDecision")) then {
        add "SelectionDecision" to wantInput(channelFor(self), MI, currentStateNumber)

        debuginfo Selection self + ": DONE, make global update"
        SetExecutionState(MI, currentStateNumber, 1)
      }
      else {
        debuginfo Selection self + ": waiting for selectionDecision"
        debuginfo Selection self + ": allow next states to be executed, but not with lower priority"
        SetExecutionState(MI, currentStateNumber, 2)
      }
  }
  else {
      let res = pickItems(selectionData(channelFor(self), MI, currentStateNumber), selectionDecision(channelFor(self), MI, currentStateNumber)) in {
        selectionResult(channelFor(self), MI, currentStateNumber) := res
      }

      ResetSelection(MI, currentStateNumber)

      debuginfo Selection self + ": REPEAT => Behaviour should be executed again for this state"
      SetExecutionState(MI, currentStateNumber, 0)
  }
}
\end{minted}
\caption{VarMan_Selection}
\label{lst:asm:VarMan_Selection}
\end{listing}




\begin{listing}[H]
\begin{minted}[fontsize=\small]{lexer.py:CoreASMLexer -x}
rule ModalSplit(MI, currentStateNumber, args) = {
  seqblock
    // start each following state
    foreach transitionNumber in outgoingNormalTransitions(processIDFor(self), currentStateNumber) do {
      let sNew = targetStateNumber(processIDFor(self), transitionNumber) in {
        AddState(MI, currentStateNumber, MI, sNew)
      }
    }

    // remove self
    RemoveState(MI, currentStateNumber, MI, currentStateNumber)

    SetExecutionState(MI, currentStateNumber, 1)
  endseqblock
}
\end{minted}
\caption{ModalSplit}
\label{lst:asm:ModalSplit}
\end{listing}




\begin{listing}[H]
\begin{minted}[fontsize=\small]{lexer.py:CoreASMLexer -x}
// Channel * MacroInstanceNumber * joinState -> Number
function joinCount : LIST * NUMBER * NUMBER -> NUMBER

rule ModalJoin(MI, currentStateNumber, args) = {
  let splitCount = nth(args, 1) in
  seqblock
    debuginfo ModalJoin self + ": state: " + statePretty(MI, currentStateNumber)
    debuginfo ModalJoin self + ": splitCount: " + splitCount

    if (joinCount(channelFor(self), MI, currentStateNumber) = undef) then {
      joinCount(channelFor(self), MI, currentStateNumber) := 1
    }
    else {
      joinCount(channelFor(self), MI, currentStateNumber) := joinCount(channelFor(self), MI, currentStateNumber) + 1
    }

    debuginfo ModalJoin self + ": joinCount_post: " + joinCount(channelFor(self), MI, currentStateNumber)

    // can we continue, or remove self and wait for next path joining?
    if (joinCount(channelFor(self), MI, currentStateNumber) < splitCount) then {
      // remove self
      RemoveState(MI, currentStateNumber, MI, currentStateNumber)

      debuginfo ModalJoin self + ": DONE, make global update"
      SetExecutionState(MI, currentStateNumber, 1)
    }
    else {
      joinCount(channelFor(self), MI, currentStateNumber) := undef
      SetCompletedAction(MI, currentStateNumber, undef) // sets executionState to REPEAT
    }
  endseqblock
}
\end{minted}
\caption{ModalJoin}
\label{lst:asm:ModalJoin}
\end{listing}




\begin{listing}[H]
\begin{minted}[fontsize=\small]{lexer.py:CoreASMLexer -x}
rule AbortCallMacro(MI, currentStateNumber) = {
  let childInstance = callMacroChildInstance(channelFor(self), MI, currentStateNumber) in {
    if (|activeStates(channelFor(self), childInstance)| > 0) then {
      AbortMacroInstance(childInstance, undef)

      SetExecutionState(MI, currentStateNumber, 1) // DONE, make global update
    }
    else {
      callMacroChildInstance(channelFor(self), MI, currentStateNumber) := undef

      SetAbortionCompleted(MI, currentStateNumber)  // sets executionState to DONE
    }
  }
}
\end{minted}
\caption{AbortCallMacro}
\label{lst:asm:AbortCallMacro}
\end{listing}




\begin{listing}[H]
\begin{minted}[fontsize=\small]{lexer.py:CoreASMLexer -x}
rule InitializeMacroArguments(MI, currentStateNumber, mIDNew, MINew, macroArgumentsValues) = {
  local
    listres1 := macroArguments(processIDFor(self), mIDNew),
    listres2 := macroArgumentsValues in
  {
    if (|listres1| != |listres2|) then {
      debuginfo CallMacro self + ": Macro '"+macroID(processIDFor(self), mIDNew)+"' takes " + |listres1| + " arguments, but " + |listres2| + " given: " + listres2
      Crash()
    }

    while (|listres1| > 0) do {
      let varnameDst = head(listres1),
        varnameSrc = head(listres2) in
      let var = loadVar(MI, varnameSrc) in
      {
        if (var = undef) then {
          debuginfo CallMacro self + ": skipped local variable '" + varnameDst + "' from '" + varnameSrc + "' as its undef"
        }
        else {
          debuginfo CallMacro self + ": load local variable '" + varnameDst + "' from '" + varnameSrc + "': '" + var + "'"
          SetVar(MINew, varnameDst, nth(var, 1), nth(var, 2))
        }
      }

      listres1 := tail(listres1)
      listres2 := tail(listres2)
    }
  }
}
\end{minted}
\caption{InitializeMacroArguments}
\label{lst:asm:InitializeMacroArguments}
\end{listing}




\begin{listing}[H]
\begin{minted}[fontsize=\small]{lexer.py:CoreASMLexer -x}
rule CallMacro(MI, currentStateNumber, args) = {
  let childInstance = callMacroChildInstance(channelFor(self), MI, currentStateNumber) in {
    // if Macro is not yet running..
    if (childInstance = undef) then {
      // TODO: consider to move this to a new rule StartCallMacro
      let mIDNew = searchMacro(head(args)),
        MINew  = nextMacroInstanceNumber(channelFor(self)) in seqblock
        nextMacroInstanceNumber(channelFor(self)) := MINew + 1
        macroNumberOfMI(channelFor(self), MINew) := mIDNew
        callMacroChildInstance(channelFor(self), MI, currentStateNumber) := MINew

        // NOTE: macroTerminationResult doesn't need to be initialized
        // as the MI part will be different in each iteration

        // if the Macro has parameters..
        if (|macroArguments(processIDFor(self), mIDNew)| > 0) then
        {
          InitializeMacroArguments(MI, currentStateNumber, mIDNew, MINew, tail(args))
        }

        debuginfo CallMacro self + ": DONE, make global update"
        SetExecutionState(MI, currentStateNumber, 1)

        StartMacro(MI, currentStateNumber, mIDNew, MINew)
      endseqblock
    }
    else {
      debuginfo CallMacro self + ": childInstance: " + childInstance
      let childResult = macroTerminationResult(channelFor(self), childInstance) in {
        if (childResult != undef) then {
          debuginfo CallMacro self + ": childResult: " + childResult

          callMacroChildInstance(channelFor(self), MI, currentStateNumber) := undef

          if (childResult = true) then { // completed without result
            SetCompletedAction(MI, currentStateNumber, undef) // sets executionState to REPEAT
          }
          else {
            SetCompletedAction(MI, currentStateNumber, childResult) // sets executionState to REPEAT
          }
        }
        else seqblock
          debuginfo CallMacro self + ": execute MacroBehaviour("+childInstance+") with activeStates(ch, "+childInstance+"): " + activeStates(channelFor(self), childInstance)

          MacroBehaviour(childInstance)

          let state = macroExecutionState(channelFor(self), childInstance) in {
            debuginfo CallMacro self + ": macroExecutionState(ch, "+childInstance+"): " + macroExecutionState(channelFor(self), childInstance)

            if (state = 1) then { // DONE
              SetExecutionState(MI, currentStateNumber, 1)
            }
            else if (state = 2) then { // NEXT
              SetExecutionState(MI, currentStateNumber, 2)
            }
            else if (state = 3) then { // LOWER
              SetExecutionState(MI, currentStateNumber, 3)
            }
            else {
              debuginfo CallMacro self + ": invalid macroExecutionState!"
              Crash()
            }
          }

          // reset
          macroExecutionState(channelFor(self), childInstance) := undef
        endseqblock
      }
    }
  }
}
\end{minted}
\caption{CallMacro}
\label{lst:asm:CallMacro}
\end{listing}




\begin{listing}[H]
\begin{minted}[fontsize=\small]{lexer.py:CoreASMLexer -x}
rule CheckCancel(MI, currentStateNumber, transitionNumber) = {
  let processID = processIDFor(self) in
  let tName = transitionLabel(processID, transitionNumber) in
  let nCancel = stateNumberFromID(processID, tName) in {
    if (contains(activeStates(channelFor(self), MI), nCancel) = true) then {
      debuginfo CheckCancel self + ": at least one state active!" // at least? if nothing went wrong it should be at most one as cancel on modal join makes no sense..
      EnableTransition(MI, transitionNumber)
    }
    else {
      debuginfo CheckCancel self + ": currently no state active!"
      DisableTransition(MI, currentStateNumber, transitionNumber)
    }
  }
}
\end{minted}
\caption{CheckCancel}
\label{lst:asm:CheckCancel}
\end{listing}




\begin{listing}[H]
\begin{minted}[fontsize=\small]{lexer.py:CoreASMLexer -x}
rule Cancel(MI, currentStateNumber, args) = {
  let processID = processIDFor(self) in
  seqblock
    forall transitionNumber in outgoingNormalTransitions(processID, currentStateNumber) do {
      CheckCancel(MI, currentStateNumber, transitionNumber)
    }

    let enabledOutgoingTransitions = outgoingEnabledTransitions(channelFor(self), MI, currentStateNumber) in {
      if (|enabledOutgoingTransitions| > 0) then {
        seqblock
        debuginfo Cancel self + ": at least one transition with active states :)"

        if (|enabledOutgoingTransitions| = 1) then {
          let transitionNumber = firstFromSet(enabledOutgoingTransitions) in {
            if (transitionIsAuto(processID, transitionNumber) = true) then {
              debuginfo Cancel self + ": making automatic decision for transition " + transitionPretty(MI, transitionNumber)
              selectedTransition(channelFor(self), MI, currentStateNumber) := transitionNumber
            }
            else {
              debuginfo Cancel self + ": can not make automatic decision, not an auto transition: " + transitionPretty(MI, transitionNumber)
            }
          }
        }
        else {
          debuginfo Cancel self + ": can not make automatic decision, too much transitions: " + enabledOutgoingTransitions
        }

        if (selectedTransition(channelFor(self), MI, currentStateNumber) != undef) then {
          debuginfo Cancel self + ": the decision has been made for: " + transitionPretty(MI, selectedTransition(channelFor(self), MI, currentStateNumber))

          SetCompletedAction(MI, currentStateNumber, transitionLabel(processID, selectedTransition(channelFor(self), MI, currentStateNumber))) // sets executionState to REPEAT
        }
        else {
          SelectTransition(MI, currentStateNumber)
        }
        endseqblock
      }
      else {
        debuginfo Cancel self + ": no transition with active states, trying later.."
        debuginfo Cancel self + ": allow all other states to be executed, especially states with lower priority"
        SetExecutionState(MI, currentStateNumber, 3)
        // TODO: discuss: LOWER or NEXT?
      }
    }
  endseqblock
}
\end{minted}
\caption{Cancel}
\label{lst:asm:Cancel}
\end{listing}




\begin{listing}[H]
\begin{minted}[fontsize=\small]{lexer.py:CoreASMLexer -x}
rule PerformTransitionCancel(MI, currentStateNumber, transitionNumber) = {
  let processID = processIDFor(self) in {
    let tLabel = transitionLabel(processID, transitionNumber) in
    let nCancel = stateNumberFromID(processID, tLabel) in {
      cancelDecision(channelFor(self), MI, nCancel) := true

      SetCompletedTransition(MI, currentStateNumber, transitionNumber) // sets executionState to REPEAT
    }
  }
}
\end{minted}
\caption{PerformTransitionCancel}
\label{lst:asm:PerformTransitionCancel}
\end{listing}






\begin{listing}[H]
\begin{minted}[fontsize=\small]{lexer.py:CoreASMLexer -x}
// no wildcards allowed
rule CloseIP(MI, currentStateNumber, args) = {
  let senderSubjID         = nth(args, 1),
      messageType          = nth(args, 2),
      correlationIDVarname = nth(args, 3) in {
      if (messageType = "*" or messageType = "?" or senderSubjID = "*" or senderSubjID = "?" or correlationIDVarname = "*" or correlationIDVarname = "?") then {
        debuginfo CloseIP self + ": no wildcards allowed. You may want to use CloseAllIPs"
        Crash()
      }
      else {
        let correlationID = loadCorrelationID(MI, correlationIDVarname) in {
          inputPoolClosed(channelFor(self), senderSubjID, messageType, correlationID) := true
          if (inputPool(channelFor(self), senderSubjID, messageType, correlationID) = undef) then {
              add [senderSubjID, messageType, correlationID] to inputPoolDefined(channelFor(self))
              inputPool(channelFor(self), senderSubjID, messageType, correlationID) := []
          }
        }
      }

      SetCompletedAction(MI, currentStateNumber, undef) // sets executionState to REPEAT
  }
}
\end{minted}
\caption{CloseIP}
\label{lst:asm:CloseIP}
\end{listing}




\begin{listing}[H]
\begin{minted}[fontsize=\small]{lexer.py:CoreASMLexer -x}
// no wildcards allowed
rule OpenIP(MI, currentStateNumber, args) = {
  let senderSubjID         = nth(args, 1),
      messageType          = nth(args, 2),
      correlationIDVarname = nth(args, 3) in {
      if (messageType = "*" or messageType = "?" or senderSubjID = "*" or senderSubjID = "?" or correlationIDVarname = "*" or correlationIDVarname = "?") then {
        debuginfo OpenIP self + ": no wildcards allowed. You may want to use CloseAllIPs"
        Crash()
      }
      else {
        let correlationID = loadCorrelationID(MI, correlationIDVarname) in {
          inputPoolClosed(channelFor(self), senderSubjID, messageType, correlationID) := false
          if (inputPool(channelFor(self), senderSubjID, messageType, correlationID) = undef) then {
              add [senderSubjID, messageType, correlationID] to inputPoolDefined(channelFor(self))
              inputPool(channelFor(self), senderSubjID, messageType, correlationID) := []
          }
        }
      }

      SetCompletedAction(MI, currentStateNumber, undef) // sets executionState to REPEAT
  }
}
\end{minted}
\caption{OpenIP}
\label{lst:asm:OpenIP}
\end{listing}




\begin{listing}[H]
\begin{minted}[fontsize=\small]{lexer.py:CoreASMLexer -x}
rule CloseAllIPs(MI, currentStateNumber, args) = {
  inputPoolClosed(channelFor(self), undef, undef, undef) := true

  forall key in inputPoolDefined(channelFor(self)) do {
      let sID = nth(key, 1),
        mT  = nth(key, 2),
        cID = nth(key, 3) in {
        inputPoolClosed(channelFor(self), sID, mT, cID) := true
      }
  }

  SetCompletedAction(MI, currentStateNumber, undef) // sets executionState to REPEAT
}
\end{minted}
\caption{CloseAllIPs}
\label{lst:asm:CloseAllIPs}
\end{listing}




\begin{listing}[H]
\begin{minted}[fontsize=\small]{lexer.py:CoreASMLexer -x}
rule OpenAllIPs(MI, currentStateNumber, args) = {
  inputPoolClosed(channelFor(self), undef, undef, undef) := false

  forall key in inputPoolDefined(channelFor(self)) do {
      let sID = nth(key, 1),
        mT  = nth(key, 2),
        cID = nth(key, 3) in {
        inputPoolClosed(channelFor(self), sID, mT, cID) := false
      }
  }

  SetCompletedAction(MI, currentStateNumber, undef) // sets executionState to REPEAT
}
\end{minted}
\caption{OpenAllIPs}
\label{lst:asm:OpenAllIPs}
\end{listing}




\begin{listing}[H]
\begin{minted}[fontsize=\small]{lexer.py:CoreASMLexer -x}
// only correlation can be wildcard (*)
rule IsIPEmpty(MI, currentStateNumber, args) = {
  debuginfo IsIPEmpty self + ": args: " + args

  local numres in
  let senderSubjID         = nth(args, 1),
      messageType          = nth(args, 2),
      correlationIDVarname = nth(args, 3) in
  seqblock

      if (correlationIDVarname = undef or correlationIDVarname = 0 or correlationIDVarname = "") then {
        numres := 0
      }
      else if (correlationIDVarname = "*") then {
        numres := undef
      }
      else if (correlationIDVarname = "?") then {
        debuginfo OpenIP self + ": correlationIDVarname must not be '?'. wildcard is '*'"
        Crash()
      }
      else {
        numres := loadCorrelationID(MI, correlationIDVarname)
      }

      // receiverChannel * senderSubjID * messageType * correlationID
      if (inputPoolIsEmpty(channelFor(self), senderSubjID, messageType, numres) = true) then {
        SetCompletedAction(MI, currentStateNumber, "true") // sets executionState to REPEAT
      }
      else {
        SetCompletedAction(MI, currentStateNumber, "false") // sets executionState to REPEAT
      }
  endseqblock
}
\end{minted}
\caption{IsIPEmpty}
\label{lst:asm:IsIPEmpty}
\end{listing}




\begin{listing}[H]
\begin{minted}[fontsize=\small]{lexer.py:CoreASMLexer -x}
// Channel * MacroInstanceNumber * StateNumber -> BOOLEAN
function selectAgentsDecision : LIST * NUMBER * NUMBER -> SET

function selectAgentsProcessID : LIST * NUMBER * NUMBER -> STRING
function selectAgentsSubjectID : LIST * NUMBER * NUMBER -> STRING
function selectAgentsCountMin  : LIST * NUMBER * NUMBER -> NUMBER
function selectAgentsCountMax  : LIST * NUMBER * NUMBER -> NUMBER

function selectAgentsResult : LIST * NUMBER * NUMBER -> SET

rule SelectAgentsAction(MI, currentStateNumber, args) = {
  let
    varname  = nth(args, 1),
    sIDLocal = nth(args, 2),
    countMin = nth(args, 3),
    countMax = nth(args, 4) in
  {
    if (selectAgentsResult(channelFor(self), MI, currentStateNumber) != undef) then {
      SetVar(MI, varname, "ChannelInformation", selectAgentsResult(channelFor(self), MI, currentStateNumber))
      selectAgentsResult(channelFor(self), MI, currentStateNumber) := undef

      SetCompletedAction(MI, currentStateNumber, undef) // sets executionState to REPEAT
    }
    else {
      SelectAgents(MI, currentStateNumber, sIDLocal, countMin, countMax)
    }
  }
}
\end{minted}
\caption{SelectAgentsAction}
\label{lst:asm:SelectAgentsAction}
\end{listing}




\begin{listing}[H]
\begin{minted}[fontsize=\small]{lexer.py:CoreASMLexer -x}
rule SelectAgents(MI, currentStateNumber, sIDLocal, countMin, countMax) = {
  if (sIDLocal = undef or sIDLocal = "?") then {
    debuginfo SelectAgents self + ": sIDLocal must not be wildcard/undef"
    Crash()
  }

  let processID = processIDFor(self),
      PI        = processInstanceFor(self) in
  let resolvedInterface = resolveInterfaceSubject(sIDLocal) in
  let resolvedProcessID = nth(resolvedInterface, 1),
      resolvedSubjectID = nth(resolvedInterface, 2) in
  if (selectAgentsDecision(channelFor(self), MI, currentStateNumber) != undef) then {
      // validate min/max
      if (hasSizeWithin(selectAgentsDecision(channelFor(self), MI, currentStateNumber), countMin, countMax) != true) then {
        debuginfo SelectAgents self + ": selectAgentsDecision is not within countMin/countMax: " + countMin + " <= " + |selectAgentsDecision(channelFor(self), MI, currentStateNumber)| + " <= " + countMax
        Crash()
      }

      local setres1 := {} in // created channels
        seq
          foreach agent in selectAgentsDecision(channelFor(self), MI, currentStateNumber) do { // note: forall not possible as nextPI is incremented
              if (resolvedProcessID = processID) then {
                // local process, use own PI
                let ch = [processID, PI, sIDLocal, agent] in {
                  InitializeSubject(ch)

                  add ch to setres1
                }
              }
              else {
                // external process, create new PI
                local numres1 in {
                  seq
                      numres1 <- StartProcess(resolvedProcessID, resolvedSubjectID, agent)
                  next {
                      let ch = [resolvedProcessID, numres1, resolvedSubjectID, agent] in
                        add ch to setres1
                  }
                }
              }
          }
        next
          selectAgentsResult(channelFor(self), MI, currentStateNumber) := setres1

      selectAgentsDecision (channelFor(self), MI, currentStateNumber) := undef

      selectAgentsCountMin (channelFor(self), MI, currentStateNumber) := undef
      selectAgentsCountMax (channelFor(self), MI, currentStateNumber) := undef
      selectAgentsProcessID(channelFor(self), MI, currentStateNumber) := undef
      selectAgentsSubjectID(channelFor(self), MI, currentStateNumber) := undef

      debuginfo SelectAgents self + ": REPEAT"
      SetExecutionState(MI, currentStateNumber, 0)
  }
  else if(hasSizeWithin(predefinedAgents(processID, PI, sIDLocal), countMin, countMax) = true) then {
      debuginfo SelectAgents self + ": apply predefinedAgents: " + predefinedAgents(processID, PI, sIDLocal)

      selectAgentsDecision(channelFor(self), MI, currentStateNumber) := predefinedAgents(processID, PI, sIDLocal)

      debuginfo SelectAgents self + ": REPEAT"
      SetExecutionState(MI, currentStateNumber, 0)
  }
  else {
      if not(contains(wantInput(channelFor(self), MI, currentStateNumber), "SelectAgentsDecision")) then {
        add "SelectAgentsDecision" to wantInput(channelFor(self), MI, currentStateNumber)

        selectAgentsProcessID(channelFor(self), MI, currentStateNumber) := resolvedProcessID
        selectAgentsSubjectID(channelFor(self), MI, currentStateNumber) := resolvedSubjectID
        selectAgentsCountMin (channelFor(self), MI, currentStateNumber) := countMin
        selectAgentsCountMax (channelFor(self), MI, currentStateNumber) := countMax

        selectAgentsResult(channelFor(self), MI, currentStateNumber) := undef

        debuginfo SelectAgents self + ": DONE, make global update"
        SetExecutionState(MI, currentStateNumber, 1)
      }
      else {
        debuginfo SelectAgents self + ": waiting for selectAgentsDecision"
        debuginfo SelectAgents self + ": NEXT"
        SetExecutionState(MI, currentStateNumber, 2)
      }
  }
}
\end{minted}
\caption{SelectAgents}
\label{lst:asm:SelectAgents}
\end{listing}


\chapter{PASS ASM Specification with detailed Comments}

In the following you can find a detailed description of the ASM semantic of PASS.

\includepdf[pages=1-51]{"A Subject-Oriented Interpreter Model for S-BPM"}


\backmatter

%\bibliographystyle{IEEEtran}
\bibliography{Standard}

\listoftodos
\listofchanges[style=list, show=all]


\end{document}
