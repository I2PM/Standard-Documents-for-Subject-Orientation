\chapter{Commands for Reciew process}

For managing changes the package change is used. Following commands are essential:\\
\textbf{Mark text without id:}\\
\hl{
	This is the way  text is marked
}\\

\textbf{Add todo withaout id:}\\
Here is some text
\todo{The original todo note withouth changed colours.\newline Here's another line.} Here is some text
\newline
\newline
\newline
\newline
\textbf{Add todo withaout id and inline:}\\
Here is some text
\todo[inline]{The original todo note withouth changed colours.\newline Here's another line.} Here is some text
\newline

\textbf{Add text with id:}\\
\added[id=AF, comment=remarks to added text]{Added text}\\

\textbf{Delete command with id:}\\
\deleted[id=WS, comment=example for delete]{Here is the text to be deleted}
Here is some text\\
\newline
\textbf{Replace text with id:}\newline
Here is some text
\replaced[id=CS, comment=Remarks to Replace text]{new text}{old text}
Here is some text\\

\textbf{highlight text with id:}\\
Here is some text
\highlight[id=AF, comment=remarks]{highlighted text}\\

\textbf{Add todo with fancy line but without id:}\\
Here is some text \todo[color=green!40, fancyline]{The original todo note.\newline Here's another line with a fancy line}
 Here is some text
\newline 
\newline
\newline
\newline 
\newline
\newline
For todo commands ids are not allowed. For the add, highlight and
Documentation for using the changes package can be found:\\
http://ctan.ebinger.cc/tex-archive/macros/latex/contrib/changes/changes.english.pdf
\newline
The ids and colors for the various reviewers can be found in the preamble (line 29-33) of the text file.
